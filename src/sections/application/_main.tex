%!TEX root = ../../../_main.tex
\chapter{Application}
\label{sec:application}

In this section we use our results on the twisted weak ordering to conclude a certain structural property for some geometrical structures. Namely, we introduce chambers systems and a specialization of them, buildings that receive structure by an underlying Coxeter system $(W,S)$. Buildings then yield another generalization, the twin buildings for which we investigate properties of certain involutory maps. These maps, called building (quasi-)flips admit an involutive Coxeter system automorphism for $(W,S)$. The intended application of our results then applies to so-called flip-flop systems admitted by a twin building and a building quasi-flip. This section is heavily based on \cite{horn:kac-moody}, but we will only introduce the bare minimum needed for our purposes. So for more details refer to \cite{horn:kac-moody}.

%!TEX root = ../../../_main.tex
\section{Chamber systems}

\begin{defi}
	\typedlabel{defi:chamber-system}
	A \defword{chamber system over $I$} is a pair $\mathcal{C} = (C,(\sim_i, i \in I))$, with a nonempty set $C$, whose members are called \defword{chambers} and a family of equivalence relations $\sim_i$, indexed by $i \in I$, that satisfies the implication
	$$ c \sim_i d \wedge c \sim_j d \Rightarrow c = d \vee i = j $$
	for all $c,d \in C$ and $i,j \in I$. The cardinality $|I|$ is called the \defword{rank} of $\mathcal{C}$. For all chamber systems we will assume that they have finite rank. If for two chambers $c,d$ we have $c \sim_i d$, then $c$ is called \defword{i-adjacent} to $d$ or just \defword{adjacent}.
\end{defi}

So the main assertion for chamber systems is, that two distinct chambers $c,d \in C$ are at most adjacent by one $i \in I$. For the rest of this section $\mathcal{C} = (C,(\sim_i, i \in I))$ will denote a chamber system.

\begin{exam}
	For an arbitrary Coxeter system let $W$ act as set of chambers and for each generator $s \in S$ define a equivalence relation $w \sim_s v$ if and only if either $w = v$ or $ws = v$. That this are really equivalence relations is easy to check. So suppose $w \sim_s v$, $w \sim_t v$ for two distinct generators $s,t \in S$. The assumption $w \neq v$ immediately yields a contradiction by $ws = v = wt \iff s = t$. Hence this is indeed a chamber system.
\end{exam}

The previous example is just a special case of a quite general recipe to create chamber systems from groups, the so-called coset chamber systems.

\begin{defi}
	\theocite{Defnition 3.6.3}{buekenhout:diagram-geometry}
	Let $G$ be an arbitrary group with a subgroup $B$ and a family of subgroups $(G_i, i \in I)$ such that $B \subseteq G_i$ for $i \in I$. Choose the chamber set $C$ as the set of all $B$-cosets $gB$ for some $g \in G$ and define the equivalence relations $(\sim_i, i \in I)$ by $gB \sim_i hB$ iff $gG_i = hG_i$. Then we call this chamber system the \defword{coset chamber system} of $G$ on $B$ with respect to $(G_i, i \in I)$.
\end{defi}

\begin{lemm}
	Coset chamber systems are chamber systems.

	\begin{proof}
		As easy to check the $\sim_i$ are equivalence relations. So suppose $gB \sim_i hB$ and $gB \sim_j hB$ and let $gB \neq hB$, i.e. $h^{-1}g \notin B$. \todo \ Different definitions of chamber system at Horn and Buekenhout/Cohen?
	\end{proof}
\end{lemm}

If two chambers $c,d \in C$ in a chamber system are not adjacent, then there might be a chain of subsequent adjacent chambers with $c$ as first and $d$ as last chamber.

\begin{defi}
	Let $G = (c_0,\ldots,c_k)$ be a finite sequence of chambers $c_i \in C$ with $c_{i-1}$ adjacent to $c_i$ for all $1 \leq i \leq k$. Then $G$ is called a \defword{gallery} in $\mathcal{C}$ whereas the integer $k$ is called the \defword{length} of $G$. The first element $c_0$ of a gallery $G$ is denoted by $\alpha(G)$ and the last by $\omega(G)$. If for two chambers $c,d \in C$ there is a gallery $G$ with $\alpha(G) = c$ and $\omega(G) = d$, then we say that $G$ \defword{joins} $c$ and $d$. A gallery with G with $\alpha(G) = \omega(G)$ is called \defword{closed} and a gallery $G = (c_0,\ldots,c_k)$ with $c_{i-1} \neq c_i$ for all $1 \leq i \leq k$ is called \defword{simple}. If a gallery $G$ of length $k$ joins two chambers $c,d$ and there is no joining gallery of shorter length, then we call $G$ a \defword{minimal gallery joining $c$ and $d$}.
\end{defi}

Note, that two chambers are adjacent if and only if they can be joined by a gallery of length 1.

\begin{defi}
	The chamber system $\mathcal{C}$ is called \defword{connected} if any two chambers $c,d \in C$ can be joined by a gallery.
\end{defi}

\begin{defi}
	Let $G = (c_0,\ldots,c_k)$ be a gallery and let $J \subset I$ be a subset. If for $1 \leq i \leq k$ there is a $j \in J$ with $c_{i-1} \sim_j c_i$, then we call $G$ a \defword{$J$-gallery}. Two chambers $c,d \in C$, that have a $J$-gallery joining them, are called \defword{$J$-equivalent}, denoted by $c \sim_J d$.
\end{defi}

\begin{defi}
	For a chamber $c \in C$ and a subset $J \subseteq I$, we call the set $R_J(c) := \{ d \in C : c \sim_J d \}$ a \defword{$J$-residue}. The set $J$ is also called the \defword{type} of a residue $R_J(c)$. If $|J| = 1$, say $J = \{i\}$, then $R_J(c) = R_{\{i\}}(c)$ is called a \defword{$i$-panel}.
\end{defi}

Note that for any chamber system $(C,(\sim_i, i \in I))$, $c \in C$ and $J \subseteq I$, the chamber system $(R_J(c), (\sim_j, j \in J))$ is connected by construction.

\begin{defi}
	Let $\mathcal{C}$ be a chamber system over $I$. We call it a \defword{residually connected} chamber system if the following holds: For every $J \subseteq I$ and every family of residues $(R_{I \setminus \{j\}}, j \in J)$ with pairwise nonempty intersection we have
	$$ \bigcap_{j \in J} R_{I \setminus \{j\}} = R_{I \setminus J}(c) $$
	for some $c \in C$.
\end{defi}
%!TEX root = ../../../_main.tex
\subsection{Buildings}

\begin{defi}
	A \defword{building} of type $(W,S)$ is a pair $(\mathcal{C}, \delta)$ with a nonempty set $\mathcal{C}$ and a map $\delta : \mathcal{C} \times \mathcal{C} \to W$, called \defword{distance function}, so that for $x,y \in \mathcal C$ and $w = \delta(x,y)$ we have
	\begin{axioms}
		\item[(Bu1)] $w = e \iff x = y$;
		\item[(Bu2)] for $z \in \mathcal C$ with $\delta(y,z) = s \in S$ we have $\delta(x,z) \in \{w,ws\}$, and if in addition $l(ws) = l(w) + 1$ then we have $\delta(x,z) = ws$;
		\item[(Bu3)] for $s \in S$ there exists a $z \in \mathcal C$ with $\delta(y,z) = s$ and $\delta(x,z) = ws$.
	\end{axioms}
\end{defi}

\begin{defi}
	Let $(\mathcal{C}, \delta)$ be building of type $(W,S)$. Then cardinality of $S$ is called the \defword{rank} of the building.
\end{defi}

\begin{defi}
	Let $(\mathcal{C}, \delta)$ be a building of type $(W,S)$. For each $s \in S$ we define $c,d \in C$ to be $s$-adjacent, if and only iff $\delta(c,d) \in \{e,s\}$. Then $(\mathcal{C}, (\sim_s, s \in S))$ is the to our building $(\mathcal{C}, \delta)$ \defword{associated chamber system}.
\end{defi}

\begin{lemm}
	Let $(\mathcal{C}, \delta)$ be a building of type $(W,S)$. Then the associated chamber system is a chamber system.

	\begin{proof}
		Let $c,d \in \mathcal{C}$ and $s,t \in S$ with $c \sim_s d$ and $c \sim_t d$. If $c \neq d$, then $\delta(c,d) = s$ and $\delta(c,d) = t$, hence $s = t$.
	\end{proof}
\end{lemm}

\begin{defi}
	A building $(\mathcal{C}, \delta)$ of type $(W,S)$ is called \defword{thick} (resp. \defword{thin}), if for every chamber $c \in \mathcal{C}$ and every $s \in S$ there are at least three (resp. exactly two) chambers $s$-adjacent to $c$.
\end{defi}

\begin{exam}
	\typedlabel{exam:thin-building}
	For a Coxeter system $(W,S)$ define a map
	$$ \delta_S : W \times W \to W : (x,y) \mapsto x^{-1}y. $$
	Then $\delta_S(x,y) = e \iff x = y$. Furthermore for $z \in W$ with $\delta_S(y,z) = s$, i.e. $z = ys$, we have $\delta_S(x,z) = x^{-1}z = x^{-1}ys = \delta(x,y)s$. For $s \in S$ and $x,y \in W$ choose $z = ys$. Then $\delta_S(y,z) = s$ and as before $\delta_S(x,z) = \delta_S(x,y)s$. Hence $(W,\delta_S)$ is a building of type $(W,S)$. More precisely, it is a thin building, since for every $s \in S$ and $x,y \in W$ we have $\delta_S(x,y) = x^{-1}y \in \{e,s\}$ if and only if $x = y$ or $y = xs$, hence there are excatly two chambers $s$-adjacent to $x$.
\end{exam}

This example for a thin building of type $(W,S)$ can be indeed called "the" thin building of type $(W,S)$ as the following lemma shows.

\begin{lemm}
	\theocite{Theorem 4.2.8}{buekenhout:diagram-geometry}
	Let $(\mathcal{C}, \delta)$ be a thin building of type $(W,S)$. Then it is isometric to the building $(W, \delta_S)$ (cf. \ref{exam:thin-building}).
\end{lemm}

\begin{defi}
	For a building $(\mathcal{C},\delta)$ of type $(W,S)$ we call a subset $\Sigma \subseteq \mathcal{C}$ an \defword{appartment}, if $(\Sigma, \delta|_\Sigma)$ is isometric to $(W,\delta_S)$ from \ref{exam:thin-building}, or equivalent if $(\sigma, \delta|_\Sigma)$ is thin.
\end{defi}

\begin{theo}
	\theocite{Theorem 11.2.5}{buekenhout:diagram-geometry}
	Let $(\mathcal{C},\delta)$ be a building. Then for any two chambers $c,d \in \mathcal{C}$ there is an appartment $\Sigma$ with $c,d \in \Sigma$. In particular every building contains at least one appartment.

	\begin{proof}
		The proof for the first statement can be found in \cite[Theorem 11.2.5]{buekenhout:diagram-geometry}. The second is an immediate conclusion of the first, since because of $|S| \geq 1$ and the third building axiom every building must at least contain two chambers. And so there is at least one pair of chambers, that has to be contained in an appartment by the first statement.
	\end{proof}
\end{theo}

So thin buildings are precisely those, that contain excatly one apparment, i.e. are appartments themself.

\begin{defi}
	A building $(\mathcal{C},\delta)$ of type $(W,S)$ is called \defword{spherical} if $W$ is finite. In this case $W$ has a longest element $w_0$ and two chambers $c,d$ are called \defword{opposite}, if $\delta(c,d) = w_0$.
\end{defi}

\begin{defi}
	Let $(\mathcal{C},\delta)$ be a building of type $(W,S)$. A set of chambers $M \subseteq \mathcal{C}$ is called \defword{connected}, if any two chambers in $M$ can be joined by a gallery completeley contained in $M$. If in addition, every minimal gallery joining two chambers in $M$ is completeley contained in $M$, then $M$ is called \defword{convex}.
\end{defi}
%!TEX root = ../../../_main.tex
\subsection{Twin buildings}

Twin buildings 
%!TEX root = ../../../_main.tex
\section{Building flips and flip-flop systems}
In this section let $\mathcal C = (\mathcal C_+, \mathcal C_-, \delta^*)$ be a twin building of type $(W,S)$.

\begin{defi}
	Let $\tilde \theta$ be a permutation of $\mathcal C_+ \cup \mathcal C_-$ satisfying
	\begin{axioms}
		\item[(Fl1)] \label{axiom:fl1} $\tilde \theta^2 = \id$,
		\item[(Fl2)] \label{axiom:fl2} $\tilde \theta(\mathcal C_+) = \mathcal C_-$ and
		\item[(Fl3)] \label{axiom:fl3} for $\varepsilon \in \{+,-\}, x,y \in \mathcal C_+$ and $z \in \mathcal C_-$ we have $x \sim y$ iff $\tilde \theta(x) \sim \tilde \theta(y)$ and $x \opp z$ iff $\tilde \theta(x) \opp \tilde \theta(z)$.
	\end{axioms}
	Then we call $\tilde \theta$ a \defword{building quasi-flip} of $\mathcal{C}$. If in addition
	\begin{axioms}
		\item[(Fl3')] \label{axiom:fl3'} for $\varepsilon \in \{+,-\}, x,y \in \mathcal C_+$ and $z \in \mathcal C_-$ we have $\delta_\varepsilon(x,y) = \delta_{-\varepsilon}(\tilde \theta(x),\tilde \theta(y))$ and $\delta^*(x,z) = \delta^*(\tilde \theta(x),\tilde \theta(y))$, 
	\end{axioms}
	then we call $\tilde \theta$ a \defword{building flip} of $\mathcal C$.
\end{defi}

So building (quasi-)flips permute the two halfes of a twin building while preserving adjacency and opposition and building flips also flip the distance and preserver the codistance. The next lemma gives a first idea, how building quasi-flips are coherent to the poset $Wk(\theta)$.

\begin{lemm}
	\theocite{Lemma 2.1.4}{horn:kac-moody}
	Let $\tilde \theta$ be a building quasi-flip of $\mathcal C$. Then $\tilde \theta$ induces an involutory (i.e. order at most 2) Coxeter system automorphism $\theta$ on $(W,S)$, so that for $\varepsilon \in \{+,-\}, x,y \in \mathcal C_+$ and $z \in \mathcal C_-$ we have $\theta(\delta_\varepsilon(x,y)) = \delta_{-\varepsilon}(\tilde \theta(x), \tilde \theta(y))$ and $\theta(\delta^*(x,z)) = \delta^*(\tilde \theta(x), \tilde \theta(z))$.
\end{lemm}

Of course the coherence between building quasi-flips and $Wk(\theta)$ is not clear by any means, but at least do building quasi-flips admit a Coxeter system and an involutory Coxeter system automorphism, hence every building quasi-flip has a corrosponding twisted weak ordering poset $Wk(W,\theta)$. But there are some definitions left until we have our objects of interest.

\begin{defi}
	For a chamber $c \in \mathcal C_+ \cup \mathcal C_-$ we call $\delta^{\tilde \theta}(c) := \delta^*(c,\tilde \theta(c))$ the \defword{$\tilde \theta$-codistance} of $c$ and $l^{\tilde \theta}(c) = l(\delta^{\tilde \theta}(c))$ the \defword{numerical $\theta$-codistance} of $c$.
\end{defi}

\begin{defi}
	Let $\tilde \theta$ be a building quasi-flip of $\mathcal C$ and let $R \subseteq \mathcal C_+$ be an arbitrary residue. The \defword{minimal numerical $\tilde \theta$-codistance} of $R$ is defined as $\min_{c \in R} l^{\tilde \theta}(c)$.
\end{defi}

According to the definition of $c_+ \opp d_-$, i.e. $l(\delta^*(c_+,d_-)) = 0$, we can consider the chambers with minimal numerical $\tilde \theta$-codistance as those, that are mapped away "as far as possible".

\begin{defi}
	Let $\tilde \theta$ be a building quasi-flip of $\mathcal C$ and let $R \subseteq \mathcal C_+$ be an arbitrary residue. The (sub)chamber system of all chambers with minimal numerical $\tilde \theta$-codistance
	$$ R^{\tilde \theta} := \{ c \in R : l^{\tilde \theta}(c) = \min_{d \in R} l^{\tilde \theta}(d) \} $$
	together with the equivalence relations inherited from $\mathcal C_+$ is called the \defword{induced flip-flop system} on $R$. In case $R = \mathcal C_+$, we call $C^{\tilde \theta} := C_+^{\tilde \theta} = R^{\tilde \theta}$ the \defword{flip-flop system} associated to $\tilde \theta$.
\end{defi}

In particular, if $\min_{c \in R} l^{\tilde \theta}(c) = 0$, then $R^{\tilde \theta} = \{ c \in R : c \opp \tilde \theta(c) \}$.