\subsection{Bruhat ordering}

We now investige ways to partially order the elements of a Coxeter group. Futhermore this ordering should be compatible with the length function. The most useful way to achieve this is the Bruhat ordering \cite[Section 5.9]{humphreys:coxeter}.

\begin{defi}
	Let $(W,S)$ be a Coxeter system and $T = \cup_{w \in W} wSw^{-1}$ the set of all reflections in $W$. We write $w' \to w$ if there is a $t \in T$ with $w't = w$ and $l(w') < l(w)$. If there is a sequence $w' = w_0 \to w_1 \to \ldots \to w_m = w$ we say $w' < w$. The resulting relation $w' \leq w$ is called \defword{Bruhat ordering}.
\end{defi}

\begin{lemm}
	Let $(W,S)$ be a Coxeter system. Then $W$ together with the Bruhat ordering is a poset.

	\begin{proof}
		The Bruhat ordering is reflexive by definition. Since the elements in sequences $e \to w_1 \to w_2 \to \ldots$ are strictly ascending in length, it must be antisymmetric. By concatenation of sequences we get the transitivity.
	\end{proof}
\end{lemm}

What we really want is the Bruhat ordering to be graded with the length function as rank function. By definition we already have $v < w$ iff $l(v) < l(w)$, but its not that obvious that two immediately adjacent elements differ in length by excatly 1. Before lets just mention two other partial orderings, where this property is obvious by definition:

\begin{defi}
	Let $(W,S)$ be a Coxeter system. The ordering $\leq_R$ defined by $u \leq_R w$ iff $uv = w$ for some $u \in W$ with $l(u) + l(v) = l(w)$ is called the \defword{right weak ordering}. The left sided version $u \leq_L w$ iff $vu = w$ is called the \defword{left weak ordering}.
\end{defi}