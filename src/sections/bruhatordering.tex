\subsection{Bruhat ordering}
\label{sec:bruhat-ordering}

We now investigate ways to partially order the elements of a Coxeter group. Futhermore this ordering should be compatible with the length function. The most useful way to achieve this is the Bruhat ordering \cite[Section 5.9]{humphreys:coxeter}.

\begin{defi}
	Let $M$ be a set. A binary relation, in this case often denoted as ``$\leq$'', is called a \defword{partial order} over $M$, if fullfills the following conditions for all $a,b,c \in M$:
	\begin{enumerate}
		\item $a \leq a$, called \defword{reflexivity}
		\item if $a \leq b$ and $b \leq a$ then $a=b$, called \defword{antisymmetry}
		\item if $a \leq b$ and $b \leq c$ then $a \leq c$, called \defword{transitivity}
	\end{enumerate}
	In this case $(M,\leq)$ is called a \defword{poset}. If two elements $a \leq b \in M$ are immediate neighbours, i.e. there is no third element $c \in M$ with $a \leq c \leq b$ we say that $b$ \defword{covers} $a$. A poset is called \defword{graded poset} if there is a map $\rho : M \to \nn$ so that $\rho(b) - 1 = \rho(ab)$ whenever $b$ covers $a$.
\end{defi}

\begin{defi}
	Let $(W,S)$ be a Coxeter system and $T = \cup_{w \in W} wSw^{-1}$ the set of all reflections in $W$. We write $w' \to w$ if there is a $t \in T$ with $w't = w$ and $l(w') < l(w)$. If there is a sequence $w' = w_0 \to w_1 \to \ldots \to w_m = w$ we say $w' < w$. The resulting relation $w' \leq w$ is called \defword{Bruhat ordering}, denoted as $\Br(W)$.
\end{defi}

\begin{lemm}
	Let $(W,S)$ be a Coxeter system. Then $\Br(W)$ is a poset.

	\begin{proof}
		The Bruhat ordering is reflexive by definition. Since the elements in sequences $e \to w_1 \to w_2 \to \ldots$ are strictly ascending in length, it must be antisymmetric. By concatenation of sequences we get the transitivity.
	\end{proof}
\end{lemm}

What we really want is the Bruhat ordering to be graded with the length function as rank function. By definition we already have $v < w$ iff $l(v) < l(w)$, but its not that obvious that two immediately adjacent elements differ in length by exactly 1. Before lets just mention two other partial orderings, where this property is obvious by definition:

\begin{defi}
	Let $(W,S)$ be a Coxeter system. The ordering $\leq_R$ defined by $u \leq_R w$ iff $uv = w$ for some $u \in W$ with $l(u) + l(v) = l(w)$ is called the \defword{right weak ordering}. The left sided version $u \leq_L w$ iff $vu = w$ is called the \defword{left weak ordering}.
\end{defi}

So lets ensure that the Bruhat ordering is graded as well. For this we need another characterization of the Bruhat ordering with subexpressions. As we will show it is $u \leq w$ iff there is a reduced expression of $u$ that is a subexpression of a reduced expression of $w$.

\begin{prop}
	Let $(W,S)$ be a Coxeter system, $u,w \in W$ with $u \leq w$ and $s \in S$. Then $us \leq w$ or $us \leq ws$ or both.

	\begin{proof}
		We can reduce the proof (WHY?) to the case $u \to w$, i.e. $ut = w$ for a $t \in T$ with $l(v) < l(u)$. Let $s = t$. Then $us \leq w$ and we are done. In case $s \neq t$ there are two alternatives for the lengths. We can have $l(us) = l(u) - 1$ which would mean $us \to u \to w$, so $us \leq w$.

		So assume $l(us) = l(u) + 1$. For the reflection $t' = sts$ we get $(us)t' = ussts = uts = ws$. So it is $us \leq ws$ iff $l(us) < l(ws)$. Assume this is not the case. Since we have assumed $l(us) = l(u) + 1$ any reduced expression $u = s_1 \cdots s_r$ for $u$ yields a reduced expression $us = s_1 \cdots s_r s$ for $us$. With the Strong Exchange Condition we can obtain $ws = ust'$ from $us$ by omitting one factor. This omitted factor cannot be $s$ since $s \neq t$. This means $ws = s_1 \cdots \hat s_i \cdots s_r s$ and so $ws = s_1 \cdots \hat s_i \cdots s_r$, contradicting to our assumption $l(u) < l(w)$
	\end{proof}
\end{prop}