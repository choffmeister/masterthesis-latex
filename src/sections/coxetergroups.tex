\section{Coxeter groups}

\subsection{Introduction to Coxeter groups}

A Coxeter group, named after Harold Scott MacDonald Coxeter, is an abstract group generated by involutions with specific relations between these generators. A simple class of a Coxeter groups are the symmetry groups of regular polyhedras in the Euclidean space.

The symmetry group of the square for example can be generated by two reflections $s,t$, whose stabilized hyperplanes enclose an angle of $\pi / 4$. In this case the map $st$ is a rotation in the plane by $\pi / 2$. So we have $s^2 = t^2 = (st)^4 = \id$. In fact this reflection group is determined up to isomorphy by $s,t$ and these three relations \cite[Theorem 1.9]{humphreys:coxeter}. Furthermore it turns out, that the finite reflection groups in the Euclidean space are precisely the finite Coxeter groups \cite[Theorem 6.4]{humphreys:coxeter}.

In this chapter we will compile some basic facts on Coxeter groups. First of all the definition:

\begin{defi}[Coxeter system]
	\label{coxeter-system}
	Let $S = \{ s_1, \ldots, s_n \}$ be a finite set of symbols and
	$$R = \{ m_{ij} \in \nn \cup \infty : 1 \leq i,j \leq n \}$$
	a set numbers (or $\infty$) with $m_{ii} = 1$, $m_{ij} > 1$ for $i \neq j$ and $m_{ij} = m_{ji}$. Then the free represented group
	$$W = \langle S \ | \ (s_i s_j)^{m_{ij}} \rangle$$
	is called a \defword{Coxeter group} and $(W,S)$ the corrosponding \defword{Coxeter system}. The cardinality of $S$ is called the \defword{rank} of the Coxeter system (and the Coxeter group).
\end{defi}

From the definiton we see, that Coxeter groups only depend on the cardinality of $S$ and the relations between the generators in $S$. A common way to visualize this information are \defword{Coxeter graphs}. A Coxeter graph is a graph containing a vertex for each generator in $S$. Let $(s_i s_j)^m = 1$. In case $m = 2$ the two corrosponding vertices have no connecting edge. In case $m = 3$ they are connected by an unlabed edge. For $m > 3$ they have an connecting edge with label $m$.

For a arbitrary element $w \in W$, $(W,S)$ a Coxeter system, we call a product $s_{i_1} \cdots s_{i_n} = w$ of generators $s_{i_1} \ldots s_{i_n} \in S$ an \defword{expression} of $w$. The present relations between the generators of a Coxeter group allow us to rewrite expressions. Hence an element $w \in W$ can have more than one expression. Obviously any element $w \in W$ has infinitly many expressions, since any expression $s_{i_1} \cdots s_{i_n} = w$ can be extended by applying $s_1^2 = 1$ from the right. But there must be a smallest number of generators needed to receive $w$. For example the neutral element $e$ can be expressed by the empty expression. Or each generator $s_i \in S$ can be expressed by itself, but any expression with less factors (i.e. the empty expression) is unequal to $s_i$.

\begin{defi}[Length function]
	\label{length-function}
	Let $(W,S)$ be a Coxeter system and $w \in W$ an element. Then there are some (not neseccarily distince) generators $s_i \in S$ with $s_1 \cdots s_r = w$. We call $r$ the \defword{expression length}. The smallest number $r \in \nn_0$ for that $w$ has an expression of length $r$ is called the \defword{length} of $w$ and each expression of $w$, that is ob minimal length, is called \defword{reduced expression}. The map
	$$ l : W \to \nn_0 $$
	that maps each element in $W$ to its length is calls \defword{length function}.
\end{defi}

\subsection{Exchange and Deletion Condition}

We now obtain a way to get a reduced expression of an arbitrary element $s_1 \cdots s_r = w \in W$. But first we define what a reflection is. Any element $w \in W$ that is conjugated to an generator $s \in S$ is called \defword{reflection}. Hence the set of all reflections in $W$ is
$$ T = \bigcup_{w \in W} wSw^{-1}. $$

\begin{theo}
	Let $(W,S)$ be a Coxeter system, $w \in W$ an arbitrary element and ${s_1 \cdots s_r = w}$ with $s_i \in S$ a not neseccarily reduced expression for $w$. For each reflection $t \in T$ with $l(wt) < l(w)$ there exists an index $i$ for which $wt = s_1 \cdots \hat s_i \cdots s_r$, where $\hat s_i$ means omission. In case we started from an reduced expression, then $i$ is unique.

	\begin{proof}
		See \cite[Theorem 5.8]{humphreys:coxeter}.
	\end{proof}
\end{theo}

This theorem is called the \defword{Strong Exchange Condition}. The same theorem can be stated for $t \in S$ (since $S \subset T$). This weaker theorem is called \defword{Exchange Condition}. The Exchange Condition immediatly yields another corollary for Coxeter groups:

\begin{coro}
	Let $(W,S)$ be a Coxeter system, $w \in W$ and $w = s_1 \cdots s_r$ with $s_i \in S$ a unreduced expression of $w$. Then there exist two indices $i,j \in \{1,\cdots,r\}$ with $i < j$, such that $w = s_1 \cdots \hat s_i \cdots \hat s_j \cdots s_r$, where $\hat s_i$ and $\hat s_j$ mean omission.

	\begin{proof}
		Since the expression is unreduced there must be an index $j$ for that the twisted length shrinks. That means for $w' = s_1 \cdots s_{j-1}$ is $l(w' s_j) < l(w')$. Using the Exchange Condition we get $w' s_j = s_1 \cdots \hat s_i \cdots s_{j-1}$ yielding $w = s_1 \cdots \hat s_i \cdots \hat s_j \cdots s_r$.
	\end{proof}
\end{coro}

This corollary is called \defword{Deletion Condition} and allows us to reduce expressions. To explain what is exactly meant by this we need another definition. For an expression $w = s_1 \cdots s_r$ and any subset of indices $S' \subset S$ we call $w' = t_1 \cdots t_r$ with $t_i = \hat s_i$ for $i \in S'$ and $t_i = s_i$ for $i \notin S'$ a \defword{subexpression} of $w$. With this definition we can precisily define the action of reducing expressions. Reducing an unreduced expression means to extract a reduced subexpression. Due to the Deletion Condition any unreduced expression can be reduced by omitting a even number of generators (we just have to apply the Deletion Condition inductively).

Both, the Exchange Condition and the Deletion Condition, are two of the most powerful tools when investigating properties of Coxeter groups.

\subsection{Finite Coxeter groups}

Coxeter groups can be finite and infinite. A simple example for the former category is the following. Let $S = \{ s \}$. Due to definition it must be $s^2 = e$. So $W$ is isomorph to $\zz_2$ and finite. An example for an infinite Coxeter group can be obtained from $S = \{s,t\}$ with $s^2=t^2=e$ and $(st)^\infty = e$ (so we have no relation between $s$ and $t$). Obviously the element $st$ has infinite order forcing $W$ to be infinite. But there are also infinite Coxeter groups without an $\infty$-relation between two generators. An example for this is $W$ obtained from $S=\{s_1,s_2,s_3\}$ with $s_1^2=s_2^2=s_3^2=(s_1 s_2)^3=(s_2 s_3)^3=(s_3 s_1)^3=e$. But how can it be seen that this $W$ is infinite?

It can be easily answered in general, if a Coxeter group is finite or infinite. The first step is to fallback to irreducible Coxeter groups. A Coxeter group (and the Coxeter system) is called \defword{irreducible}, if the corrosponding Coxeter graph is connected. The following lemma shows, why this simplification is valid:

\begin{lemm}
	Let $(W,S)$ be a reducible Coxeter system. This means that there is a partition of $S$ into $I, J$ with $(s_i s_j)$
\end{lemm}