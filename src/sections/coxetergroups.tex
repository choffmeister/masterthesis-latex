\section{Coxeter groups}

\subsection{Introduction to Coxeter groups}

A Coxeter group, named after Harold Scott MacDonald Coxeter, is an abstract group generated by involutions with specific relations between these generators. A simple class of a Coxeter groups are the symmetry groups of regular polyhedras in the Euclidean space. The symmetry group of the square for example can be generated by two reflections $s,t$, whose stabilized hyperplanes enclose an angle of $\pi / 4$. In this case the map $st$ is a rotation in the plane by $\pi / 2$. So we have $s^2 = t^2 = (st)^4 = \id$. In fact this reflection group is determined up to isomorphy by $s,t$ and these three relations \cite[Theorem 1.9]{humphreys:coxeter}. Furthermore it turns out, that the finite reflection groups in the Euclidean space are precisely the finite Coxeter groups \cite[Theorem 6.4]{humphreys:coxeter}.

In this chapter we will compile some basic facts on Coxeter groups. First of all the definition:

\begin{defi}[Coxeter system]
	\label{coxeter-system}
	Let $S = \{ s_1, \ldots, s_n \}$ be a finite set of symbols and
	$$R = \{ m_{ij} \in \nn \cup \infty : 1 \leq i,j \leq n \}$$
	a set numbers (or $\infty$) with $m_{ii} = 1$ and $m_{ij} = m_{ji}$. Then the free represented group
	$$W = \langle S \ | \ (s_i s_j)^{m_{ij}} \rangle$$
	is called a \glos{Coxeter group} and $(W,S)$ the corrosponding \glos{Coxeter system}.
\end{defi}

let $(W,S)$ be a Coxeter system and $w \in W$ an arbitrary element. We call a product $s_{i_1} \cdots s_{i_n} = w$ of generators $s_{i_1} \ldots s_{i_n} \in S$ an \glos{expression} of $w$. The present relations between the generators of a Coxeter group allow us to rewrite expressions. Hence an element $w \in W$ can have more than one expression. Obviously any element $w \in W$ has infinitly many expressions, since any expression $s_{i_1} \cdots s_{i_n} = w$ can be extended by applying $s_1^2 = 1$ from the right. But there must be a smallest number of generators needed to receive $w$. For example the neutral element $e$ can be expressed by the empty expression. Or each generator $s_i \in S$ can be expressed by itself, but any shorter expression (i.e. the empty expression) is unequal to $s_i$.

\begin{defi}[Length function]
	\label{length-function}
	Let $(W,S)$ be a Coxeter system and $w \in W$ an element. Then there are some (not neseccarily distince) generators $s_i \in S$ with $s_1 \cdots s_n = w$. We call $n$ the length of such an expression. The smallest number $n \in \nn_0$ for that $w$ has an expression of length $n$ is called the \glos{length} of $w$. The map
	$$ l : W \to \nn_0 $$
	that maps each element in $W$ to its length is calls $\glos{length function}$.
\end{defi}