\section{Coxeter groups}

\subsection{Introduction to Coxeter groups}

A Coxeter group, named after Harold Scott MacDonald Coxeter, is an abstract group generated by involutions with specific relations between these generators. A simple class of a Coxeter groups are the symmetry groups of regular polyhedras in the Euclidean space.

The symmetry group of the square for example can be generated by two reflections $s,t$, whose stabilized hyperplanes enclose an angle of $\pi / 4$. In this case the map $st$ is a rotation in the plane by $\pi / 2$. So we have $s^2 = t^2 = (st)^4 = \id$. In fact this reflection group is determined up to isomorphy by $s,t$ and these three relations \cite[Theorem 1.9]{humphreys:coxeter}. Furthermore it turns out, that the finite reflection groups in the Euclidean space are precisely the finite Coxeter groups \cite[Theorem 6.4]{humphreys:coxeter}.

In this chapter we will compile some basic facts on Coxeter groups. First of all the definition:

\begin{defi}[Coxeter system]
	\label{coxeter-system}
	Let $S = \{ s_1, \ldots, s_n \}$ be a finite set of symbols and
	$$R = \{ m_{ij} \in \nn \cup \infty : 1 \leq i,j \leq n \}$$
	a set numbers (or $\infty$) with $m_{ii} = 1$ and $m_{ij} = m_{ji}$. Then the free represented group
	$$W = \langle S \ | \ (s_i s_j)^{m_{ij}} \rangle$$
	is called a \defword{Coxeter group} and $(W,S)$ the corrosponding \defword{Coxeter system}.
\end{defi}

For a arbitrary element $w \in W$, $(W,S)$ a Coxeter system, we call a product $s_{i_1} \cdots s_{i_n} = w$ of generators $s_{i_1} \ldots s_{i_n} \in S$ an \defword{expression} of $w$. The present relations between the generators of a Coxeter group allow us to rewrite expressions. Hence an element $w \in W$ can have more than one expression. Obviously any element $w \in W$ has infinitly many expressions, since any expression $s_{i_1} \cdots s_{i_n} = w$ can be extended by applying $s_1^2 = 1$ from the right. But there must be a smallest number of generators needed to receive $w$. For example the neutral element $e$ can be expressed by the empty expression. Or each generator $s_i \in S$ can be expressed by itself, but any expression with less factors (i.e. the empty expression) is unequal to $s_i$.

\begin{defi}[Length function]
	\label{length-function}
	Let $(W,S)$ be a Coxeter system and $w \in W$ an element. Then there are some (not neseccarily distince) generators $s_i \in S$ with $s_1 \cdots s_r = w$. We call $r$ the \defword{expression length}. The smallest number $r \in \nn_0$ for that $w$ has an expression of length $r$ is called the \defword{length} of $w$ and each expression of $w$, that is ob minimal length, is called \defword{reduced expression}. The map
	$$ l : W \to \nn_0 $$
	that maps each element in $W$ to its length is calls \defword{length function}.
\end{defi}

The next theorem yields a way to obtain a reduced expression of an arbitrary element $s_1 \cdots s_r = w \in W$. But first we define what a reflection is. Any element $w \in W$ that is conjugated to an generator $s \in S$ is called \defword{reflection}. Hence the set of all reflections in $W$ is
$$ T = \bigcup_{w \in W} wSw^{-1}. $$

\begin{theo}[Strong Exchange Condition]
	Let $(W,S)$ be a Coxeter system, $w \in W$ an arbitrary element and ${s_1 \cdots s_r = w}$ with $s_i \in S$ a not neseccarily reduced expression for $w$. For each reflection $t \in T$ with $l(wt) < l(w)$ there exists an index $i$ for which $wt = s_1 \cdots \hat s_i \cdots s_r$, where $\hat s_i$ means omission. In case we started from an reduced expression, then $i$ is unique.

	\begin{proof}
		sad
	\end{proof}
\end{theo}