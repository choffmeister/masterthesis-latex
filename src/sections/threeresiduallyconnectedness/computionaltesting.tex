%!TEX root = ../../../_main.tex
\section{Computional testing for 3-residually connectedness}
\label{sec:3rc-compution-testing}

In section~\ref{sec:twisted-involutions-algorithms} we introduced an effective algorithm to calculate the poset graph for an arbitrary $Wk(\theta)$ until a given maximal twisted length. So the idea is obvious to use these data and simply test for every combination of $K,S_1,S_2,S_3$, if they yield a counterexample to \axiomref{3RC}. If no counterexample can be found we have proved that $Wk(\theta)$ is 3-residually connected. If $(W,S)$ is itself finite, then there is not problem with this approach. For infinite Coxeter systems we cannot calculate the whole poset graph. But there are some infinte ones, that can also be addressed in this way.

\begin{prop}
	Suppose the parabolic subgroup $W_{S'}$ to be finite for any $S' \subsetneqq S$. Define
	$$ E := \{ w \in \ti{\theta} : K,S_1,S_2,S_3 \subseteq S, w \in w_K C_{S_{12}} \cap w_K C_{S_{23}} \cap w_K C_{S_{31}}, w \notin w_K C_T \} $$
	and assume $E \neq \emptyset$, i.e. $Wk(\theta)$ is not 3-residually connected. Then there is a $\rho_E \in \nn_0$ with $\max_{w \in E} \rho(w) \leq \rho_E$.

	\begin{proof}
		For any counterexample the sets $S_1,S_2,S_3$ cannot equal $S$ by \ref{prop:3rc-special-3}, hence by assumption the sets $S_1,S_2,S_3$ are spherical. But then $S_{12},S_{23},S_{31}$ as well as $\theta(S_{12}),\theta(S_{23}),\theta(S_{31})$ must be spherical, too. Let $w \in E$, then $w \in W_{\theta(S_{12})} w_K W_{S_{12}}$, hence $l(w) \leq l(w_{\theta(S_{12})}) + l(w_K) + l(w_{S_{12}})$. Since there is only a finite count of proper subsets of $S$, we can choose
		$$ l_E = \max_{K,S' \subsetneqq S} l(w_{\theta(S')}) + l(w_K) + l(w_{S'}) \leq \max_{S' \subsetneqq S} 3 \cdot l(w_{S'}) = 3 \cdot \max_{s \in S} l(w_{S \setminus \{s\}}) < \infty $$
		as upper bound for $l(w), w \in E$. For all $w' \in \ti{\theta}$ we have $\rho(w') \leq l(w')$ and so $\rho_E = l_E$ is an upper bound for $\rho(w), w \in E$, too.
	\end{proof}
\end{prop}

\begin{rema}
	Note, that this proposition also gives a manual, how to calculate the upper bound, since the length of the longest element in a finite Coxeter group can be easily calculate. There are 4 families of finite Coxeter groups and 6 finite Coxeter groups of exceptional type (cf. \ref{theo:irreducible-finit-coxeter-systems}). The length of the longest element in each of them can be seen in Table~\ref{tab:length-w0}. For more details on how to calculate those values (resp. formulas) see \cite[Section 1.2]{franzsen:automorphisms} and \cite[Section 2.11]{humphreys:coxeter}.
\end{rema}

\begin{prop}
	Let $(W,S) = (W_1 \times W_2, S_1 \cup S_2)$ be a reducible Coxeter group. Then the length of the longest element in $W$ is the sum of the lengths of the longest elements in $W_1$ and $W_2$.

	\begin{proof}
		This is immediate, since $\ord(st) = 2$ for $s \in S_1$, $t \in S_2$.
	\end{proof}
\end{prop}

\begin{table}[ht]
	\centering
	\begin{tabular}{|c|cccccccccc|}
		\hline
		$W$ & $A_n$ & $B_n$ & $D_n$ & $E_6$ & $E_7$ & $E_8$ & $F_4$ & $H_3$ & $H_4$ & $I_2(m)$ \\
		\hline
		$l(w_0)$ & $n(n+1)/2$ & $n^2$ & $n(n-1)$ & $36$ & $63$ & $120$ & $24$ & $15$ & $60$ & $m$ \\
		\hline
	\end{tabular}
	\caption{Length of longest element in finite Coxeter groups}
	\label{tab:length-w0}	
\end{table}

\begin{exam}
	Let $W = \tilde A_2$ with $S = \{s_1, s_2, s_3\}$. Because of symmetry we can calculate $l_E$ with the set $S' = \{s_1,s_2\}$. Then $\langle S' \rangle \cong A_2$, hence the length of the longest word in $\langle S' \rangle$ is $3$. Therefore $l_E = 9$ and $\rho_E = 5$. This means, to validate if $\tilde A_2$ is 3-residually connected, we only need to calculate the poset graph of $Wk(\theta)$ until we have all twisted involutions of twisted length 5.
\end{exam}

Now we actually want to calculate the maximum element length over all proper parabolic subgroups for some Coxeter groups, where this value is finite. Table~\ref{tab:parabolic-max-length} shows the results for some interesting one, e.g. the affine Coxeter groups. In order to simplify notation in the table we sometimes does not properly distinguish between certain cases, i.e. we use types like $B_{n-3}$ without insisting on $n \geq 5$. For $n=3$ we would have $B_0$. In this case, we will consider this type as invalid possibility. For $n=4$ we would have $B_1$, which is not defined either. But in this case we will threat this type as $A_1$, which seems suitable.

\begin{table}[ht]
	\centering
	\begin{tabular}{l|p{7cm}|l}
		$W$ & All possible types for $W_{S \setminus \{s\}}$ & $\max_{s \in S} l(w_{S \setminus \{s\}})$ \\
		\hline
		$\tilde A_1$ & $A_1$ & 1 \\
		$\tilde A_n (n \geq 2)$ & $A_n$ & $n(n+1)/2$ \\
		$\tilde B_2 = \tilde C_2$ & $A_1 \times A_1$, $I_2(4)$ & $4$ \\
		$\tilde B_n (n \geq 3)$ & $B_n$, $D_n$, $D_{n-1} \times A_1$, $D_{n-m} \times B_m$, $A_3 \times B_{n-3}$, $B_{n-2} \times A_1 \times A_1$ & $n^2$ \\
		$\tilde C_n (n \geq 3)$ & $B_n$, $B_{n-1} \times A_1$, $B_{n-m} \times B_m$ & $n^2$ \\
		$\tilde D_n (n \geq 4)$ & $D_n$, $D_{n-2} \times A_1 \times A_1$, $D_{n-m} \times D_m$ & $n(n-1)$ \\
		$\tilde E_6$ & $E_6$, $A_5 \times A_1$, $A_2 \times A_2 \times A_2$ & $36$ \\
		$\tilde E_7$ & $E_7$, $D_6 \times A_1$, $A_5 \times A_2$, $A_7$, $A_3 \times A_3 \times A_1$ & $63$ \\
		$\tilde E_8$ & $E_8$, $E_7 \times A_1$, $E_6 \times A_2$, $D_5 \times A_3$, $A_4 \times A_4$, $A_5 \times A_2 \times A_1$, $A_8$, $A_7 \times A_1$, $D_8$ & $120$ \\
		$\tilde F_4$ & $F_4$, $B_3 \times A_1$, $A_2 \times A_2$, $A_3 \times A_1$, $B_4$ & $24$ \\
		$\tilde G_2$ & $I_2(6)$, $A_1 \times A_1$, $A_2$ & $6$ \\
	\end{tabular}
	\caption{Maximum element lengths in proper parabolic subgroups}
	\label{tab:parabolic-max-length}	
\end{table}