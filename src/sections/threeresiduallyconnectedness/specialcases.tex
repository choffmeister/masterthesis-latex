%!TEX root = ../../../_main.tex
\section{Special cases}
\label{sec:3rc-special-cases}

%\begin{exam}
%	Let $W = A_5$, $\theta = \id$ and $w = \ul s_1 \ul s_5 \ul s_3 \ul s_4 \ul s_2 \ul s_3$ as in Figure~\ref{fig:main-thesis-weakend-hypothesis-counterexample}. Denote the maximal element by $w_0$. Let $S_{12} = \{s_1,s_2\}$, $S_{23} = \{s_5,s_6\}$ and $S_{31} = \{s_1,s_5,s_6\}$. Then $w_0 \in wC_{S_i}$ for $i=1,2,3$ by $w_0 = w \ul s_2 \ul s_1 \ul s_2 = w \ul s_6 \ul s_5 \ul s_6 = w \ul s_6 \ul s_1 \ul s_5$, but $w_0 \notin wC_{S_{12} \cap S_{23} \cap S_{31}} = wC_{\emptyset} = \{w\}$.
%
%	\begin{figure}[ht]
%		\centering
%		%!TEX root = ../../_main.tex
\begin{tikzpicture}[scale=1,bend angle=10]
\newcommand{\xspace}{1}
\newcommand{\yspace}{1}
\tikzstyle{vertex}=[draw,thick,circle,minimum size=2mm,inner sep=0pt]
\tikzstyle{edge}=[->]
\tikzstyle{onesided}=[edge,dashed]
\tikzstyle{bothsided}=[edge]
\tikzstyle{unhighlighted}=[]
\tikzstyle{highlighted}=[ultra thick]
\definecolor{s1color}{RGB}{130,76,253}
\definecolor{s2color}{RGB}{76,253,78}
\definecolor{s3color}{RGB}{253,76,124}
\definecolor{s4color}{RGB}{76,176,253}
\definecolor{s5color}{RGB}{228,253,76}
\tikzstyle{s1}=[s1color]
\tikzstyle{s2}=[s2color]
\tikzstyle{s3}=[s3color]
\tikzstyle{s4}=[s4color]
\tikzstyle{s5}=[s5color]
\node[vertex,unhighlighted] (56) at (\xspace*-3.375,\yspace*9) {};
\node[vertex,unhighlighted] (57) at (\xspace*-2.625,\yspace*9) {};
\node[vertex,unhighlighted] (58) at (\xspace*-1.875,\yspace*9) {};
\node[vertex,unhighlighted] (59) at (\xspace*-1.125,\yspace*9) {};
\node[vertex,unhighlighted] (60) at (\xspace*-0.375,\yspace*9) {};
\node[vertex,unhighlighted] (61) at (\xspace*0.375,\yspace*9) {};
\node[vertex,unhighlighted] (62) at (\xspace*1.125,\yspace*9) {};
\node[vertex,unhighlighted] (63) at (\xspace*1.875,\yspace*9) {};
\node[vertex,highlighted] (64) at (\xspace*2.625,\yspace*9) {};
\node[vertex,unhighlighted] (65) at (\xspace*3.375,\yspace*9) {};
\node[vertex,unhighlighted] (66) at (\xspace*-1.875,\yspace*10.5) {};
\node[vertex,unhighlighted] (67) at (\xspace*-1.125,\yspace*10.5) {};
\node[vertex,highlighted] (68) at (\xspace*-0.375,\yspace*10.5) {};
\node[vertex,unhighlighted] (69) at (\xspace*0.375,\yspace*10.5) {};
\node[vertex,unhighlighted] (70) at (\xspace*1.125,\yspace*10.5) {};
\node[vertex,highlighted] (71) at (\xspace*1.875,\yspace*10.5) {};
\node[vertex,unhighlighted] (72) at (\xspace*-0.75,\yspace*12) {};
\node[vertex,highlighted] (73) at (\xspace*0,\yspace*12) {};
\node[vertex,highlighted] (74) at (\xspace*0.75,\yspace*12) {};
\node[vertex,highlighted] (75) at (\xspace*0,\yspace*13.5) {};
\draw[s5,bothsided,unhighlighted] (56) edge (66);
\draw[s3,bothsided,unhighlighted] (57) edge (66);
\draw[s4,onesided,unhighlighted] (57) edge (67);
\draw[s5,bothsided,unhighlighted] (58) edge (67);
\draw[s3,bothsided,unhighlighted] (58) edge (68);
\draw[s2,bothsided,unhighlighted] (59) edge (66);
\draw[s4,bothsided,unhighlighted] (59) edge (69);
\draw[s2,onesided,unhighlighted] (60) edge (67);
\draw[s3,bothsided,unhighlighted] (60) edge (69);
\draw[s5,bothsided,unhighlighted] (61) edge (69);
\draw[s2,bothsided,unhighlighted] (61) edge (70);
\draw[s1,bothsided,unhighlighted] (62) edge (66);
\draw[s4,bothsided,unhighlighted] (62) edge (70);
\draw[s1,bothsided,unhighlighted] (63) edge (67);
\draw[s3,bothsided,unhighlighted] (63) edge (71);
\draw[s1,bothsided,highlighted,bend right] (64) edge (68);
\draw[s4,bothsided,highlighted,bend left] (64) edge (68);
\draw[s2,bothsided,highlighted,bend right] (64) edge (71);
\draw[s5,bothsided,highlighted,bend left] (64) edge (71);
\draw[s1,bothsided,unhighlighted] (65) edge (69);
\draw[s4,bothsided,unhighlighted] (66) edge (72);
\draw[s3,bothsided,unhighlighted] (67) edge (73);
\draw[s5,bothsided,highlighted] (68) edge (73);
\draw[s2,bothsided,highlighted] (68) edge (74);
\draw[s2,bothsided,unhighlighted] (69) edge (72);
\draw[s1,bothsided,unhighlighted,bend right] (70) edge (72);
\draw[s5,bothsided,unhighlighted,bend left] (70) edge (72);
\draw[s3,onesided,unhighlighted] (70) edge (74);
\draw[s1,bothsided,highlighted] (71) edge (73);
\draw[s4,bothsided,highlighted] (71) edge (74);
\draw[s3,onesided,unhighlighted] (72) edge (75);
\draw[s2,bothsided,highlighted,bend right] (73) edge (75);
\draw[s4,bothsided,highlighted,bend left] (73) edge (75);
\draw[s1,bothsided,highlighted,bend right] (74) edge (75);
\draw[s5,bothsided,highlighted,bend left] (74) edge (75);
\end{tikzpicture}
%		\caption{Upper end of Hasse diagram of $Wk(A_5,\id)$}
%		\label{fig:main-thesis-weakend-hypothesis-counterexample}
%	\end{figure}
%\end{exam}

In this section we investigate some results and examples, in special situations. We fix some notation, namely let $K \subseteq S$ be fixed by $\theta$ and spherical, $S_1,S_2,S_3 \subseteq S$ have a pairwise non-empty intersection, $S_{ij} = S_i \cap S_j$, $T = S_1 \cap S_2 \cap S_3$ and $w_K$ denote the longest element in $\langle K \rangle$.

\begin{exam}
	\typedlabel{exam:trivial-counterexample-for-arbitary-sets-of-generators}
	Let $W = A_3$ and $\theta$ be the Coxeter system autmorphism swapping $s_1$ and $s_3$ and let $w = s_1s_3 = s_3s_1$. We have $e \ul s_1 = s_3 s_1 = w = s_1 s_3 = e \ul s_3$. Hence $w \in eC_{\{s_1\}}$ and $w \in eC_{\{s_3\}}$ but $w \notin eC_{\{s_1\} \cap \{s_1\} \cap \{s_3\}} = eC_\emptyset = \{e\}$.
\end{exam}

Such a trivial counterexample like in \ref{exam:trivial-counterexample-for-arbitary-sets-of-generators} can not occur in the situation from \ref{defi:3rc}.

\begin{prop}
	Let $w,v \in \ti{\theta}$ with $\rho(v) - \rho(w) = 1$ and let $v \in w C_{S_{ij}}$ for $1 \leq i < j \leq 3$. Then we have $v \in wC_T$.

	\begin{proof}
		By \ref{prop:no-triple-edges} there are at most two (not necessarily distinct) $s,t \in S$ with $w \ul s = v$ and $w \ul t = v$. Each set $S_{12},S_{23},S_{31}$ must at least contain $s$ or $t$, hence $s$ or $t$ is at least in two sets, say $s \in S_{12},S_{23}$. Hence $s \in S_1,S_2,S_3$ and therefore $v \in wC_T$.
	\end{proof}
\end{prop}

A property, that is much stronger than 3-residually connectedness, reads $wC_I \cap wC_J = wC_{I \cap J}$. If $Wk(\theta)$ satisfies this, then its 3-residually connectedness could be concluded immediately. Unfortunately it proves to be false in general. Again, double-edges yield a simple counterexample.

\begin{exam}
	\typedlabel{exam:cap-of-residues}
	Let $w \in \ti{\theta}$ and $s,t$ two distinct generators with $w \ul s = w \ul t = v$. Then $wC_{\{s\}} \cap wC_{\{t\}} = \{w,v\} \neq \{w\} = wC_{\emptyset} = wC_{\{s\} \cap \{t\}}$.
\end{exam}

\begin{prop}
	\typedlabel{prop:3rc-special}
	Suppose one of the following cases is current for some distinct $i,j \in \{1,2,3\}$:
	\begin{enumerate}
		\item \label{prop:3rc-special-1} $S_i = \emptyset$,
		\item \label{prop:3rc-special-2} $S_i \subseteq S_j$ or
		\item \label{prop:3rc-special-3} $S_i = S$.
	\end{enumerate}
	Then \axiomref{3RC} holds.

	\begin{proof}
		\begin{enumerate}
			\item We have $\bigcap_{1 \leq m < n \leq 3} w_K C_{S_{mn}} \subseteq w_K C_{S_{ij}} \subseteq w_K C_{S_{i}} = w_K C_\emptyset = w_K C_T$.
			\item We have $S_{ij} = S_i$, hence $T = S_i \cap S_j \cap S_k = S_{ij} \cap S_k = S_i \cap S_k = S_{ik}$. Therefore $\bigcap_{1 \leq m < n \leq 3} w_K C_{S_{mn}} \subseteq w_K C_{S_{ik}} = w_K C_T$.
			\item We have $S_j \subseteq S = S_i$ and so with \ref{prop:3rc-special-2} we are done. \qedhere
		\end{enumerate}
	\end{proof}
\end{prop}

\begin{coro}
	\typedlabel{prop:rank-leq-2-main-special-case}
	Suppose $|S| \leq 2$. Then $Wk(\theta)$ is 3-residually connected.

	\begin{proof}
		If one set of $S_1,S_2,S_3$ is empty or equal to $S$, then we are done by \ref{prop:3rc-special-1} and \ref{prop:3rc-special-3}. Else at least two sets of $S_1,S_2,S_3$ must be equal. In this case we are done by \ref{prop:3rc-special-2}.
	\end{proof}
\end{coro}

\begin{coro}
	\typedlabel{prop:rank-leq-3-main-special-case}
	Suppose $|S| \leq 3$. Then $Wk(\theta)$ is 3-residually connected.

	\begin{proof}
		The pairwise intersections of $S_1,S_2,S_3$ must be non-empty. By \ref{prop:3rc-special-2} every configuration, where on $S_i$ contains one other $S_j$, cannot yield a counterexample to \axiomref{3RC}. It remains essentially a single configuration to consider: $S_{12} = \{s_1\}, S_{23} = \{s_2\}, S_{31} = \{s_3\}$. But this cannot yield a counterexample either, since by \ref{prop:no-triple-edges} we have $w C_{\{s_1\}} \cap w C_{\{s_2\}} \cap w C_{\{s_3\}} = \{ w \} = w C_\emptyset = w C_T$.
	\end{proof}
\end{coro}

\begin{rema}
	\typedlabel{rema:rank-leq-4-main-special-case}
	It might be possible to handle the case $|S| = 4$ in a similar way. Having \ref{prop:3rc-special} in mind there are essentially three configurations left for $S_{12},S_{23},S_{31}$ to consider. The first configuration is $S_{12} = \{s_1\}$, $S_{23} = \{s_2\}$, $S_{31} = \{s_3\}$, which is analogue to \ref{prop:rank-leq-3-main-special-case} and so cannot yield a counter example to \axiomref{3RC}. So there are two configurations left to investigate:
	\begin{enumerate}
		\item $S_{12} = \{s_1\}$, $S_{23} = \{s_2\}$, $S_{31} = \{s_3,s_4\}$.
		\item $S_{12} = \{s_1,s_4\}$, $S_{23} = \{s_2,s_4\}$, $S_{31} = \{s_3,s_4\}$.
	\end{enumerate}
\end{rema}

\begin{prop}
	\typedlabel{prop:intersections-in-k-and-t}
	Let $\theta = \id$. Then the poset $Wk(\id)$ is 3-residually connected if and only if \axiomref{3RC} holds for all pairs $K,S_1,S_2,S_3$ that satisfy $S_{ij} \setminus T \subseteq K$ for $1 \leq i < j \leq 3$.

	\begin{proof}
		If $Wk(\id)$ is 3-residually connected, then \axiomref{3RC} holds for all pairs. In return suppose $Wk(\id)$ not to be 3-residually connected, i.e. there is a $w \in \ti{\id}$ with $w = s_1 \ldots s_n \in w_K C_{S_{12}} \cap C_{S_{23}} \cap C_{S_{31}}$ and $w \notin w_K C_T$. Since $w \in w_K C_{S_{12}}$ and \ref{lemm:w-reduced-expression-letters-independent} every reduced expression for $w$ (and $s_1 \ldots s_n$ in particular) uses only generators from $K \cup S_{12}$. This is true with the same argument for $K \cup S_{23}$ and $K \cup S_{31}$. Hence
		\begin{align*}
			s_i	& \in (K \cup S_{12}) \cap (K \cup S_{23}) \cap (K \cup S_{31}) = K \cup (S_{12} \cap S_{23} \cap S_{31}) = K \cup T
		\end{align*}
		for $1 \leq i \leq n$. So our counterexample to \axiomref{3RC} is also a counterexample if we replace $S_{ij}$ by $S'_{ij} := S_{ij} \cap (K \cup T)$ for $1 \leq i < j \leq 3$. But then
		\begin{align*}
			S'_{ij} \setminus T & = [S_{ij} \cap (K \cup T)] \setminus T \subseteq [K \cup T] \setminus T \subseteq K. \qedhere
		\end{align*}
	\end{proof}
\end{prop}

\begin{coro}
	\typedlabel{prop:theta-id-k-leq-2-main-special-case}
	Let $\theta = \id$ and $|K| \leq 2$. Then \axiomref{3RC} holds.

	\begin{proof}
		Assume that \axiomref{3RC} does not hold. Then by \ref{prop:intersections-in-k-and-t} we can assume that for our counterexample $S_{ij} \setminus T \subseteq K$ holds. Since $|K| \leq 2$ at least two of the sets $S_{ij}$ have to be equal. But then \ref{prop:3rc-special} says that \axiomref{3RC} holds, which contradicts our assumption.
	\end{proof}
\end{coro}