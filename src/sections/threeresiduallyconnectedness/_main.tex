%!TEX root = ../../../_main.tex
\chapter{Twisted weak ordering 3-residually connectedness}
\label{sec:main-thesis}

\begin{defi}
	\typedlabel{defi:3rc}
	Let $(W,S)$ be a Coxeter system and $\theta : W \to W$ an automorphism of $W$ with $\theta^2 = \id$ and $\theta(S) = S$. We call $Wk(\theta)$ \defword{3-residually connected}, if the following holds: For every possible spherical $K \subseteq S$ (i.e. $\langle K \rangle \leq W$ is finite) and every $S_1,S_2,S_3 \subseteq S$ with pairwise non-empty intersection the statement
	\begin{axioms}
		\axiomitem{3RC} $w_K C_{S_{12}} \cap w_K C_{S_{23}} \cap w_K C_{S_{31}} \subseteq w_K C_T$
	\end{axioms}
	holds, whereas $w_K$ denotes the longest element in $\langle K \rangle$, $S_{ij} = S_i \cap S_j$ and $T = S_1 \cap S_2 \cap S_3$. By construction $w_K C_T \subseteq w_K C_{S_{12}} \cap w_K C_{S_{23}} \cap w_K C_{S_{31}}$ always holds. So \axiomref{3RC} is equivalent to
	\begin{axioms}
		\axiomitem{3RCa} $w_K C_{S_{12}} \cap w_K C_{S_{23}} \cap w_K C_{S_{31}} = w_K C_T$.
	\end{axioms}
\end{defi}

For arbitrary sets $S_{12},S_{23},S_{31}$ that do not come from pairwise intersections it is easy to find pairs of Coxeter systems and Coxeter system automorphism that do not satisfy the (modified) 3-residually connectedness, as seen in \ref{exam:trivial-counterexample-for-arbitary-sets-of-generators}. The following proposition shows us, what distinguishes our special configuration of sets of generators from the arbitrary configuration.

\begin{prop}
	\typedlabel{prop:three-sets-pairwise-intersection}
	Let $M$ be a set and $S_{12},S_{23},S_{31} \subseteq M$ three subsets. Then there are three sets $S_1,S_2,S_3 \subseteq M$ with $S_{ij} = S_i \cap S_j$ iff no element $x \in M$ is precisely in two of the sets $S_{ij}$.

	\begin{proof}
		Let $S_{12},S_{23},S_{31}$ be the pairwise intersection of three sets $S_1,S_2,S_3$. If an element $x \in M$ is in none or in one of the sets $S_i$, then it is in none of the sets $S_{ij}$. If it is in two of the sets $S_i$, say $x \in S_1,S_2$, then $x \in S_{12}$, but $x$ is not in one of the other two $S_{ij}$. If $x$ is in all three $S_i$, then it is in all three $S_{ij}$, too. Hence there is no $x \in M$, that is in precisely two of the sets $S_{ij}$. Conversely, suppose $S_{12},S_{23},S_{31}$ to be arbitrary with the constraint, that there is no element $x \in M$ in precisely two of them. Then we can construct three sets $S_1,S_2,S_3$, whose pairwise intersections coincides with the sets $S_{ij}$ by $x \in S_i \wedge x \in S_j$ iff $x \in S_{ij}$. With this construction and the previous considerations, it is clear that these $S_i$ have the $S_{ij}$ as pairwise intersection. Note that this construction is not unique in general, since when there is a $x \in M$, that is in none of the sets $S_{ij}$, then we could add it to $S_1$, $S_2$ or $S_3$ or just omit it without changing there pairwise intersection.
	\end{proof}
\end{prop}

%!TEX root = ../../../_main.tex
\section{Special cases}
\label{sec:3rc-special-cases}

%\begin{exam}
%	Let $W = A_5$, $\theta = \id$ and $w = \ul s_1 \ul s_5 \ul s_3 \ul s_4 \ul s_2 \ul s_3$ as in Figure~\ref{fig:main-thesis-weakend-hypothesis-counterexample}. Denote the maximal element by $w_0$. Let $S_{12} = \{s_1,s_2\}$, $S_{23} = \{s_5,s_6\}$ and $S_{31} = \{s_1,s_5,s_6\}$. Then $w_0 \in wC_{S_i}$ for $i=1,2,3$ by $w_0 = w \ul s_2 \ul s_1 \ul s_2 = w \ul s_6 \ul s_5 \ul s_6 = w \ul s_6 \ul s_1 \ul s_5$, but $w_0 \notin wC_{S_{12} \cap S_{23} \cap S_{31}} = wC_{\emptyset} = \{w\}$.
%
%	\begin{figure}[ht]
%		\centering
%		%!TEX root = ../../_main.tex
\begin{tikzpicture}[scale=1,bend angle=10]
\newcommand{\xspace}{1}
\newcommand{\yspace}{1}
\tikzstyle{vertex}=[draw,thick,circle,minimum size=2mm,inner sep=0pt]
\tikzstyle{edge}=[->]
\tikzstyle{onesided}=[edge,dashed]
\tikzstyle{bothsided}=[edge]
\tikzstyle{unhighlighted}=[]
\tikzstyle{highlighted}=[ultra thick]
\definecolor{s1color}{RGB}{130,76,253}
\definecolor{s2color}{RGB}{76,253,78}
\definecolor{s3color}{RGB}{253,76,124}
\definecolor{s4color}{RGB}{76,176,253}
\definecolor{s5color}{RGB}{228,253,76}
\tikzstyle{s1}=[s1color]
\tikzstyle{s2}=[s2color]
\tikzstyle{s3}=[s3color]
\tikzstyle{s4}=[s4color]
\tikzstyle{s5}=[s5color]
\node[vertex,unhighlighted] (56) at (\xspace*-3.375,\yspace*9) {};
\node[vertex,unhighlighted] (57) at (\xspace*-2.625,\yspace*9) {};
\node[vertex,unhighlighted] (58) at (\xspace*-1.875,\yspace*9) {};
\node[vertex,unhighlighted] (59) at (\xspace*-1.125,\yspace*9) {};
\node[vertex,unhighlighted] (60) at (\xspace*-0.375,\yspace*9) {};
\node[vertex,unhighlighted] (61) at (\xspace*0.375,\yspace*9) {};
\node[vertex,unhighlighted] (62) at (\xspace*1.125,\yspace*9) {};
\node[vertex,unhighlighted] (63) at (\xspace*1.875,\yspace*9) {};
\node[vertex,highlighted] (64) at (\xspace*2.625,\yspace*9) {};
\node[vertex,unhighlighted] (65) at (\xspace*3.375,\yspace*9) {};
\node[vertex,unhighlighted] (66) at (\xspace*-1.875,\yspace*10.5) {};
\node[vertex,unhighlighted] (67) at (\xspace*-1.125,\yspace*10.5) {};
\node[vertex,highlighted] (68) at (\xspace*-0.375,\yspace*10.5) {};
\node[vertex,unhighlighted] (69) at (\xspace*0.375,\yspace*10.5) {};
\node[vertex,unhighlighted] (70) at (\xspace*1.125,\yspace*10.5) {};
\node[vertex,highlighted] (71) at (\xspace*1.875,\yspace*10.5) {};
\node[vertex,unhighlighted] (72) at (\xspace*-0.75,\yspace*12) {};
\node[vertex,highlighted] (73) at (\xspace*0,\yspace*12) {};
\node[vertex,highlighted] (74) at (\xspace*0.75,\yspace*12) {};
\node[vertex,highlighted] (75) at (\xspace*0,\yspace*13.5) {};
\draw[s5,bothsided,unhighlighted] (56) edge (66);
\draw[s3,bothsided,unhighlighted] (57) edge (66);
\draw[s4,onesided,unhighlighted] (57) edge (67);
\draw[s5,bothsided,unhighlighted] (58) edge (67);
\draw[s3,bothsided,unhighlighted] (58) edge (68);
\draw[s2,bothsided,unhighlighted] (59) edge (66);
\draw[s4,bothsided,unhighlighted] (59) edge (69);
\draw[s2,onesided,unhighlighted] (60) edge (67);
\draw[s3,bothsided,unhighlighted] (60) edge (69);
\draw[s5,bothsided,unhighlighted] (61) edge (69);
\draw[s2,bothsided,unhighlighted] (61) edge (70);
\draw[s1,bothsided,unhighlighted] (62) edge (66);
\draw[s4,bothsided,unhighlighted] (62) edge (70);
\draw[s1,bothsided,unhighlighted] (63) edge (67);
\draw[s3,bothsided,unhighlighted] (63) edge (71);
\draw[s1,bothsided,highlighted,bend right] (64) edge (68);
\draw[s4,bothsided,highlighted,bend left] (64) edge (68);
\draw[s2,bothsided,highlighted,bend right] (64) edge (71);
\draw[s5,bothsided,highlighted,bend left] (64) edge (71);
\draw[s1,bothsided,unhighlighted] (65) edge (69);
\draw[s4,bothsided,unhighlighted] (66) edge (72);
\draw[s3,bothsided,unhighlighted] (67) edge (73);
\draw[s5,bothsided,highlighted] (68) edge (73);
\draw[s2,bothsided,highlighted] (68) edge (74);
\draw[s2,bothsided,unhighlighted] (69) edge (72);
\draw[s1,bothsided,unhighlighted,bend right] (70) edge (72);
\draw[s5,bothsided,unhighlighted,bend left] (70) edge (72);
\draw[s3,onesided,unhighlighted] (70) edge (74);
\draw[s1,bothsided,highlighted] (71) edge (73);
\draw[s4,bothsided,highlighted] (71) edge (74);
\draw[s3,onesided,unhighlighted] (72) edge (75);
\draw[s2,bothsided,highlighted,bend right] (73) edge (75);
\draw[s4,bothsided,highlighted,bend left] (73) edge (75);
\draw[s1,bothsided,highlighted,bend right] (74) edge (75);
\draw[s5,bothsided,highlighted,bend left] (74) edge (75);
\end{tikzpicture}
%		\caption{Upper end of Hasse diagram of $Wk(A_5,\id)$}
%		\label{fig:main-thesis-weakend-hypothesis-counterexample}
%	\end{figure}
%\end{exam}

In this section we investigate some results and examples, in special situations. We fix some notation, namely let $K \subseteq S$ be spherical, $S_1,S_2,S_3 \subseteq S$ have a pairwise non-empty intersection, $S_{ij} = S_i \cap S_j$, $T = S_1 \cap S_2 \cap S_3$ and $w_K$ denote the longest element in $\langle K \rangle$.

\begin{exam}
	\typedlabel{exam:trivial-counterexample-for-arbitary-sets-of-generators}
	Let $W = A_3$ and $\theta$ be the Coxeter system autmorphism swapping $s_1$ and $s_3$ and let $w = s_1s_3 = s_3s_1$. We have $e \ul s_1 = s_3 s_1 = w = s_1 s_3 = e \ul s_3$. Hence $w \in eC_{\{s_1\}}$ and $w \in eC_{\{s_3\}}$ but $w \notin eC_{\{s_1\} \cap \{s_1\} \cap \{s_3\}} = eC_\emptyset = \{e\}$.
\end{exam}

Such a trivial counterexample like in \ref{exam:trivial-counterexample-for-arbitary-sets-of-generators} can not occur in the situation from \ref{defi:3rc}.

\begin{prop}
	Let $w,v \in \ti{\theta}$ with $\rho(v) - \rho(w) = 1$ and let $v \in w C_{S_{ij}}$ for $1 \leq i < j \leq 3$. Then we have $v \in wC_T$.

	\begin{proof}
		By \ref{prop:no-triple-edges} there are at most two (not necessarily distinct) $s,t \in S$ with $w \ul s = v$ and $w \ul t = v$. Each set $S_{12},S_{23},S_{31}$ must at least contain $s$ or $t$, hence $s$ or $t$ is at least in two sets, say $s \in S_{12},S_{23}$. Hence $s \in S_1,S_2,S_3$ and therefore $v \in wC_T$.
	\end{proof}
\end{prop}

A property, that is much stronger than 3-residually connectedness, reads $wC_I \cap wC_J = wC_{I \cap J}$. If $Wk(\theta)$ satisfies this, then its 3-residually connectedness could be concluded immediately. Unfortunately it proves to be false in general. Again, double-edges yield a simple counterexample.

\begin{exam}
	\typedlabel{exam:cap-of-residues}
	Let $w \in \ti{\theta}$ and $s,t$ two distinct generators with $w \ul s = w \ul t = v$. Then $wC_{\{s\}} \cap wC_{\{t\}} = \{w,v\} \neq \{w\} = wC_{\emptyset} = wC_{\{s\} \cap \{t\}}$.
\end{exam}

\begin{prop}
	\typedlabel{prop:3rc-special}
	Suppose one of the following cases is current for some pairwise distinct $i,j,k \in \{1,2,3\}$:
	\begin{enumerate}
		\item \label{prop:3rc-special-1} $S_i = \emptyset$,
		\item \label{prop:3rc-special-2} $S_i \subseteq S_j$ or
		\item \label{prop:3rc-special-3} $S_i = S$.
	\end{enumerate}
	Then \axiomref{3RC} holds.

	\begin{proof}
		\begin{enumerate}
			\item We have $\bigcap_{1 \leq m < n \leq 3} w_K C_{S_{mn}} \subseteq w_K C_{S_{ij}} \subseteq w_K C_{S_{i}} = w_K C_\emptyset = w_K C_T$.
			\item We have $S_{ij} = S_i$, hence $T = S_i \cap S_j \cap S_k = S_{ij} \cap S_k = S_i \cap S_k = S_{ik}$. Therefore $\bigcap_{1 \leq m < n \leq 3} w_K C_{S_{mn}} \subseteq w_K C_{S_{ik}} = w_K C_T$.
			\item We have $S_j \subseteq S = S_i$ and so with \ref{prop:3rc-special-2} we are done. \qedhere
		\end{enumerate}
	\end{proof}
\end{prop}

\begin{coro}
	\typedlabel{prop:rank-leq-2-main-special-case}
	Suppose $|S| \leq 2$. Then $Wk(\theta)$ is 3-residually connected.

	\begin{proof}
		If one set of $S_1,S_2,S_3$ is empty or equal to $S$, then we are done by \ref{prop:3rc-special-1} and \ref{prop:3rc-special-3}. Else at least two sets of $S_1,S_2,S_3$ must be equal. In this case we are done by \ref{prop:3rc-special-2}.
	\end{proof}
\end{coro}
%!TEX root = ../../../_main.tex
\section{Reducible case}
\label{sec:3rc-reducible-case}

\begin{lemm}
	\typedlabel{lemm:properties-of-wk-in-reducible-coxeter-groups}
	Let $(W, S_1 \dotcup S_2)$ be a reducible Coxeter system with $\ord(st) = 2$ for $s \in S_1, t \in S_2$. Let $\theta = \id$, $s_1,\ldots,s_m,s \in S_1$ and $t_1,\ldots,t_n,t \in S_2$. Then

	\begin{enumerate}
		\item $\ul s$ acts by twisted conjugation on $\ul s_1 \ldots \ul s_m \ul t_1 \ldots \ul t_n$ if and only if it acts by twisted conjugation on $\ul s_1 \ldots \ul s_m$,
		\item $\ul t$ acts by twisted conjugation on $\ul s_1 \ldots \ul s_m \ul t_1 \ldots \ul t_n$ if and only if it acts by twisted conjugation on $\ul t_1 \ldots \ul t_m$, and
		\item $\ul s_1 \ldots \ul s_m \ul t_1 \ldots \ul t_n \ul s = \ul s_1 \ldots \ul s_m \ul s \ul t_1 \ldots \ul t_n$.
	\end{enumerate}

	\begin{proof}
		We have $\ul s_1 \ldots \ul s_m \ul t_1 \ldots \ul t_n = t_{i_q} \cdots t_{i_1} s_{j_r} \cdots s_{j_1} s_1 \cdots s_m t_1 \cdots t_n$ for some well chosen indices $1 \leq i_1 < \ldots < i_q \leq m$ and $1 \leq j_1 < \ldots < j_r \leq n$.

		\begin{enumerate}
			\item We prove this by a straight forward chain of equivalences.
			\begin{align*}
						& s (\ul s_1 \ldots \ul s_m \ul t_1 \ldots \ul t_n) s  =  \ul s_1 \ldots \ul s_m \ul t_1 \ldots \ul t_n \\
				\iff	& s (t_{i_q} \cdots t_{i_1} s_{j_r} \cdots s_{j_1} s_1 \cdots s_m t_1 \cdots t_n) s  =  t_{i_q} \cdots t_{i_1} s_{j_r} \cdots s_{j_1} s_1 \cdots s_m t_1 \cdots t_n \\
				\iff	& (t_{i_q} \cdots t_{i_1} t_1 \cdots t_n) s s_{j_r} \cdots s_{j_1} s_1 \cdots s_m s  =  (t_{i_q} \cdots t_{i_1} t_1 \cdots t_n) s_{j_r} \cdots s_{j_1} s_1 \cdots s_m \\
				\iff	& s s_{j_r} \cdots s_{j_1} s_1 \cdots s_m s  =  s_{j_r} \cdots s_{j_1} s_1 \cdots s_m \\
				\iff	& s (\ul s_1 \ldots \ul s_m) s = \ul s_1 \ldots \ul s_m
			\end{align*}
			\item This part is almost the same as before.
			\begin{align*}
						& t (\ul s_1 \ldots \ul s_m \ul t_1 \ldots \ul t_n) t  =  \ul s_1 \ldots \ul s_m \ul t_1 \ldots \ul t_n \\
				\iff	& t (t_{i_q} \cdots t_{i_1} s_{j_r} \cdots s_{j_1} s_1 \cdots s_m t_1 \cdots t_n) t  =  t_{i_q} \cdots t_{i_1} s_{j_r} \cdots s_{j_1} s_1 \cdots s_m t_1 \cdots t_n \\
				\iff	& t t_{i_q} \cdots t_{i_1} t_1 \cdots t_n t (s_{j_r} \cdots s_{j_1} s_1 \cdots s_m)  =  t_{i_q} \cdots t_{i_1} t_1 \cdots t_n (s_{j_r} \cdots s_{j_1} s_1 \cdots s_m) \\
				\iff	& t t_{i_q} \cdots t_{i_1} t_1 \cdots t_n t  =  t_{i_q} \cdots t_{i_1} t_1 \cdots t_n \\
				\iff	& t (\ul t_1 \ldots \ul t_n) t = \ul t_1 \ldots \ul t_n
			\end{align*}
			Note that the last equivalence is not true in general. Suppose $v \in \ti{\theta}$ to be an arbitrary twisted expression. In general we cannot deduce the action of $\ul s$ on a subexpression of $v$ from the action of $\ul s$ on $v$ itself. But with the first part of this lemma we can first conclude, that $\ul t_1$ acts by twisted conjugation on $e$ if and only if it acts by twisted conjugation on $\ul s_1 \ldots \ul s_m$. Again with the same argument $\ul t_2$ acts by twisted conjugation on $\ul t_1$ iff it acts by twisted conjugation on $\ul s_1 \ldots \ul s_m \ul t_1$ and so forth.
			\item To avoid having to repeat the proof for twisted conjugative and multiplicative action of $\ul s$ we set $s' = s$ if $\ul s$ acts by twisted conjugation and else $s' = e$.
			\begin{align*}
					& \ul s_1 \ldots \ul s_m \ul t_1 \ldots \ul t_n \ul s \\
				= \	& s' (t_{i_q} \cdots t_{i_1} s_{j_r} \cdots s_{j_1} s_1 \cdots s_m t_1 \cdots t_n) s \\
				= \	& t_{i_q} \cdots t_{i_1} (s' s_{j_r} \cdots s_{j_1} s_1 \cdots s_m s) t_1 \cdots t_n \\
				= \	& t_{i_q} \cdots t_{i_1} (s_1 \ldots \ul s_m \ul s) t_1 \cdots t_n \\
				= \	& s_1 \ldots \ul s_m \ul s \ul t_1 \ldots \ul t_n
			\end{align*}
			Again note that the last to equalities need the two previous parts of this lemma. \qedhere
		\end{enumerate}
	\end{proof}
\end{lemm}

\begin{coro}
	\typedlabel{coro:wk-of-reducible-coxeter-groups}
	Let $(W,S_1 \dotcup S_2)$ be Coxeter system with $\ord(st) = 2$ whenever $s \in S_1, t \in S_2$. In particular $W$ is reducible. Let $W := W_{S_1}$ and $W_2 := W_{S_2}$ be the parabolic subgroups of $W$ corrosponding to $S_1$ and $S_2$. Then we have $Wk(W,\id) \cong Wk(W_1,\id) \times Wk(W_2,\id)$.

	\begin{proof}
		We denote the relation in $W$ (resp. in $W_1$, $W_2$) by $\preceq_W$ (resp. by $\preceq_{W_1}$, $\preceq_{W_2}$). By \ref{lemm:properties-of-wk-in-reducible-coxeter-groups} for every element $w \in \ti{\id}(W)$ we can find a twisted expression like $w = \ul s_1 \ldots \ul s_m \ul t_1 \ldots \ul t_n$ with $s \in S_1, t \in S_2$. Hence the map
		$$ \varphi : \ti{\id}(W_1) \times \ti{\id}(W_2) \to \ti{\id}(W) : (\ul s_1 \ldots \ul s_m, \ul t_1 \ldots \ul t_n) \mapsto \ul s_1 \ldots \ul s_m \ul t_1 \ldots \ul t_n $$
		is surjective. The injectivity is due to \ref{prop:reducible-coxeter-systems-isomorph-to-parabolic-subgroups}. It remains to show that $\preceq_W$ satisfies \ref{defi:direct-product-of-posets}. Let $v_1,w_1 \in \ti{\id}(W_1)$, $v_2,w_2 \in \ti{\id}(W_2)$ and $v = v_1v_2 = \varphi(v_1,v_2), w = w_1w_2 = \varphi(w_1,w_2) \in \ti{\id}(W)$. Suppose $v_i \preceq_{W_i} w_i$ for $i=1,2$. Then we have
		\begin{align*}
			v_1	& = \ul s_1 \ldots \ul s_m, &
			w_1	& = \ul s_1 \ldots \ul s_m \ldots \ul s_{m'} = v_1 \ul s_{m+1} \ldots \ul s_{m'}, \\
			v_2	& = \ul t_1 \ldots \ul t_n \textrm{ and} &
			w_2	& = \ul t_1 \ldots \ul t_n \ldots \ul t_{n'} = v_2 \ul t_{n+1} \ldots \ul t_{n'}
		\end{align*}
		for some well chosen generators $s_i \in S_1, t_i \in S_2$ and $0 \leq m \leq m', 0 \leq n \leq n'$. Hence
		\begin{align*}
			v	& = v_1 v_2 = \ul s_1 \ldots \ul s_m \ul t_1 \ldots \ul t_n \preceq_W \ul s_1 \ldots \ul s_m \ul t_1 \ldots \ul t_n \ul s_{m+1} \ldots \ul s_{m'} \ul t_{n+1} \ldots \ul t_{n'} \\
				& = \ul s_1 \ldots \ul s_m \ul s_{m+1} \ldots \ul s_{m'} \ul t_1 \ldots \ul t_n  \ul t_{n+1} \ldots \ul t_{n'} = w_1 w_2 = w.
		\end{align*}
		In return suppose $v \preceq_W w$. Then we have
		\begin{align*}
			v	& = \ul s_1 \ldots \ul s_m \ul t_1 \ldots \ul t_n \textrm{ and} \\
			w	& = \ul s_1 \ldots \ul s_m \ul t_1 \ldots \ul t_n \ul s_{m+1} \ldots \ul s_{m'} \ul t_{n+1} \ldots \ul t_{n'}
		\end{align*}
		for some well chosen generators $s_i \in S_1, t_i \in S_2$ and $0 \leq m \leq m', 0 \leq n \leq n'$. Again with similar arguments we have 
		\begin{align*}
			v_1	& = \ul s_1 \ldots \ul s_m \preceq_{W_1} \ul s_1 \ldots \ul s_m \ul s_{m+1} \ldots \ul s_{m'} = w_1 \textrm{ and} \\
			w_1	& = \ul t_1 \ldots \ul t_n \preceq_{W_2} \ul t_1 \ldots \ul t_n \ul t_{n+1} \ldots \ul t_{n'} = w_2. \qedhere
		\end{align*}
	\end{proof}
\end{coro}

\begin{rema}
	Note that \ref{lemm:properties-of-wk-in-reducible-coxeter-groups} and \ref{coro:wk-of-reducible-coxeter-groups} still hold, if we drop the premise $\theta = \id$ and instead insist on $\theta(S_i) = S_i$ for $i=1,2$. They also remain true, if we have a partition of the generator set in more than two subsets. Hence for $(W,S_1 \dotcup \ldots \dotcup S_n)$ with $\ord(st) = 2$ whenever $s \in S_i, t \in S_j, i \neq j$ we have
	$$ Wk(W,\id) = Wk(W_{S_1},\id) \times \ldots \times Wk(W_{S_n},\id). $$
\end{rema}

\begin{theo}
	Let $(W,S)$ be a reducible Coxeter system with $S = S' \cup S''$ and $\ord(st) = 2$ whenever $s \in S', t \in S''$ and let $\theta = \id$. Then $Wk(W,\id)$ is 3-residually connected if and only if $Wk(W_{S'},\id)$ and $Wk(W_{S''},\id)$ are 3-residually connected.

	\begin{proof}
		If $Wk(W,\id)$ is 3-residually connected, then $Wk(W_{S'},\id)$ and $Wk(W_{S''},\id)$ are so in particular. In return suppose $Wk(W_{S'},\id)$ and $Wk(W_{S''},\id)$ to be 3-residually connected. For a set $M \subseteq S$ we define $M' := M \cap S'$ and $M'' := M \cap S''$, hence $M = M' \dotcup M''$. This is compatible with our definition of $S_{ij}$ and $T$:
		\begin{align*}
			S_{ij}	& = S_i \cap S_j = (S'_i \dotcup S''_i) \cap (S'_j \dotcup S''_j) = (S'_i \cap S'_j) \dotcup (S''_i \cap S''_j) = S'_{ij} \dotcup S''_{ij} \\
			T		& = S_1 \cap S_2 \cap S_3 = (S'_{12} \dotcup S''_{12}) \cap (S'_3 \dotcup S''_3) = (S'_{12} \cap S'_3) \dotcup (S''_{12} \cap S''_3) = T' \dotcup T''
		\end{align*}
		Let $w_K = \ul s'_1 \ldots \ul s'_{m'} \ul s''_1 \ldots \ul s''_{m''}$ with $s'_i \in K',s''_i \in K''$. Then
		$w_{K'} = \ul s'_1 \ldots \ul s'_{m'}$ (resp. $w_{K''} = \ul s''_1 \ldots \ul s''_{m''}$) is the corrosponding longest elements in $\langle K' \rangle \leq W_{S'} \leq W$ (resp. $\langle K'' \rangle \leq W_{S''} \leq W$).
		We have three twisted expressions
		\begin{align*}
			w	& = w_K \ul a'_1 \ldots \ul a'_{n'} \ul a''_1 \ldots \ul a''_{n''} \\
				& = w_K \ul b'_1 \ldots \ul b'_{n'} \ul b''_1 \ldots \ul b''_{n''} \\
				& = w_K \ul c'_1 \ldots \ul c'_{n'} \ul c''_1 \ldots \ul c''_{n''}
		\end{align*}
		with $a'_i,a''_i \in S_1$, $b'_i,b''_i \in S_2$ and $c'_i,c''_i \in S_3$. Thanks to \ref{lemm:properties-of-wk-in-reducible-coxeter-groups} we can assume without loss of generality that $a',b',c' \in S'$ and $a'',b'',c'' \in S''$. Hence we have also
		\begin{align*}
			w'	& = w_{K'} \ul a'_1 \ldots \ul a'_{n'} = s'_1 \ldots \ul s'_m \ul a'_1 \ldots \ul a'_{n'} \\
				& = w_{K'} \ul b'_1 \ldots \ul b'_{n'} = s'_1 \ldots \ul s'_m \ul b'_1 \ldots \ul b'_{n'} \\
				& = w_{K'} \ul c'_1 \ldots \ul c'_{n'} = s'_1 \ldots \ul s'_m \ul c'_1 \ldots \ul c'_{n'}
		\end{align*}
		and so $w' \in w_{K'}C_{T'}$, since \axiomref{3RC} holds in $Wk(W_{S'},\id)$. Analogue we get $w'' \in w_{K''}C_{T''}$. Hence
		$$ w' = \ul s'_1 \ldots \ul s'_{m'} \ul d'_1 \ldots \ul d'_{l'} \textrm{ and } w'' = \ul s''_1 \ldots \ul s''_{m''} \ul d''_1 \ldots \ul d''_{l''} $$
		for $d'_i \in T'$ and $d''_i \in T''$. This yields a twisted expression
		\begin{align*}
			w	& = w' w'' = \ul s'_1 \ldots \ul s'_{m'} \ul d'_1 \ldots \ul d'_{l'} \ul s''_1 \ldots \ul s''_{m''} \ul d''_1 \ldots \ul d''_{l''} \\
				& = \ul s'_1 \ldots \ul s'_{m'} \ul s''_1 \ldots \ul s''_{m''} \ul d'_1 \ldots \ul d'_{l'} \ul d''_1 \ldots \ul d''_{l''} \\
				& = w_K \ul d'_1 \ldots \ul d'_{l'} \ul d''_1 \ldots \ul d''_{l''}
		\end{align*}
		with $d'_i,d''_i \in T' \dotcup T'' = T$. Thus $w \in w_KC_T$.
	\end{proof}
\end{theo}
%!TEX root = ../../../_main.tex
\section{Computional testing for 3-residually connectedness}
\label{sec:3rc-compution-testing}

In section~\ref{sec:twisted-involutions-algorithms} we introduced an effective algorithm to calculate the poset graph for an arbitrary $Wk(\theta)$ until a given maximal twisted length. So the idea is obvious to use these data and simply test for every combination of $K,S_1,S_2,S_3$, if they yield a counterexample to \axiomref{3RC}. If no counterexample can be found we have proved that $Wk(\theta)$ is 3-residually connected. If $(W,S)$ is itself finite, then there is not problem with this approach. For infinite Coxeter systems we cannot calculate the whole poset graph. But there are some infinte ones, that can also be addressed in this way.

\begin{prop}
	Suppose the parabolic subgroup $W_{S'}$ to be finite for any $S' \subsetneqq S$. Define
	$$ E := \{ w \in \ti{\theta} : K,S_1,S_2,S_3 \subseteq S, w \in w_K C_{S_{12}} \cap w_K C_{S_{23}} \cap w_K C_{S_{31}}, w \notin w_K C_T \} $$
	and assume $E \neq \emptyset$, i.e. $Wk(\theta)$ is not 3-residually connected. Then there is a $\rho_E \in \nn_0$ with $\max_{w \in E} \rho(w) \leq \rho_E$.

	\begin{proof}
		For any counterexample the sets $S_1,S_2,S_3$ cannot equal $S$ by \ref{prop:3rc-special-3}, hence by assumption the the sets $S_1,S_2,S_3$ are spherical. But then the sets $S_{12},S_{23},S_{31}$ as well as $\theta(S_{12}),\theta(S_{23}),\theta(S_{31})$ must be spherical, too. Let $w \in E$, then $w \in W_{\theta(S_{12})} w_K W_{S_{12}}$, hence $l(w) \leq l(w_{\theta(S_{12})}) + l(w_K) + l(w_{S_{12}})$. Since there is only a finite count of proper subsets of $S$, we can choose
		$$ l_E = \max_{K,S' \subsetneqq S} l(w_{\theta(S')}) + l(w_K) + l(w_{S'}) \leq \max_{S' \subsetneqq S} 3 \cdot l(w_{S'}) < \infty $$
		as upper bound for $l(w)$. For all $w' \in \ti{\theta}$ we have $\rho(w') \leq 2 \cdot l(w')$ and so $\rho_E = \lceil l_E/2 \rceil$ is an upper bound for $\rho(w)$.
	\end{proof}
\end{prop}

\begin{rema}
	Note, that this proposition also gives a manual, how to calculate the upper bound, since the length of the longest element in a finite Coxeter group can be easily calculate. There are 4 families of finite Coxeter groups and 6 finite Coxeter groups of exceptional type (cf. \ref{theo:irreducible-finit-coxeter-systems}). The length of the longest element in each of them can be seen in Table~\ref{tab:length-w0}. For more details on how to calculate those values (resp. formulas) see \cite[Section 1.2]{franzsen:automorphisms} and \cite[Section 2.11]{humphreys:coxeter}.
\end{rema}

\begin{table}
	\centering
	\begin{tabular}{|c|cccccccccc|}
		\hline
		$W$ & $A_n$ & $B_n$ & $D_n$ & $E_6$ & $E_7$ & $E_8$ & $F_4$ & $H_3$ & $H_4$ & $I_2(m)$ \\
		\hline
		$l(w_0)$ & $n(n+1)/2$ & $n^2$ & $n(n-1)$ & $36$ & $63$ & $120$ & $24$ & $15$ & $60$ & $m$ \\
		\hline
	\end{tabular}
	\caption{Length of longest element in finite Coxeter groups}
	\label{tab:length-w0}	
\end{table}

\begin{exam}
	Let $W = \tilde A_2$ with $S = \{s_1, s_2, s_3\}$. Because of symmetry we can calculate $l_E$ with the set $S' = \{s_1,s_2\}$. Then $\langle S' \rangle \cong A_2$, hence the length of the longest word in $\langle S' \rangle$ is $3$. Therefore $l_E = 9$ and $\rho_E = 5$. This means, to validate if $\tilde A_2$ is 3-residually connected, we only need to calculate the poset graph of $Wk(\theta)$ until we have all twisted involutions of twisted length 5.
\end{exam}