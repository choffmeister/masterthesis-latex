%!TEX root = ../../../_main.tex
\section{Coxeter groups}
\label{sec:coxeter-groups}

A Coxeter group, named after Harold Scott MacDonald Coxeter, is an abstract group generated by involutions with specific relations between these generators. A simple class of Coxeter groups are the symmetry groups of regular polyhedras in the Euclidean space.

The symmetry group of the square for example can be generated by two reflections $s,t$, whose stabilized hyperplanes enclose an angle of $\pi / 4$. In this case the map $st$ is a rotation in the plane by $\pi / 2$. So we have $s^2 = t^2 = (st)^4 = \id$. In fact, this reflection group is determined up to isomorphy by $s,t$ and these three relations \cite[Theorem 1.9]{humphreys:coxeter}. Furthermore it turns out, that the finite reflection groups in the Euclidean space are precisely the finite Coxeter groups \cite[Theorem 6.4]{humphreys:coxeter}.

In this chapter we compile some basic well-known facts on Coxeter groups, based on \cite{humphreys:coxeter}.

\begin{defi}
	\typedlabel{defi:coxeter-system}
	Let $S = \{ s_1, \ldots, s_n \}$ be a finite set of symbols and
	$$R = \{ m_{ij} \in \nn \cup \infty : 1 \leq i,j \leq n \}$$
	a set numbers (or $\infty$) with $m_{ii} = 1$, $m_{ij} > 1$ for $i \neq j$ and $m_{ij} = m_{ji}$. Then the free represented group
	$$W = \langle S \ | \ (s_i s_j)^{m_{ij}} \rangle$$
	is called a \defword{Coxeter group} and $(W,S)$ the corrosponding \defword{Coxeter system}. The cardinality of $S$ is called the \defword{rank} of the Coxeter system (and the Coxeter group).
\end{defi}

From the definiton we see, that Coxeter groups only depend on the cardinality of $S$ and the relations between the generators in $S$. A common way to visualize this information are Coxeter graphs.

\begin{defi}
	\typedlabel{defi:coxeter-graph}
	Let $(W,S)$ be a Coxeter system. Create a graph by adding a vertex for each generator in $S$. Let $(s_i s_j)^m = 1$. In case $m = 2$ the two corrosponding vertices have no connecting edge. In case $m = 3$ they are connected by an unlabed edge. For $m > 3$ they have an connecting edge with label $m$. We call this graph the \defword{Coxeter graph} of our Coxeter system $(W,S)$.
\end{defi}

\begin{defi}
	Let $(W,S)$ be a Coxeter system. For an arbitrary element $w \in W$ we call a product $s_{i_1} \cdots s_{i_n} = w$ of generators $s_{i_1} \ldots s_{i_n} \in S$ an \defword{expression} of $w$. Any expression that can be obtained from $s_{i_1} \cdots s_{i_n}$ by omitting some (or all) factors, is called a \defword{subexpression} of $w$.
\end{defi}

The present relations between the generators of a Coxeter group allow us to rewrite expressions. Hence an element $w \in W$ can have more than one expression. Obviously any element $w \in W$ has infinitly many expressions, since any expression $s_{i_1} \cdots s_{i_n} = w$ can be extended by applying $s_1^2 = 1$ from the right. But there must be a smallest number of generators needed to receive $w$. For example the neutral element $e$ can be expressed by the empty expression. Or each generator $s_i \in S$ can be expressed by itself, but any expression with less factors (i.e. the empty expression) is unequal to $s_i$.

\begin{defi}
	\typedlabel{defi:length-function}
	Let $(W,S)$ be a Coxeter system and $w \in W$ an element. Then there are some (not necessarily distinct) generators $s_i \in S$ with $s_1 \cdots s_r = w$. We call $r$ the \defword{expression length}. The smallest number $r \in \nn_0$ for that $w$ has an expression of length $r$ is called the \defword{length} of $w$ and each expression of $w$, that is of minimal length, is called \defword{reduced expression}. The map
	$$ l : W \to \nn_0 $$
	that maps each element in $W$ to its length is called \defword{length function}.
\end{defi}

\begin{defi}
	Let $(W,S)$ be a Coxeter system. We define
	$$ D_R(w) := \{ s \in S : l(ws) < l(w) \} $$
	as the \defword{right descending set} of $w$. The analogue left version
	$$ D_L(w) := \{ s \in S : l(sw) < l(w) \} $$
	is called \defword{left descending set} of $w$. Since the left descending set is not need in this paper, we will often call the right descending just \defword{descending set} of $w$.
\end{defi}

The next lemma yields some useful identities and relations for the length function.

\begin{lemm}
	\typedlabel{lemm:length-function-properties}
	\theocite{Section 5.2}{humphreys:coxeter}
	Let $(W,S)$ be a Coxeter system, $s \in S$, $u, w \in W$ and $l : W \to \nn$ the length function. Then
	\begin{enumerate}
		\item $l(w) = l(w^{-1})$,
		\item $l(w) = 0$ iff $w = e$,
		\item $l(w) = 1$ iff $w \in S$,
		\item $l(uw) \leq l(u) + l(w)$,
		\item $l(uw) \geq l(u) - l(w)$ and
		\item $l(ws) = l(w) \pm 1$.
	\end{enumerate}
\end{lemm}