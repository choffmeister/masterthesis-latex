%!TEX root = ../../../_main.tex
\subsubsection{Exchange and Deletion Condition}
\label{sec:coxeter-groups-exchange-deletion-condition}

We now obtain a way to get a reduced expression of an arbitrary element $s_1 \cdots s_r = w \in W$.

\begin{defi}
	\typedlabel{defi:reflection}
	Let $(W,S)$ be a Coxeter system. Any element $w \in W$ that is conjugated to an generator $s \in S$ is called \defword{reflection}. Hence the set of all reflections in $W$ is
	$$ T = \bigcup_{w \in W} wSw^{-1}. $$
\end{defi}

\begin{theo}[Strong Exchange Condition]
	\namedlabel{theo:strong-exchange-condition}
	\theocite{Theorem 5.8}{humphreys:coxeter}
	Let $(W,S)$ be a Coxeter system, $w \in W$ an arbitrary element and ${s_1 \cdots s_r = w}$ with $s_i \in S$ a not necessarily reduced expression for $w$. For each reflection $t \in T$ with $l(wt) < l(w)$ there exists an index $i$ for which $wt = s_1 \cdots \hat s_i \cdots s_r$, where $\hat s_i$ means omission. In case we start from a reduced expression, then $i$ is unique.
\end{theo}

The \ref{theo:strong-exchange-condition} can be weaken, when insisting on $t \in S$ to receive the following corollary.

\begin{coro}[Exchange Condition]
	\namedlabel{coro:exchange-condition}
	Let $(W,S)$ be a Coxeter system, $w \in W$ an arbitrary element and ${s_1 \cdots s_r = w}$ with $s_i \in S$ a not necessarily reduced expression for $w$. For each generator $s \in S$ with $l(ws) < l(w)$ there exists an index $i$ for which $ws = s_1 \cdots \hat s_i \cdots s_r$, where $\hat s_i$ means omission. In case we start from a reduced expression, then $i$ is unique.

	\begin{proof}
		Directly from \ref{theo:strong-exchange-condition}.
	\end{proof}
\end{coro}

\begin{rema}
	\typedlabel{rema:exchange-condition-left-sided}
	Note that both, \ref{theo:strong-exchange-condition} and \ref{coro:exchange-condition} have an analogues left-sided version
	$$ l(tw) < l(w) \Rightarrow tw = t s_1 \cdots s_k = s_1 \cdots \hat s_i \cdots s_k $$
	for all reflections $t \in T$, hence for all generators $s \in S$ in particular.
\end{rema}

\begin{coro}[Deletion Condition]
	\namedlabel{coro:deletion-condition}
	\theocite{Corollary 5.8}{humphreys:coxeter}
	Let $(W,S)$ be a Coxeter system, $w \in W$ and $w = s_1 \cdots s_r$ with $s_i \in S$ an unreduced expression of $w$. Then there exist two indices $i,j \in \{1,\cdots,r\}$ with $i < j$, such that $w = s_1 \cdots \hat s_i \cdots \hat s_j \cdots s_r$, where $\hat s_i$ and $\hat s_j$ mean omission.

	\begin{proof}
		Since the expression is unreduced there must be an index $j$ for that the twisted length shrinks. That means for $w' = s_1 \cdots s_{j-1}$ is $l(w' s_j) < l(w')$. Using the \ref{coro:exchange-condition} we get $w' s_j = s_1 \cdots \hat s_i \cdots s_{j-1}$ yielding $w = s_1 \cdots \hat s_i \cdots \hat s_j \cdots s_r$.
	\end{proof}
\end{coro}

This corollary is called \defword{Deletion Condition} and allows us to reduce expressions, i.e. to find a subexpression that is reduced. Due to the Deletion Condition any unreduced expression can be reduced by omitting an even number of generators (we just have to apply the Deletion Condition inductively).

The \ref{theo:strong-exchange-condition}, the \ref{coro:exchange-condition} and the \ref{coro:deletion-condition}, are some of the most powerful tools when investigating properties of Coxeter groups. We can use the second to prove a very handy property of Coxeter groups. The intersection of two parabolic subgroups is again a parabolic subgroup.

\begin{defi}
	\typedlabel{defi:parabolic-subgroup}
	Let $(W,S)$ be a Coxeter system. For a subset of generators $I \subset S$ we call the subgroup $W_I \leq W$, that is generated by the elements in $I$ with the corrosponding relations, a \defword{parabolic subgroup} of $W$.
\end{defi}

\begin{lemm}
	\typedlabel{lemm:w-reduced-expression-letters-independent}
	\theocite{Section 5.8}{humphreys:coxeter}
	Let $(W,S)$ be a Coxeter system and $w \in W$. Let $w = s_1 \cdots s_k$ any reduced expression for $w$. Then $\{s_1, \ldots, s_k\} \subset S$ is independent of the particular choosen reduced expression. It only depends on $w$ itself.
\end{lemm}

This means, that two reduced expressions for an element $w \in W$ use exactly the same generators. A related fact, is the following lemma.

\begin{lemm}
	\theocite{Section 5.8}{humphreys:coxeter}
	Let $(W,S)$ be a Coxeter system and $I,J \subset S$ two subsets of generators. Then ${W_I \cap W_J} = W_{I \cap J}$.
\end{lemm}