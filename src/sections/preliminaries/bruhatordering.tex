%!TEX root = ../../../_main.tex
\section{Bruhat ordering}
\label{sec:coxeter-groups-bruhat-ordering}

We now investigate ways to partially order the elements of a Coxeter group. Furthermore, this ordering should be compatible with the length function, i.e. for $w,v \in W$ we have $l(w) < l(v)$ whenever $w < v$.

\begin{defi}
	\typedlabel{defi:bruhat-ordering}
	Let $(W,S)$ be a Coxeter system and $T = \cup_{w \in W} wSw^{-1}$ the set of all reflections in $W$. We write $w' \to w$ if there is a $t \in T$ with $w't = w$ and $l(w') < l(w)$. If there is a sequence $w' = w_0 \to w_1 \to \ldots \to w_m = w$, we say $w' < w$. The resulting relation $w' \leq w$ is called \defword{Bruhat ordering}, denoted by $\Br(W)$.
\end{defi}

\begin{lemm}
	\typedlabel{lemm:bruhat-ordering-is-poset}
	\theocite{Section 5.9}{humphreys:coxeter}
	Let $(W,S)$ be a Coxeter system. Then $\Br(W)$ is a poset.

	\begin{proof}
		The Bruhat ordering is reflexive by definition. Since the elements in sequences $e \to w_1 \to w_2 \to \ldots$ are strictly ascending in length, it must be antisymmetric. By concatenation of sequences we get the transitivity.
	\end{proof}
\end{lemm}

We now show that $Br(W)$ is graded. By definition we already have $v < w$ iff $l(v) < l(w)$, but its not that obvious that two immediately adjacent elements differ in length by exactly 1. Beforehand let us just mention two other partial orderings that are graded by definition.

\begin{defi}
	Let $(W,S)$ be a Coxeter system. The ordering $\leq_R$ defined by $u \leq_R w$ iff $uv = w$ for some $u \in W$ with $l(u) + l(v) = l(w)$ is called the \defword{right weak ordering}. The left-sided version $u \leq_L w$ iff $vu = w$ is called the \defword{left weak ordering}.
\end{defi}

To ensure the Bruhat ordering is graded as well, we need another characterization of the Bruhat ordering in terms of subexpressions.

\begin{prop}
	\typedlabel{prop:u-leq-w-then-us-leq-w-or-us-leq-ws}
	\theocite{Proposition 5.9}{humphreys:coxeter}
	Let $(W,S)$ be a Coxeter system, $u,w \in W$ with $u \leq w$ and $s \in S$. Then $us \leq w$ or $us \leq ws$ or both.
\end{prop}

\begin{theo}[Subword property]
	\namedlabel{theo:bruhat-subexpression-characterization}
	\theocite{Theorem 5.10}{humphreys:coxeter}
	Let $(W,S)$ be a Coxeter system and $w \in W$ with a fixed, but arbitrary, reduced expression $w = s_1 \cdots s_r$, $s_i \in S$. Then $u \leq w$ (in the Bruhat ordering) iff $u$ can be obtained as a subexpression of this reduced expression.
\end{theo}

\begin{coro}
	\typedlabel{coro:bruhat-intervals-are-finite}
	Let $u,w \in W$. Then the interval $[u,w]$ in the Bruhat order $\Br(W)$ is finite.

	\begin{proof}
		We have $[u,w] \subseteq [e,w]$. All elements $v \in [e,w]$ can be obtained as subexpressions of one fixed reduced expression for $w$. Let $s_1 \ldots s_k = w$ be such a reduced expression. Then there are at most $2^k$ many subexpressions, hence $[u,w]$ is finite.
	\end{proof}
\end{coro}

This characterization of the Bruhat ordering is very handy. With it and the following short lemma we will be in the position to show that $\Br(W)$ is graded with rank function $l$.

\begin{lemm}
	\typedlabel{lemm:dont-know}
	\theocite{Lemma 5.11}{humphreys:coxeter}
	Let $(W,S)$ be a Coxeter system, $u,w \in W$ with $u < w$ and $l(w) = l(u) + 1$. In case there is a generator $s \in S$ with $u < us$ but $us \neq w$, then both $w < ws$ and $us < ws$.
\end{lemm}

\begin{prop}
	\typedlabel{prop:bruhat-intervals}
	\theocite{Proposition 5.11}{humphreys:coxeter}
	Let $(W,S)$ be a Coxeter system and $u < w$. Then there are elements $w_0,\ldots,w_m \in W$ such that $u = w_0 < w_1 < \ldots < w_m = w$ with $l(w_i) = l(w_{i-1}) + 1$ for $1 \leq i \leq m$.
\end{prop}

\begin{coro}
	\typedlabel{coro:bruhat-ordering-is-graded}
	Let $(W,S)$ be a Coxeter system and $\Br(W)$ the Bruhat ordering poset of $W$. Then $\Br(W)$ is graded with $l:W \to \nn$ as rank function.

	\begin{proof}
		Let $u,w \in W$ with $w$ covering $u$. Then \ref{prop:bruhat-intervals} says there is a sequence $u = w_0 < \ldots < w_m = w$ with $l(w_i) = l(w_{i-1}) + 1$ for $1 \leq i \leq m$. Since $w$ covers $u$ it must be $m = 1$ and so $u < w$ with $l(w) = l(u) + 1$.
	\end{proof}
\end{coro}

\begin{theo}[Lifting Property]
	\namedlabel{theo:lifting-property}
	\theocite{Theorem 1.1}{deodhar:bruhat-order}
	Let $(W,S)$ be a Coxeter system and $v,w \in W$ with $v \leq w$. Suppose $s \in S$ with $s \in D_R(w)$. Then
	\begin{enumerate}
		\item $vs \leq w$,
		\item $s \in D_R(v) \Rightarrow vs \leq ws$.
	\end{enumerate}
\end{theo}

\begin{rema}
	Note that the \ref{theo:lifting-property} has an analogue left-sided version: Let $(W,S)$ be a Coxeter system and $v,w \in W$ with $v \leq w$. Suppose $s \in S$ with $s \in D_L(w)$. Then
	\begin{enumerate}
		\item $sv \leq w$,
		\item $s \in D_L(v) \Rightarrow sv \leq sw$.
	\end{enumerate}
\end{rema}

The \ref{theo:lifting-property} seems quite innocent, but when trying to investigate facts around the Bruhat ordering it proves to be one of the key tools in many cases.

\begin{prop}
	\typedlabel{prop:bruhat-is-directed}
	\theocite{Proposition 7}{denton:coxeter}
	The poset $\Br(W)$ is directed.
\end{prop}

\begin{prop}
	\typedlabel{prop:bruhat-is-bounded}
	\theocite{Proposition 8}{denton:coxeter}
	\begin{enumerate}
		\item Let $W$ be finite, then there exists an unique element $w_0 \in W$ with $w \leq w_0$ for all $w \in W$.
		\item If $W$ contains an element $w$, with $D_R(w) = S$, then $W$ is finite and $w$ is the unique element $w_0$.
	\end{enumerate}
\end{prop}

\begin{coro}
	Let $(W,S)$ be a finite Coxeter system. Then $\Br(W)$ is graded, directed and bounded.

	\begin{proof}
		$\Br(W)$ is graded due to \ref{coro:bruhat-ordering-is-graded}, directed due to \ref{prop:bruhat-is-directed} and bounded due to \ref{prop:bruhat-is-bounded}.
	\end{proof}
\end{coro}

\begin{coro}
	Let $(W,S)$ be a Coxeter system and $w,v \in W$ with $w < v$. Then the interval $[w,v]$ is a finite, graded, directed and bounded poset.

	\begin{proof}
		The poset structure and the graduation transfers directly from $Br(W)$. By \ref{coro:bruhat-intervals-are-finite} intervals in $\Br(W)$ are finite. Since $v$ is the unique maximal element and $w$ the unique minimal element, it is bounded. By definition of intervals we have $u \leq v$ for every element $u \in [w,v]$, hence it is directed.
	\end{proof}
\end{coro}