%!TEX root = ../../../_main.tex
\section{Buildings}

\begin{defi}
	A \defword{building} of type $(W,S)$ is a pair $(\mathcal{C}, \delta)$ with a nonempty set $\mathcal{C}$ and a map $\delta : \mathcal{C} \times \mathcal{C} \to W$, called \defword{distance function}, so that for $x,y \in \mathcal C$ and $w = \delta(x,y)$ we have
	\begin{axioms}
		\axiomitem{Bu1} $w = e \iff x = y$;
		\axiomitem{Bu2} for $z \in \mathcal C$ with $\delta(y,z) = s \in S$ we have $\delta(x,z) \in \{w,ws\}$, and if in addition $l(ws) = l(w) + 1$ then we have $\delta(x,z) = ws$;
		\axiomitem{Bu3} for $s \in S$ there exists a $z \in \mathcal C$ with $\delta(y,z) = s$ and $\delta(x,z) = ws$.
	\end{axioms}
\end{defi}

For the rest of the subsection let $(\mathcal{C}, \delta)$ always be a building of type $(W,S)$.

\begin{defi}
	Then cardinality of $S$ is called the \defword{rank} of the building.
\end{defi}

\begin{defi}
	For each $s \in S$ we define $c,d \in C$ to be $s$-adjacent if and only iff $\delta(c,d) \in \{e,s\}$. Then $(\mathcal{C}, (\sim_s, s \in S))$ is called the \defword{associated chamber system} to $(\mathcal{C}, \delta)$.
\end{defi}

\begin{lemm}
	Then the associated chamber system is a chamber system.

	\begin{proof}
		Let $c,d \in \mathcal{C}$ and $s,t \in S$ with $c \sim_s d$ and $c \sim_t d$. If $c \neq d$, then $\delta(c,d) = s$ and $\delta(c,d) = t$, hence $s = t$.
	\end{proof}
\end{lemm}

\begin{defi}
	A \defword{gallery}, \defword{residue} or \defword{panel} in a building is a gallery, residue or panel in the associated chamber system.
\end{defi}

\begin{defi}
	We call the building $(\mathcal{C}, \delta)$ \defword{thick} (resp. \defword{thin}), if for every chamber $c \in \mathcal{C}$ and every $s \in S$ there are at least three (resp. exactly two) chambers $s$-adjacent to $c$.
\end{defi}

\begin{exam}
	\typedlabel{exam:thin-building}
	For a Coxeter system $(W,S)$ define a map
	$$ \delta_S : W \times W \to W : (x,y) \mapsto x^{-1}y. $$
	Then $\delta_S(x,y) = e \iff x = y$. Furthermore for $z \in W$ with $\delta_S(y,z) = s$, i.e. $z = ys$, we have $\delta_S(x,z) = x^{-1}z = x^{-1}ys = \delta(x,y)s$. For $s \in S$ and $x,y \in W$ choose $z = ys$. Then $\delta_S(y,z) = s$ and as before $\delta_S(x,z) = \delta_S(x,y)s$. Hence $(W,\delta_S)$ is a building of type $(W,S)$. More precisely, it is a thin building, since for every $s \in S$ and $x,y \in W$ we have $\delta_S(x,y) = x^{-1}y \in \{e,s\}$ if and only if $x = y$ or $y = xs$, hence there are excatly two chambers $s$-adjacent to $x$.
\end{exam}

This example for a thin building of type $(W,S)$ can be indeed called "the" thin building of type $(W,S)$ as the following lemma shows.

\begin{lemm}
	\theocite{Theorem 4.2.8}{buekenhout:diagram-geometry}
	Let $(\mathcal{C}, \delta)$ be a thin. Then it is isometric to the building $(W, \delta_S)$ (cf. \ref{exam:thin-building}).
\end{lemm}

\begin{defi}
	We call a subset $\Sigma \subseteq \mathcal{C}$ an \defword{apartment} if $(\Sigma, \delta|_\Sigma)$ is isometric to $(W,\delta_S)$ from \ref{exam:thin-building}, or equivalent if $(\Sigma, \delta|_\Sigma)$ is thin.
\end{defi}

\begin{theo}
	\theocite{Theorem 11.2.5}{buekenhout:diagram-geometry}
	For any two chambers $c,d \in \mathcal{C}$ there is an apartment $\Sigma$ with $c,d \in \Sigma$. In particular every building contains at least one apartment.

	\begin{proof}
		The proof for the first statement can be found in \cite[Theorem 11.2.5]{buekenhout:diagram-geometry}. The second is an immediate conclusion of the first, since because of $|S| \geq 1$ and \axiomref{Bu3} every building must at least contain two chambers. And so there is at least one pair of chambers, that has to be contained in an apartment by the first statement.
	\end{proof}
\end{theo}

So thin buildings are precisely those, that contain excatly one apparment, i.e. are apartments themself.

\begin{defi}
	The building $(\mathcal{C},\delta)$ is called \defword{spherical} if $W$ is finite. In this case $W$ has a longest element $w_0$ and two chambers $c,d$ are called \defword{opposite} if $\delta(c,d) = w_0$, denoted by $c \opp d$.
\end{defi}

\begin{defi}
	A set of chambers $M \subseteq \mathcal{C}$ is called \defword{connected} if any two chambers in $M$ can be joined by a gallery completeley contained in $M$. If in addition, every minimal gallery joining two chambers in $M$ is completeley contained in $M$, then $M$ is called \defword{convex}.
\end{defi}