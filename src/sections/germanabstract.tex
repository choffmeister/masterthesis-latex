%!TEX root = ../../_main.tex
\chapter*{Deutsche Zusammenfassung}

Regelmäßige Flächen und Körper sowie ihre Symmetriegruppen sind in der Mathematik seit jeher von großem Interesse. Mit der Einführung von Coxetergruppen durch Harold Scott MacDonald Coxeter im Jahr 1934 konnten diese Symmetriegruppen abstrakt gefasst werden und zugleich eine deutlich größere Klasse von Gruppen untersucht werden. Heutzutagen spielen Coxetergruppen in vielen Bereichen der Mathematik eine große Rolle.

Für ein sogenanntes Coxetersystem $(W,S)$, bestehend aus einer Menge von involutorischen Erzeugern $S$ und der von ihnen erzeugten Gruppe $W$, sei $\theta$ ein Automorphismus von $W$ der $S$ festhält und maximal Ordnung 2 hat. Dann heißt die Menge $\ti{\theta}$ der Elemente $w \in W$, die von $\theta$ auf ihr inverses abgebildet werden, die Menge der $\theta$-getwisteten Involutionen. Es existiert dann eine spezielle Abbildung $(w,s) \mapsto w \ul s$, welche die Eigenschaft hat, dass der Orbit des neutralen Elements von $W$ bzgl. dieser Abbildung gerade die Menge der $\theta$-getwisteten Involutionen ist und mit dessen Hilfe sich eine bestimmte Halbordnung $\preceq$ auf dieser Menge definieren lässt. Der Verband $(\ti{\theta},\preceq)$ heißt dann getwistete schwache Ordnung $Wk(W,\theta)$. Für ein Element $w \in \ti{\theta}$ und eine Teilmenge von Erzeugern $S' \subseteq S$ heißt die Menge aller getwisteten Involutionen, die von der Form $w \ul s_1 \ldots \ul s_n$ mit $s_1,\ldots,s_n \in S'$ sind, das $S'$-Residuum von $w$, geschrieben als $w C_{S'}$.

Im Rahmen dieser Arbeit heißt $Wk(W,\theta)$ 3-residuell zusammenhängend, falls folgendes gilt: Seien $K,S_1,S_2,S_3 \subseteq S$ Mengen von Erzeugern, wobei $K$ spährisch ist und von $\theta$ festgehalten wird und sich die $S_1,S_2,S_3$ paarweise nicht leer schneiden. Weiter sei $w_K$ das maximale Element im Residuum $w C_K$. Dann gilt
$$ w C_{S_1} \cap w C_{S_2} \cap w C_{S_3} \subseteq w C_{S_1 \cap S_2 \cap S_3}. $$
Die offene Fragestellung, um die es in dieser Arbeit gehen soll, ist, für welche Paare $((W,S),\theta)$ die getwistete schwache Ordnung 3-residuell zusammenhängend ist. Um dies zu überprüfen, wird in dieser Arbeit nach einer Einleitung in die Theorie zuerst ein effizienter Algorithmus entwickelt, um den Verband $Wk(W,\theta)$ berechnen zu können. Dann werden diese Ergebnisse benutzt, um mit einem weiteren Algorithmus nach Gegenbeispielen für den 3-residuellen Zusammenhang zu suchen. Dies ist für endliche Coxetersysteme in Gänze möglich. Für unendliche Coxetersystem jedoch, scheitert dieses Vorgehen im Allgemeinen. Es wird jedoch gezeigt, wie sich mit diesem Vorgehen zumindest einige unendliche Coxetersysteme behandlen lassen, nämlich die affinen und kompakten hyperbolischen. Zum Abschluss der Arbeit wird dann noch ein kurzer Exkurs in die Gebäudetheorie gemacht. Dabei wird gezeigt, dass für alle $Wk(W,\theta)$, für die der 3-residuelle Zusammenhang gezeigt werden konnte, auch gilt, dass für jedes sogenannte Zwillingsgebäude vom Typ $W$ das sogenannte Flipflop-System $\mathcal{C}^\theta$ residuell zusammenhängend ist.