\section{Misc}

\begin{ques}
\label{ques:main}
Let $(W,S)$ be a Coxeter system, $\theta : W \to W$ an automorphism of $W$ with
$\theta^2 = \id$ and $\theta(K) = K$, and $K \subset S$ a subset of $S$ generating
a finite subgroup of $W$. Futhermore let $T,S_1,S_2,S_3 \subset S$ be four pairwise
disjunct sets of generators. For which Coxeter group $W$ does the implication
\begin{equation}
\label{equ:main}
w \in w_K C_{T \cup S_i}, i=1,2,3 \Rightarrow w \in w_K C_T
\end{equation}
hold for any possible $K,\theta,T,S_1,S_2,S_3$ and $w$?
\end{ques}

\begin{prop}
\label{prop:counterexample-simplification}
Let $(W,S)$ be a Coxeter system and $K,T,S_1,S_2,S_3$ be like in \ref{ques:main}.
Suppose we have $w \in W$ and $a_1,\ldots,a_n \in T \cup S_1$, $b_1,\ldots,b_n \in T \cup S_2$, 
$c_1,\ldots,c_n \in T \cup S_3$ with
\begin{align*}
w & = w_K \ul{ a_1 \cdots a_n } \\
  & = w_K \ul{ b_1 \cdots b_n } \\
  & = w_K \ul{ c_1 \cdots c_n }
\end{align*}
and equation \ref{equ:main} does not hold for these three expressions, i.e. $w \notin w_K C_T$.
Then there exist $t_1,\ldots,t_m \in T$ and $a'_1,\ldots,a'_{n-m} \in T \cup S_1$,
$b'_1,\ldots,b'_{n-m} \in T \cup S_2$, $c'_1,\ldots,c'_{n-m} \in T \cup S_3$ such that
\begin{align*}
w \ul {t_1 \ldots t_m} & = w_K \ul{ a'_1 \cdots a'_{n-m} } \\
                       & = w_K \ul{ b'_1 \cdots b'_{n-m} } \\
                       & = w_K \ul{ c'_1 \cdots c'_{n-m} }
\end{align*}
with $a'_{n-m},b'_{n-m},c'_{n-m} \notin T$.

\begin{proof}
Suppose at least one element of $a_n,b_n,c_n$ to be in $T$, for example $a_n \in T$.
Then we can apply $\ul{a_n}$ to all three expressions. Since $\rho (w \ul{a_n}) < \rho(w)$
the exchange condition for $\ti{\theta}$ \cite[Proposition 3.10]{hultman:comb-twisted-invo}
yields
\begin{align*}
w \ul{a_n} & = w_K \ul{ a_1 \cdots a_n a_n } = w_K \ul{ a_1 \cdots a_{n-1} } \\
           & = w_K \ul{ b_1 \cdots b_n a_n } = w_K \ul{ b_1 \cdots \hat b_i \cdots b_n } \\
           & = w_K \ul{ c_1 \cdots c_n a_n } = w_K \ul{ c_1 \cdots \hat c_j \cdots c_n }
\end{align*}
where $\hat \cdot$ means omission. The omission cannot occur within $w_K$ since all three
expressions are still of same twisted length and in the first expression we can see, that
$w_K \leq w \ul {a_n}$ still holds. This step can be repeated until $w = w_K$ or
$a_n,b_n,c_n \notin T$.
\end{proof}
\end{prop}

\begin{lemm}
A counterexample to \ref{ques:main} can only be found, if there is an element $u \in w C_T$
and three distinct generators $s_1,s_2,s_3 \in D_r(u)$ such that $u \ul {s_i} \notin w C_T$
for $i=1,2,3$.

\begin{proof}
According to proposition \ref{prop:counterexample-simplification}.
\end{proof}
\end{lemm}