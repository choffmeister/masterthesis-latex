%TODO translate
\section{Zweizykel in der getwisteten schwachen Ordnung}

%TODO translate
\begin{defi}[Ein- und beidseitige Wirkung]
Seien $(W,S)$ ein Coxetersystem, $w \in W$ und $s \in S$. Falls $w \ul s =
\theta(s)ws$ ist, so sagen wir, dass $s$ beidseitig auf $w$ wirkt.
Andernfalls sagen wir $s$ wirkt einseitig auf $w$.
\end{defi}

%TODO translate
\begin{defi}[Ein- und beidseitige endende Gesamtwirkung]
Seien $(W,S)$ ein Coxetersystem, $w \in W$ und $s_1,\ldots,s_n \in S$. Falls $w
\ul {s_1 \cdots s_n} = \theta(s_n)(w \ul {s_1 \cdots s_{n-1}}) s_n$ ist, so
sagen wir, dass $s_1 \cdots s_n$ eine beidseitig endende Gesamtwirkung auf
$w$ hat. Andernfalls sagen wir $s_1 \cdots s_n$ hat eine einseitig endende
Gesamtwirkung auf $w$.
\end{defi}

%TODO translate
\begin{defi}
Seien $(W,S)$ ein Coxetersystem und $s,t \in S$ zwei
verschiedene Erzeuger. Wir definieren:
$$[st]^n :=
\begin{cases}
(st)^{\frac{n}{2}}, & n \textrm{ gerade} \\
(st)^{\frac{n-1}{2}}s, & n \textrm{ ungerade} 
\end{cases}$$
\end{defi}

%TODO translate
\begin{defi}[Zweizykel]
Seien $(W,S)$ ein Coxetersystem und $s,t \in S$ zwei verschiedene Erzeuger. Dann
nennen wir $wC_{\{s,t\}}$  den von $s$ und $t$ erzeugten Zweizykel bezüglich
$w$.
\end{defi}

%TODO translate
\begin{assu}
Seien $(W,S)$ ein Coxetersystem und $s,t \in S$ zwei verschiedene Erzeuger
von $W$. Dann gilt:
\begin{enumerate}
  \item \label{max-twocycle-length} Sei $m = \ord (st) < \infty$. Falls
  $w \ul{[st]^n} \neq w$ ist für alle $n \in \nn, n < 2m$, dann gilt $w \ul{(st)^{2m}} = w$.
  \item \label{twocycle-is-convex} In $wC_{\{s,t\}}$ existieren keine drei
  Elemente derselben getwisteten Länge.
  \item \label{onesided-operations-only-at-top-or-bottom-end-of-twocycle}
  Falls $s$ einseitig auf $w$ wirkt, dann gilt $w \ul{st} < w \ul{s}$ oder $w
  \ul{t} > w$.
  \item \label{twocycle-symmetry} Sei $w \ul{[st]^n} = w$ für ein $n \in
  \nn$. Dann ist $n$ gerade und es gilt eine der beiden folgenden Eigenschaften:
  	\begin{enumerate}
  	  \item Für jedes $m \in \nn$ hat das Element $[st]^m$ genau dann eine
  	  beidseitig endende Gesamt\-wirkung auf $w$, wenn $[st]^{n/2+m}$ eine
  	  beidseitig endende Gesamtwirkung auf $w$ hat.
  	  \item Für jedes $m \in \nn$ hat das Element $[st]^m$ genau dann eine
  	  beidseitig endende Gesamt\-wirkung auf $w$, wenn $[st]^{n-m+1}$ eine
  	  beidseitig endende Gesamtwirkung auf $w$ hat.
  	\end{enumerate}
\end{enumerate}
\end{assu}

%TODO translate
\begin{rema}
	\itemref{twocycle-is-convex} bedeutet, dass Zweizykel in einem
	gewissen Sinne konkav sind.
	\itemref{onesided-operations-only-at-top-or-bottom-end-of-twocycle} bedeutet,
	dass innerhalb eines Zweizykels einseitige Wirkungen ausschließlich am bzgl.
	der ge\-twis\-te\-ten Länge oberen oder unteren Ende auftreten können.
	\itemref{twocycle-symmetry} bedeutet, dass in einem Zweizykel die
	ein- und beidseitigen Wirkungen achsen- oder punktsymmetrisch verteilt sind.
\end{rema}

\begin{lemm}[\itemref{twocycle-is-convex}]
\begin{proof}
Let $(W,S)$ be a Coxeter system, $w \in W$ with $\rank w = k$, $s, t \in S$ with
$s \neq t$. Without loss of generality we can choose $w$ such that $w < w \ul s$
and $w < w \ul t$. Assume the existence of an element $u \in wC_{\{s,t\}}$ with
$u \ul s < u$ and $u \ul t < u$. Then \cite[Lemma
3.8]{hultman:comb-twisted-invo} yields $s,t \in D_R(u)$. By using \cite[Lemma
3.9]{hultman:comb-twisted-invo} we conclude that $w \ul s \leq u$ and $w \ul t
\leq u$. Hence there cannot exist more than two Elements of same twisted
length.

If no such $u$ exists, then $wC_{\{s,t\}} = w \ \dot \cup \ \{ w \ul{[st]^n} : n
\in \nn \} \ \dot \cup \ \{ w \ul{[ts]^n} : n \in \nn \}$ and the assumption
still holds.
\end{proof}
\end{lemm}