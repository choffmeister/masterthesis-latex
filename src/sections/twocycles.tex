\section{Zweizykel in der getwisteten schwachen Ordnung}

\begin{defi}[Getwistete und ungetwistete Operation]
Seien $(W,S)$ ein Coxetersystem, $w \in W$ und $s \in S$. Falls $w \ul s =
\theta(s)ws$ ist, so sagen wir, dass $s$ getwistet auf $w$ operiert. Andernfalls
sagen wir $s$ operiert ungetwistet auf $w$.
\end{defi}

\begin{defi}
Seien $(W,S)$ ein Coxetersystem und $s,t \in S$ zwei
verschiedene Erzeuger. Wir definieren:
$$[st]^n :=
\begin{cases}
(st)^{\frac{n}{2}}, & n \textrm{ gerade} \\
(st)^{\frac{n-1}{2}}s, & n \textrm{ ungerade} 
\end{cases}$$
\end{defi}

\begin{defi}[Zweizykel]
Seien $(W,S)$ ein Coxetersystem und $s,t \in S$ zwei verschiedene Erzeuger. Dann
definieren wir $\cyc{w}{s}{t} := \{ w \ul{[st]^n} : n \in \nn \} \cup \{ w
\ul{[ts]^n} : n \in \nn \}$. Diese Menge nennen wir den von $s$ und $t$
erzeugten Zweizykel bezüglich $w$.
\end{defi}

\begin{verm}
Seien $(W,S)$ ein Coxetersystem und $s,t \in S$ zwei verschiedene Erzeuger
von $W$. Dann gilt:
\begin{enumerate}
  \item \label{verm:max-twocycle-length} Sei $\ord(st) = m < \infty$. Falls $w \ul{[st]^n}
  \neq w$ ist für alle $n < 2m$, dann gilt $w \ul{(st)^n} = w$.
  \item \label{verm:twocycle-is-convex} In $\cyc{w}{s}{t}$ existieren keine drei
  Element mit derselben getwisteten Länge.
  \item \label{verm:untwisted-operations-only-at-top-or-bottom-end-of-twocycle}
  Falls $s$ ungetwistet of $w$ operiert, dann gilt $w \ul{st} < w \ul{s}$ oder $w \ul{t} > w$.
  \item \label{verm:twocycle-symmetry} Sei $w \ul{[st]^n} = w$. Dann ist $n$
  gerade und es gilt eine der beiden folgenden Eigenschaften:
  	\begin{enumerate}
  	  \item Die letzte Operation von $w (\ul{st})^{(m)}$ ist genau dann
  	  getwisted, wenn die letzte Operation von $w (\ul{st})^{(n/2+m)}$ getwistet
  	  ist.
  	  \item Die letzte Operation von $w (\ul{st})^{(m)}$ ist genau dann
  	  getwisted, wenn die letzte Operation von $w (\ul{st})^{(n-m+1)}$ getwistet
  	  ist.
  	\end{enumerate}
\end{enumerate}
\end{verm}

\begin{anm}
	Vermutung \ref{verm:twocycle-is-convex} bedeutet, dass Zweizykel in einem
	gewissen Sinne konkav sind. Vermutung
	\ref{verm:untwisted-operations-only-at-top-or-bottom-end-of-twocycle} bedeutet,
	dass innerhalb eines Zweizykels ungetwistete Operationen ausschließlich am
	bzgl. der getwisteten Länge oberen oder unteren Ende auftreten können.
	Vermutung \ref{verm:twocycle-symmetry} bedeutet, dass in einem Zweizykel die
	getwisteten und ungetwisteten Operationen achsen- oder punktsymmetrisch verteilt sind.
\end{anm}