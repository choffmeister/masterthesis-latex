%!TEX root = ../../../_main.tex
\section{Implementing the twisted weak ordering algorithms}
\label{sec:implementing-twisted-involutions-algorithms}

In this section we will look at a concrete implementation of the algorithm \ref{algo:twoa1} from \cite{brennemann:twoa} and \cite{haas:twoa} and of the improved versions \ref{algo:twoa2} and \ref{algo:twoa3}, that we have just introduced. The source codes of the test implementations can be found in the appendix, Section~\ref{sec:sourcecodes}. They are written in \cite{gap}, using the \cite{gap:io} package for reading and write the results to hard disk. It supplies a powerful programming language and can handle with free represented groups, in particular it allows comparisons of elements in such groups. The following algorithm benchmarks have been executed on a computer running Debian Linux in Verion 6.0.5 with an Intel\textsuperscript{\textregistered} Core\textsuperscript{\texttrademark} i7-965 CPU with four cores at 3.2 GHz and 8 GiB RAM. The version 4.5.5 of GAP is used. Note that our implementations do not support multithreding.

At first we compare the count of element comparisons needed for our three algorithms. For this we calculate $Wk(W,\id)$ for a selection of finite Coxeter systems and count the comparisons. In Figure~\ref{fig:twoa123-element-comparisons} we see the count of needed element comparisons plotted against the size of the set of $\id$-twisted involutions.

\begin{figure}[ht]
	\centering
	\begin{tikzpicture}
		\selectcolormodel{gray}
		\pgfplotstableread[col sep=comma]{resources/data/benchmark-twoa1.txt} \dataTwoaOne
		\pgfplotstableread[col sep=comma]{resources/data/benchmark-twoa2.txt} \dataTwoaTwo
		\pgfplotstableread[col sep=comma]{resources/data/benchmark-twoa3.txt} \dataTwoaThree
		\begin{axis}[width=0.9\textwidth,legend pos=south east,grid=major,xmode=log,ymode=log,
			xlabel={Size of $Wk(W,\theta)$},ylabel=Count of element comparisons]
			\addplot table[x=V, y=C] from \dataTwoaOne;
			\addlegendentry{TWOA1}
			\addplot table[x=V, y=C] from \dataTwoaTwo;
			\addlegendentry{TWOA2}
			\addplot table[x=V, y=C] from \dataTwoaThree;
			\addlegendentry{TWOA3}
		\end{axis}
	\end{tikzpicture}
	\caption{Element comparisons needed for calculating $Wk(\id)$}
	\label{fig:twoa123-element-comparisons}
\end{figure}

The first observation is the much lower count of needed element comparisons of \ref{algo:twoa2} and \ref{algo:twoa3} in comparison to \ref{algo:twoa1}, just as we intended it with our improvements. Figure~\ref{fig:twoa123-runtime} plots the runtimes against the size of $Wk(\theta)$. The complete table of benchmark results can be found in the appendix, Section~\ref{sec:benchmarkresults}.

\begin{figure}[ht]
	\centering
	\begin{tikzpicture}
		\selectcolormodel{gray}
		\pgfplotstableread[col sep=comma]{resources/data/benchmark-twoa1.txt} \dataTwoaOne
		\pgfplotstableread[col sep=comma]{resources/data/benchmark-twoa2.txt} \dataTwoaTwo
		\pgfplotstableread[col sep=comma]{resources/data/benchmark-twoa3.txt} \dataTwoaThree
		\begin{axis}[width=0.9\textwidth,legend pos=south east,grid=major,xmode=log,ymode=log,
			xlabel={Size of $Wk(W,\theta)$},ylabel=Runtime in seconds]
			\addplot table[x=V, y=T] from \dataTwoaOne;
			\addlegendentry{TWOA1}
			\addplot table[x=V, y=T] from \dataTwoaTwo;
			\addlegendentry{TWOA2}
			\addplot table[x=V, y=T] from \dataTwoaThree;
			\addlegendentry{TWOA3}
		\end{axis}
	\end{tikzpicture}
	\caption{Runtime for calculating $Wk(\id)$}
	\label{fig:twoa123-runtime}
\end{figure}