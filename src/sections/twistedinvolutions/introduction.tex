\subsection{Introduction to twisted involutions}
\label{sec:twisted-involutions-introduction}

\begin{defi}
	An automorphism $\theta : W \to W$ with $\theta(S) = S$ is called a \defword{Coxeter system automorphism} of $(W,S)$.
\end{defi}

\begin{defi}
	Each $w \in W$ with $\theta(w) = w^{-1}$ is called a \defword{twisted involution}. The set of all twisted involutions in $W$ regarding $\theta$ is denoted with $\ti{\theta}(W)$. Often we will just omit the Coxeter group and write $\ti{\theta}$, when it is clear from the context which $W$ is meant.
\end{defi}

Lets take a quick look at some examples. First of all the trivial one.

\begin{exam}
	Let $\theta = \id_W$. Then $\theta$ is an Coxeter system automorphism and $\ti{\theta} = \{ w \in W : w = w^{-1} \}$.
\end{exam}

The next example is more helpfull, since it reveals a way to think of $\ti{\theta}$ as a generalization of ordinary Coxeter groups.

\begin{exam}
	Let $\theta$ be a automorphism of $W \times W$ with
	$$ \theta : W \times W \to W \times W : (u,w) \mapsto (w,u). $$
	Note that $\theta$ is no Coxeter system automorphism, but we can think of it as one if we identify $S \subset W$ with $S \times S \subset W \times W$.
	Then the set of twisted involutions is
	$$ \ti{\theta} = \{ (w,w^{-1}) \in W \times W : w \in W \}. $$
	This yields a canonical bijection between $\ti{\theta}$ and $W$.
\end{exam}

The map we will define right now is of superior importance to this whole paper, since it is needed to define the poset, the main thesis is about.

\begin{defi}
	\typedlabel{defi:twisted-operation}
	Let $\ul S := \{ \ul s : s \in S \}$ be a set of symbols. Each element in $\ul S$ acts from the right on $W$ by the following definition:
	$$ w \ul s = \begin{cases}
		ws, & \textrm{if } \theta(s)ws = w \\
		\theta(s)ws, & \textrm{else} \\
	\end{cases} $$
	This operation can be extended on the whole free monoid over $\ul S$ by
	$$ w \ul{s_{i_1} s_{i_2} \ldots s_{i_k}} = (\ldots ((w \ul{s_{i_1}}) \ul{s_{i_2}}) \ldots) \ul{s_{i_k}}. $$
\end{defi}

\begin{lemm}
	For all $w \in W$ and $s \in S$ it is $w \ul {ss} = w$.

	\begin{proof}
		For $w \ul s$ there are two cases. Suppose $w \ul s = ws$. This means $\theta(s)ws = w$. For $ws \ul s$ there are again two possible options.
		$$ ws \ul s = \begin{cases}
			wss = w, & \textrm{if } \theta(s)wss = ws \\
			\theta(s)wss = ws, & \textrm{else} \\
		\end{cases} $$
		The second option contradicts itself. So lets now suppose $w \ul s = \theta(s)ws$. This means $\theta(s)ws \neq w$ and for $(\theta(s)ws) \ul s$ there are again two possible options.
		$$ (\theta(s)ws) \ul s = \begin{cases}
			\theta(s)wss = \theta(s)w, & \textrm{if } \theta(s) \theta(s) w ss = \theta(s) w s \\
			\theta(s)\theta(s)wss = w, & \textrm{else} \\
		\end{cases} $$
		The first option is impossible since $\theta(s) \theta(s) w ss = w$ and we have assumed $\theta(s)ws \neq w$. So the only cases possible yield $w \ul s \ul s = w$.
	\end{proof}
\end{lemm}