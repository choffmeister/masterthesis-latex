%!TEX root = ../../../_main.tex
\subsection{Twisted weak ordering}
\label{sec:twisted-involutions-twisted-weak-ordering}

In this section we introduce the twisted weak ordering $Wk(\theta)$ on the set $\ti{\theta}$ of $\theta$-twisted involutions.

\begin{defi}
	Let $\ti{\theta}$ be the set of twisted involutions. For $v,w \in \ti{\theta}$ we define $v \preceq w$ iff there are $\ul s_1, \ldots, \ul s_k \in \ul S$ with $w = v \ul{ s_1 \ldots s_k }$ and $\rho(v) = \rho(w) - k$. We denote the poset $(\ti{\theta},\preceq)$ with $Wk(\theta)$.
\end{defi}

\begin{lemm}
	The poset $Wk(\theta)$ is a graded poset with rank function $\preceq$.

	\begin{proof}
		Follows immediatly from the definition of $\preceq$.
	\end{proof}
\end{lemm}

\begin{exam}
	In Figure \ref{fig:a4} we see Hasse diagram of $Wk(\id)$ on the involutions in the Coxeter group $A_4$. Solid edges represent bothsided actions and dashed edges represent onesided actions.
	\begin{figure}[ht]
		\centering
		\input{resources/tikz/a4}
		\caption{Hasse diagram of $Wk(\id), W = A_4$}
		\label{fig:a4}
	\end{figure}
\end{exam}

\begin{defi}
	Let $v,w \in W$ with $\rho(w) - \rho(v) = n$. A sequence $v = w_0 \prec w_1 \prec \ldots \prec w_n = w$ is called a \defword{geodesic} from $v$ to $w$.
\end{defi}

\todo