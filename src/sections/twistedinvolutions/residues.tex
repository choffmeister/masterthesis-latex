%!TEX root = ../../../_main.tex
\section{Residues}
\label{sec:twisted-involutions-residues}

Residues in $Wk(\theta)$ are subsets of $\theta$-twisted involutions that can be ``reached'' from a fixed starting point by using just certain $\ul s \in \ul S$ as the following definition specifies.

\begin{defi}
	\typedlabel{defi:residue}
	Let $w \in \ti{\theta}$ and $I \subseteq S$ be a subset of generators. Then we define
	$$ wC_I := \{ w \ul s_1 \ldots \ul s_k : k \in \nn_0, s_i \in S \} $$
	as the \defword{$I$-residue} of $w$ or just \defword{residue}. To emphasize the size of $I$, say $|I| = n$, we also speak of a \defword{rank-$n$-residue}.
\end{defi}

\begin{exam}
	Let $w \in \ti{\theta}$. Then $wC_\emptyset = \{ w \}$ and $wC_S = \ti{\theta}$.
\end{exam}

\begin{lemm}
	\typedlabel{lemm:all-elements-of-one-residue-yield-the-same-residue}
	Let $w \in \ti{\theta}$ and $I \subset S$. If $v \in wC_I$, then $vC_I = wC_I$.

	\begin{proof}
		Suppose $v \in wC_I$. Then $v = w \ul s_1 \ldots \ul s_n$ for some $s_i \in I$. Suppose $u = w \ul t_1 \ldots \ul t_m \in wC_I$ is any other element in $wC_I$ with $t_i \in I$. Then
		$$ u = w \ul t_1 \ldots \ul t_m = (v \ul s_n \ldots \ul s_1) \ul t_1 \ldots \ul t_m $$
		and so $u \in vC_I$. This yields $wC_I \subset vC_I$. Since $w \in vC_I$ we can swap $v$ and $w$ to get the other inclusion.
	\end{proof}
\end{lemm}

\begin{coro}
	Let $v, w \in \ti{\theta}$ and $I \subset S$. Then either $vC_I \cap wC_I = \emptyset$ or $vC_I = wC_I$.

	\begin{proof}
		Immediately follows from \ref{lemm:all-elements-of-one-residue-yield-the-same-residue}.
	\end{proof}
\end{coro}

\begin{prop}
	\typedlabel{prop:residues-have-unique-wk-theta-minimal-element}
	\theocite{Lemma 5.6}{hultman:comb-twisted-invo}
	Let $w \in \ti{\theta}$, $I \subseteq S$ be a set of generators. Then there exists a unique element $w_0 \in wC_I$ with $w_0 \preceq w_0 \ul s$ for all $s \in I$.

	\begin{proof}
		Suppose there is no such element. Then for each $w \in wC_I$ we can find a $s \in I$ with $w' = w \ul s \preceq w$ and $e' \in wC_I$. By repetition of \ref{prop:deletion-property-for-twisted-expressions} we get $e \in wC_I$, but $e$ has the property, which we assumed that no element in $wC_I$ has. Hence there must be at least one such element. Now suppose there are two distinct elements $u,v$ with the desired property. Note that this means, that $u$ and $w$ have no reduced twisted expression ending with some $\ul s \in I$. Let $v$ have a reduced twisted expression $v = \ul s_1 \ldots \ul s_k$. Since $u$ and $v$ are both in $wC_I$ there must be a twisted $v$-expression for $u$
		$$ u = v \ul s_{k+1} \ldots \ul s_{k+l} = \ul s_1 \ldots \ul s_{k+l} $$
		with $s_n \in I$ for $k+1 \leq n \leq k+l$. This twisted expression cannot be reduced, since it ends with $\ul s_{k+l} \in I$. Then \ref{prop:deletion-property-for-twisted-expressions} yields that this twisted expression contains a reduced twisted subexpression for $u$. It cannot end with $\ul s_n$ for $k+1 \leq n \leq k+l$. Hence, it is a twisted subexpression of $\ul s_1 \ldots \ul s_k = v$, too. So $u \leq v$ by \ref{theo:bruhat-subexpression-characterization}. Because of symmetry we have $v \leq u$ and so $u = v$, contradicting to our assumption $u \neq v$.
	\end{proof}
\end{prop}

\begin{coro}
	\typedlabel{coro:residues-have-unique-minimal-element}
	Let $w \in \ti{\theta}$, $I \subseteq S$ be a set of generators and let $k$ be the minimal twisted length within the residue $wC_I$. Then the element $w_0$ from \ref{prop:residues-have-unique-wk-theta-minimal-element} is the only element with twisted length $k$.

	\begin{proof}
		The minimal twisted length $k$ is well-defined, since the image of $\rho$ is in $\nn_0$, which is well-ordered, and $wC_I \neq \emptyset$. Suppose some element $v \neq w_0$ has twisted length $k$. Then there is a $s \in I$ with $s \in D_R(v)$ and hence $v \ul s \in wC_I$ with $\rho(v \ul s) < \rho(v) = k$. This contradicts to the minimality of $k$. Since at least one element has twisted length $k$ and only $w_0$ is left we are done.
	\end{proof}
\end{coro}

We proceed with some properties of rank-2-residues. Our interest in these residues stems from the fact that their properties are needed later in Section \ref{sec:twisted-involutions-algorithms} to construct an effective algorithm for calculating the twisted weak ordering, i.e. calculating the Hasse diagram of $Wk(W,\theta)$ for arbitrary Coxeter systems $(W,S)$ and Coxeter system automorphisms $\theta$.

\begin{defi}
	Let $s,t \in S$ be two distinct generators. We define:
	$$[\ul s \ul t]^n :=
	\begin{cases}
	(\ul s \ul t)^{\frac{n}{2}} & n \textrm{ even}, \\
	(\ul s \ul t)^{\frac{n-1}{2}} \ul s & n \textrm{ odd}. 
	\end{cases}$$
\end{defi}

This definition allows us to express rank-2-residues differently. Suppose we have an element $w \in \ti{\theta}$ and two distinct generators $s,t \in S$. Thanks to \ref{lemm:all-elements-of-one-residue-yield-the-same-residue} and \ref{coro:residues-have-unique-minimal-element} we can assume, that $w = min(w,\{s,t\})$. Then
$$ wC_{\{s,t\}} = \{ w \} \ \cup \ \{ w [\ul s \ul t]^n : n \in \nn \} \ \cup \ \{ w [\ul t \ul s]^n : n \in \nn \}. $$
This encourages the following definition.

\begin{defi}
	Let $w \in \ti{\theta}$ and let $s,t \in S$ be two distinct generators. Suppose $w = min(w,\{s,t\})$. Then we call $\{ w [\ul s \ul t]^n : n \in \nn \}$ the \defword{$s$-branch} and $\{ w [\ul t \ul s]^n : n \in \nn \}$ the \defword{$t$-branch} of $wC_{\{s,t\}}$.
\end{defi}

One question arises immediately: Are the $s$- and the $t$-branch disjoint? With the following propositions, corollaries and lemmas we will get a much better idea of the structure of rank-2-residues and answer this question.

\begin{prop}
	\typedlabel{prop:rank-2-residues-are-convex}
	Let $w \in \ti{\theta}$ and let $s,t \in S$ be two distinct generators. Without loss of generality suppose $w = \min(w, \{s,t\})$. If there is a $v \in wC_{\{s,t\}}$ with $v \ul s \prec v$ and $v \ul t \prec v$, then it is unique with this property in $wC_{\{s,t\}}$. Hence $wC_{\{s,t\}}$ consists of two geodesics from $w$ to $v$ intersecting only in these two elements. Else, the $s$- and $t$-branch are disjoint, strictly ascending in twisted length and of infinite size.

	\begin{proof}
		Suppose there is a $v$ in the $s$-branch with $v \ul s \prec v$ and $v \ul t \prec v$, say $v = w[\ul s \ul t]^n$ and $n$ is minimal with this property. Because of the uniqueness of a minimal element from \ref{prop:residues-have-unique-wk-theta-minimal-element} we have $w[\ul s \ul t]^{m+1} \prec w[\ul s \ul t]^m$ for all $m \in \nn$ with $n \leq m \leq 2n - 1$. With the same argument we have $w[\ul s \ul t]^{2n} = w$. If no such $v$ exists, then the $s$- and $t$-branch must be disjoint, strictly ascending in twisted length and so of infinite size.
	\end{proof}
\end{prop}

The assertion that \ref{prop:rank-2-residues-are-convex} makes can be thought of some kind of convexity of rank-2-residues. A rank-2-residue cannot have a concave structure like in Figure~\ref{fig:concave-rank-2-residue}.

\begin{figure}[ht]
	\centering
	%!TEX root = ../../_main.tex
\begin{tikzpicture}[scale=1,bend angle=10]
\newcommand{\xspace}{1}
\newcommand{\yspace}{1}
\tikzstyle{vertex}=[draw,thick,circle,minimum size=2mm,inner sep=0pt]
\tikzstyle{edge}=[thick,->]
\tikzstyle{label}=[fill=white,inner sep=1.5pt]
\node[vertex] (0) at (\xspace*-0.375,\yspace*0) {};
\node[vertex] (1) at (\xspace*-1.5,\yspace*1.5) {};
\node[vertex] (2) at (\xspace*-0.75,\yspace*1.5) {};
\node[vertex] (3) at (\xspace*0.75,\yspace*1.5) {};
\node[vertex] (4) at (\xspace*-1.125,\yspace*3) {};
\node[vertex] (5) at (\xspace*-0.375,\yspace*3) {};
\node[vertex] (6) at (\xspace*0.375,\yspace*3) {};
\node[vertex] (7) at (\xspace*-0,\yspace*4.5) {};
\draw[edge] (0) edge node[label] {$\ul s$} (1);
\draw[edge] (0) edge node[label] {$\ul t$} (3);
\draw[edge] (1) edge node[label] {$\ul t$} (4);
\draw[edge] (2) edge node[label] {$\ul s$} (4);
\draw[edge] (2) edge node[label] {$\ul t$} (5);
\draw[edge] (3) edge node[label] {$\ul s$} (6);
\draw[edge] (5) edge node[label] {$\ul s$} (7);
\draw[edge] (6) edge node[label] {$\ul t$} (7);
\end{tikzpicture}
	\caption{Impossible concave structure of a rank-2-residues}
	\label{fig:concave-rank-2-residue}
\end{figure}

\begin{prop}
	\typedlabel{prop:onesided-operations-only-at-top-or-bottom-end-of-twocycle}
	Let $w \in S$ and $s,t \in S$ be two distinct generators with $w \ul s \prec w$. If $\ul s$ acts by multiplication on $w$, then $w \ul s \ul t \succ w \ul s$ or $w \ul t \prec w$.

	\begin{proof}
		Suppose $w \ul s \ul t \prec w \ul s \prec w$, hence $l(w \ul s \ul t) < l(w \ul s) < l(w)$ in particular. If $\ul t$ acts by multiplication on $w \ul s$, then we have $l(w \ul s \ul t) = l(\theta(s)(wt)) = l(w) - 2$. If it acts by twisted conjugation, then we have $l(w \ul s \ul t) = l(\theta(t)\theta(s)(wt)) = l(w) - 3$. In both cases we have $l(wt) < l(w)$, hence $t \in D_R(w)$ and so $w \ul t \prec w$.
	\end{proof}
\end{prop}

Note that this proposition could be strengthend by insisting on an exclusive or, since we cannot have both cases at the same time. By the proof of \ref{prop:no-triple-edges} we see that we cannot have $w \ul s \ul t = w$, since double edges are always twisted conjugations. Hence having $w \ul s \ul t \succ w \ul s \prec w \succ w \ul t$ would contradict to the convexity from \ref{prop:rank-2-residues-are-convex}. The next corollary ensures that multiplicative actions in $Wk(\theta)$ can only occur at the top or bottom end of rank-2-residues. 

\begin{coro}
	\typedlabel{coro:onesided-operations-only-at-top-or-bottom-end-of-twocycle}
	Let $w \in S$ and let $s,t \in S$ be two distinct generators and suppose $\ul s$ acts by multiplication on $w$. Then $w$ or $w \ul s$ is the unique minimal or maximal element in $wC_{\{s,t\}}$.

	\begin{proof}
		Suppose $w$ is not maximal, i.e. $w \ul t \succ w$. Then by \ref{prop:onesided-operations-only-at-top-or-bottom-end-of-twocycle} we have $w \ul s \ul t \succ w \ul s$, hence $w \ul s$ is minimal. Suppose $w$ is not minimal, i.e. $w \ul s \ul t \prec w \ul s$. Then with the same argument we have $w \ul t \prec w$, hence $w$ is maximal. Supposing $w \ul s$ not to be maximal or not to be minimal yields analogue results.
	\end{proof}
\end{coro}

Again, this corollary can be strengthend by insisting on an exclusive or with the same arguments as before.


\begin{defi}
	Let $w \in \ti{\theta}$, $s,t \in S$ be two distinct generators with $\ord(st) < \infty$ and $C := wC_{\{s,t\}}$ the corrosponding rank-2-residue. We classify rank-2-residues according to Figure~\ref{fig:dist-one-bothsided-actions-in-rank-2-residue}.

	\begin{figure}[ht]
		\centering
		\begin{tabular}{ccc}
			%!TEX root = ../../_main.tex
\begin{tikzpicture}[scale=1,bend angle=10]
\newcommand{\xspace}{1}
\newcommand{\yspace}{0.6}
\tikzstyle{vertex}=[draw,thick,circle,minimum size=2mm,inner sep=1pt]
\tikzstyle{edge}=[thick,->]
\tikzstyle{edge2}=[dotted,thick]
\tikzstyle{onesided}=[edge,dashed]
\tikzstyle{bothsided}=[edge]
\node[vertex, label=270:$w$] (0) at (\xspace*0,\yspace*0) {};
\node[vertex] (1) at (\xspace*-0.375,\yspace*1.5) {};
\node[vertex] (2) at (\xspace*0.375,\yspace*1.5) {};
\node[vertex] (3) at (\xspace*-0.375,\yspace*3.0) {};
\node[vertex] (4) at (\xspace*0.375,\yspace*3.0) {};
\node[vertex, label=90:$v$] (5) at (\xspace*0,\yspace*4.5) {};
\draw[bothsided] (0) edge (1);
\draw[bothsided] (0) edge (2);
\draw[edge2] (1) edge (3);
\draw[edge2] (2) edge (4);
\draw[bothsided] (3) edge (5);
\draw[bothsided] (4) edge (5);
\end{tikzpicture} &
			%!TEX root = ../../_main.tex
\begin{tikzpicture}[scale=1,bend angle=10]
\newcommand{\xspace}{1}
\newcommand{\yspace}{0.6}
\tikzstyle{vertex}=[draw,thick,circle,minimum size=2mm,inner sep=1pt]
\tikzstyle{edge}=[thick,->]
\tikzstyle{edge2}=[dotted,thick]
\tikzstyle{onesided}=[edge,dashed]
\tikzstyle{bothsided}=[edge]
\node[vertex] (0) at (\xspace*0,\yspace*0) {w};
\node[vertex] (1) at (\xspace*-0.375,\yspace*1.5) {};
\node[vertex] (2) at (\xspace*0.375,\yspace*1.5) {};
\node[vertex] (3) at (\xspace*-0.375,\yspace*3.0) {};
\node[vertex] (4) at (\xspace*0.375,\yspace*3.0) {};
\node[vertex] (5) at (\xspace*0,\yspace*4.5) {v};
\draw[onesided] (0) edge (1);
\draw[onesided] (0) edge (2);
\draw[edge2] (1) edge (3);
\draw[edge2] (2) edge (4);
\draw[onesided] (3) edge (5);
\draw[onesided] (4) edge (5);
\end{tikzpicture} &
			%!TEX root = ../../_main.tex
\begin{tikzpicture}[scale=1,bend angle=10]
\newcommand{\xspace}{1}
\newcommand{\yspace}{0.6}
\tikzstyle{vertex}=[draw,thick,circle,minimum size=2mm,inner sep=1pt]
\tikzstyle{edge}=[thick,->]
\tikzstyle{edge2}=[dotted,thick]
\tikzstyle{onesided}=[edge,dashed]
\tikzstyle{bothsided}=[edge]
\node[vertex] (0) at (\xspace*0,\yspace*0) {w};
\node[vertex] (1) at (\xspace*-0.375,\yspace*1.5) {};
\node[vertex] (2) at (\xspace*0.375,\yspace*1.5) {};
\node[vertex] (3) at (\xspace*-0.375,\yspace*3.0) {};
\node[vertex] (4) at (\xspace*0.375,\yspace*3.0) {};
\node[vertex] (5) at (\xspace*0,\yspace*4.5) {v};
\draw[onesided] (0) edge (1);
\draw[onesided] (0) edge (2);
\draw[edge2] (1) edge (3);
\draw[edge2] (2) edge (4);
\draw[bothsided] (3) edge (5);
\draw[bothsided] (4) edge (5);
\end{tikzpicture} \\
			\defword{non-multiplicative} & \defword{maximal-multiplicative} & \defword{bottom-multiplicative} \\
			%!TEX root = ../../_main.tex
\begin{tikzpicture}[scale=1,bend angle=10]
\newcommand{\xspace}{1}
\newcommand{\yspace}{0.6}
\tikzstyle{vertex}=[draw,thick,circle,minimum size=2mm,inner sep=1pt]
\tikzstyle{edge}=[thick,->]
\tikzstyle{edge2}=[dotted,thick]
\tikzstyle{onesided}=[edge,dashed]
\tikzstyle{bothsided}=[edge]
\node[vertex] (0) at (\xspace*0,\yspace*0) {w};
\node[vertex] (1) at (\xspace*-0.375,\yspace*1.5) {};
\node[vertex] (2) at (\xspace*0.375,\yspace*1.5) {};
\node[vertex] (3) at (\xspace*-0.375,\yspace*3.0) {};
\node[vertex] (4) at (\xspace*0.375,\yspace*3.0) {};
\node[vertex] (5) at (\xspace*0,\yspace*4.5) {v};
\draw[bothsided] (0) edge (1);
\draw[bothsided] (0) edge (2);
\draw[edge2] (1) edge (3);
\draw[edge2] (2) edge (4);
\draw[onesided] (3) edge (5);
\draw[onesided] (4) edge (5);
\end{tikzpicture} &
			%!TEX root = ../../_main.tex
\begin{tikzpicture}[scale=1,bend angle=10]
\newcommand{\xspace}{1}
\newcommand{\yspace}{0.6}
\tikzstyle{vertex}=[draw,thick,circle,minimum size=2mm,inner sep=1pt]
\tikzstyle{edge}=[thick,->]
\tikzstyle{edge2}=[dotted,thick]
\tikzstyle{onesided}=[edge,dashed]
\tikzstyle{bothsided}=[edge]
\node[vertex] (0) at (\xspace*0,\yspace*0) {w};
\node[vertex] (1) at (\xspace*-0.375,\yspace*1.5) {};
\node[vertex] (2) at (\xspace*0.375,\yspace*1.5) {};
\node[vertex] (3) at (\xspace*-0.375,\yspace*3.0) {};
\node[vertex] (4) at (\xspace*0.375,\yspace*3.0) {};
\node[vertex] (5) at (\xspace*0,\yspace*4.5) {v};
\draw[bothsided] (0) edge (1);
\draw[onesided] (0) edge (2);
\draw[edge2] (1) edge (3);
\draw[edge2] (2) edge (4);
\draw[onesided] (3) edge (5);
\draw[bothsided] (4) edge (5);
\end{tikzpicture} &
			%!TEX root = ../../_main.tex
\begin{tikzpicture}[scale=1,bend angle=10]
\newcommand{\xspace}{1}
\newcommand{\yspace}{0.6}
\tikzstyle{vertex}=[draw,thick,circle,minimum size=2mm,inner sep=1pt]
\tikzstyle{edge}=[thick,->]
\tikzstyle{edge2}=[dotted,thick]
\tikzstyle{onesided}=[edge,dashed]
\tikzstyle{bothsided}=[edge]
\node[vertex, label=270:$w$] (0) at (\xspace*0,\yspace*0) {};
\node[vertex] (1) at (\xspace*-0.375,\yspace*1.5) {};
\node[vertex] (2) at (\xspace*0.375,\yspace*1.5) {};
\node[vertex] (3) at (\xspace*-0.375,\yspace*3.0) {};
\node[vertex] (4) at (\xspace*0.375,\yspace*3.0) {};
\node[vertex, label=90:$v$] (5) at (\xspace*0,\yspace*4.5) {};
\draw[bothsided] (0) edge (1);
\draw[onesided] (0) edge (2);
\draw[edge2] (1) edge (3);
\draw[edge2] (2) edge (4);
\draw[onesided] (3) edge (5);
\draw[bothsided] (4) edge (5);
\end{tikzpicture} \\
			\defword{top-multiplicative} & \multicolumn{2}{c}{\defword{diagonal-multiplicative}}
		\end{tabular}
		\caption{Classification of rank-2-residues}
		\label{fig:dist-one-bothsided-actions-in-rank-2-residue}
	\end{figure}
\end{defi}

\begin{lemm}
	\typedlabel{lemm:one-both-sided-action-symmetric-in-rank-2-residues}
	Let $s,t \in S$ be two distinct generators and $w \in S$ with $w = min(w,\{s,t\})$. Suppose $v \in wC_{\{s,t\}}$ with $v \ul s \prec v$ and $v \ul t \prec v$. Then $wC_{\{s,t\}}$ is either non-, maximal-, bottom-, top- or diagonal-multiplicative. In particular the twisted conjugations and mulitplications are distributed axisymmetrically or pointsymmetrically.

	\begin{proof}
		If $u$ covers $w$, then there are only two edges and the assumption holds. So suppose $wC_{\{s,t\}}$ contains at least four edges. Due to \ref{coro:onesided-operations-only-at-top-or-bottom-end-of-twocycle} the actions by multiplication can only occur next to $w$ and $v$. Hence there are $2^4 = 16$ configurations possible. \ref{prop:geodesics-have-same-count-of-multiplicative-steps} wipes out ten out of the 16 configurations. The remaining are those from Figure \ref{fig:dist-one-bothsided-actions-in-rank-2-residue}.
	\end{proof}
\end{lemm}

\begin{exam}
	In Figure \ref{fig:a4_s1s3-and-a4_s2s4} we see three Hasse diagrams of $Wk(A_4, \id)$. The left one only contains edges with labels $s_1,s_2$, the middle one only edges with labels $s_1,s_3$ and the right one only edges with labels $s_1,s_4$.
	\begin{figure}[ht]
		\centering
		\begin{tikzpicture}[scale=1,bend angle=10]
\newcommand{\xspace}{1}
\newcommand{\yspace}{1}
\tikzstyle{vertex}=[draw,thick,circle,minimum size=2mm,inner sep=0pt]
\tikzstyle{edge}=[thick,->]
\tikzstyle{onesided}=[edge,dashed]
\tikzstyle{bothsided}=[edge]
\tikzstyle{unhighlighted}=[]
\tikzstyle{highlighted}=[]
\definecolor{s1color}{RGB}{130,76,253}
\definecolor{s2color}{RGB}{76,253,78}
\definecolor{s3color}{RGB}{253,76,124}
\definecolor{s4color}{RGB}{76,176,253}
\tikzstyle{s1}=[s1color]
\tikzstyle{s2}=[s2color]
\tikzstyle{s3}=[s3color]
\tikzstyle{s4}=[s4color]
\node[vertex,unhighlighted] (0) at (\xspace*0,\yspace*0) {};
\node[vertex,unhighlighted] (1) at (\xspace*-1.125,\yspace*1.5) {};
\node[vertex,unhighlighted] (2) at (\xspace*-0.375,\yspace*1.5) {};
\node[vertex,unhighlighted] (3) at (\xspace*0.375,\yspace*1.5) {};
\node[vertex,unhighlighted] (4) at (\xspace*1.125,\yspace*1.5) {};
\node[vertex,unhighlighted] (5) at (\xspace*-1.875,\yspace*3) {};
\node[vertex,unhighlighted] (6) at (\xspace*-1.125,\yspace*3) {};
\node[vertex,unhighlighted] (7) at (\xspace*-0.375,\yspace*3) {};
\node[vertex,unhighlighted] (8) at (\xspace*0.375,\yspace*3) {};
\node[vertex,unhighlighted] (9) at (\xspace*1.125,\yspace*3) {};
\node[vertex,unhighlighted] (10) at (\xspace*1.875,\yspace*3) {};
\node[vertex,unhighlighted] (11) at (\xspace*-1.875,\yspace*4.5) {};
\node[vertex,unhighlighted] (12) at (\xspace*-1.125,\yspace*4.5) {};
\node[vertex,unhighlighted] (13) at (\xspace*-0.375,\yspace*4.5) {};
\node[vertex,unhighlighted] (14) at (\xspace*0.375,\yspace*4.5) {};
\node[vertex,unhighlighted] (15) at (\xspace*1.125,\yspace*4.5) {};
\node[vertex,unhighlighted] (16) at (\xspace*1.875,\yspace*4.5) {};
\node[vertex,unhighlighted] (17) at (\xspace*-1.5,\yspace*6) {};
\node[vertex,unhighlighted] (18) at (\xspace*-0.75,\yspace*6) {};
\node[vertex,unhighlighted] (19) at (\xspace*0,\yspace*6) {};
\node[vertex,unhighlighted] (20) at (\xspace*0.75,\yspace*6) {};
\node[vertex,unhighlighted] (21) at (\xspace*1.5,\yspace*6) {};
\node[vertex,unhighlighted] (22) at (\xspace*-0.75,\yspace*7.5) {};
\node[vertex,unhighlighted] (23) at (\xspace*0,\yspace*7.5) {};
\node[vertex,unhighlighted] (24) at (\xspace*0.75,\yspace*7.5) {};
\node[vertex,unhighlighted] (25) at (\xspace*0,\yspace*9) {};
\draw[s1,onesided,unhighlighted] (0) edge (1);
\draw[s2,onesided,unhighlighted] (0) edge (2);
\draw[s2,bothsided,unhighlighted] (1) edge (5);
\draw[s1,bothsided,unhighlighted] (2) edge (5);
\draw[s1,onesided,unhighlighted] (3) edge (6);
\draw[s2,bothsided,unhighlighted] (3) edge (8);
\draw[s1,onesided,unhighlighted] (4) edge (7);
\draw[s2,onesided,unhighlighted] (4) edge (9);
\draw[s2,bothsided,unhighlighted] (6) edge (13);
\draw[s2,bothsided,unhighlighted] (7) edge (12);
\draw[s1,bothsided,unhighlighted] (8) edge (11);
\draw[s1,bothsided,unhighlighted] (9) edge (12);
\draw[s1,onesided,unhighlighted] (10) edge (14);
\draw[s2,bothsided,unhighlighted] (10) edge (15);
\draw[s2,onesided,unhighlighted] (11) edge (17);
\draw[s1,bothsided,unhighlighted,bend right] (13) edge (17);
\draw[s2,bothsided,unhighlighted] (14) edge (20);
\draw[s1,bothsided,unhighlighted] (15) edge (18);
\draw[s1,bothsided,unhighlighted] (16) edge (19);
\draw[s2,bothsided,unhighlighted,bend right] (16) edge (21);
\draw[s2,onesided,unhighlighted] (18) edge (22);
\draw[s2,bothsided,unhighlighted] (19) edge (24);
\draw[s1,bothsided,unhighlighted] (20) edge (22);
\draw[s1,bothsided,unhighlighted] (21) edge (23);
\draw[s2,bothsided,unhighlighted] (23) edge (25);
\draw[s1,bothsided,unhighlighted,bend right] (24) edge (25);
\end{tikzpicture}
		\quad \quad
		%!TEX root = ../../_main.tex
\begin{tikzpicture}[scale=1,bend angle=10]
\newcommand{\xspace}{1}
\newcommand{\yspace}{1}
\tikzstyle{vertex}=[draw,thick,circle,minimum size=2mm,inner sep=0pt]
\tikzstyle{edge}=[thick,->]
\tikzstyle{onesided}=[edge,dashed]
\tikzstyle{bothsided}=[edge]
\tikzstyle{unhighlighted}=[]
\tikzstyle{highlighted}=[]
\definecolor{s1color}{RGB}{130,76,253}
\definecolor{s2color}{RGB}{76,253,78}
\definecolor{s3color}{RGB}{253,76,124}
\definecolor{s4color}{RGB}{76,176,253}
\tikzstyle{s1}=[s1color]
\tikzstyle{s2}=[s2color]
\tikzstyle{s3}=[s3color]
\tikzstyle{s4}=[s4color]
\node[vertex,unhighlighted] (0) at (\xspace*0,\yspace*0) {};
\node[vertex,unhighlighted] (1) at (\xspace*-1.125,\yspace*1.5) {};
\node[vertex,unhighlighted] (2) at (\xspace*-0.375,\yspace*1.5) {};
\node[vertex,unhighlighted] (3) at (\xspace*0.375,\yspace*1.5) {};
\node[vertex,unhighlighted] (4) at (\xspace*1.125,\yspace*1.5) {};
\node[vertex,unhighlighted] (5) at (\xspace*-1.875,\yspace*3) {};
\node[vertex,unhighlighted] (6) at (\xspace*-1.125,\yspace*3) {};
\node[vertex,unhighlighted] (7) at (\xspace*-0.375,\yspace*3) {};
\node[vertex,unhighlighted] (8) at (\xspace*0.375,\yspace*3) {};
\node[vertex,unhighlighted] (9) at (\xspace*1.125,\yspace*3) {};
\node[vertex,unhighlighted] (10) at (\xspace*1.875,\yspace*3) {};
\node[vertex,unhighlighted] (11) at (\xspace*-1.875,\yspace*4.5) {};
\node[vertex,unhighlighted] (12) at (\xspace*-1.125,\yspace*4.5) {};
\node[vertex,unhighlighted] (13) at (\xspace*-0.375,\yspace*4.5) {};
\node[vertex,unhighlighted] (14) at (\xspace*0.375,\yspace*4.5) {};
\node[vertex,unhighlighted] (15) at (\xspace*1.125,\yspace*4.5) {};
\node[vertex,unhighlighted] (16) at (\xspace*1.875,\yspace*4.5) {};
\node[vertex,unhighlighted] (17) at (\xspace*-1.5,\yspace*6) {};
\node[vertex,unhighlighted] (18) at (\xspace*-0.75,\yspace*6) {};
\node[vertex,unhighlighted] (19) at (\xspace*0,\yspace*6) {};
\node[vertex,unhighlighted] (20) at (\xspace*0.75,\yspace*6) {};
\node[vertex,unhighlighted] (21) at (\xspace*1.5,\yspace*6) {};
\node[vertex,unhighlighted] (22) at (\xspace*-0.75,\yspace*7.5) {};
\node[vertex,unhighlighted] (23) at (\xspace*0,\yspace*7.5) {};
\node[vertex,unhighlighted] (24) at (\xspace*0.75,\yspace*7.5) {};
\node[vertex,unhighlighted] (25) at (\xspace*0,\yspace*9) {};
\draw[s1,onesided,unhighlighted] (0) edge (1);
\draw[s3,onesided,unhighlighted] (0) edge (3);
\draw[s3,onesided,unhighlighted] (1) edge (6);
\draw[s1,bothsided,unhighlighted] (2) edge (5);
\draw[s3,bothsided,unhighlighted] (2) edge (8);
\draw[s1,onesided,unhighlighted] (3) edge (6);
\draw[s1,onesided,unhighlighted] (4) edge (7);
\draw[s3,bothsided,unhighlighted] (4) edge (10);
\draw[s3,bothsided,unhighlighted] (5) edge (11);
\draw[s3,bothsided,unhighlighted] (7) edge (14);
\draw[s1,bothsided,unhighlighted] (8) edge (11);
\draw[s1,bothsided,unhighlighted] (9) edge (12);
\draw[s3,bothsided,unhighlighted] (9) edge (16);
\draw[s1,onesided,unhighlighted] (10) edge (14);
\draw[s3,bothsided,unhighlighted] (12) edge (19);
\draw[s1,bothsided,unhighlighted,bend right] (13) edge (17);
\draw[s3,bothsided,unhighlighted,bend left] (13) edge (17);
\draw[s1,bothsided,unhighlighted] (15) edge (18);
\draw[s3,onesided,unhighlighted] (15) edge (21);
\draw[s1,bothsided,unhighlighted] (16) edge (19);
\draw[s3,onesided,unhighlighted] (18) edge (23);
\draw[s1,bothsided,unhighlighted] (20) edge (22);
\draw[s3,bothsided,unhighlighted] (20) edge (24);
\draw[s1,bothsided,unhighlighted] (21) edge (23);
\draw[s3,bothsided,unhighlighted] (22) edge (25);
\draw[s1,bothsided,unhighlighted,bend right] (24) edge (25);
\end{tikzpicture}
		\quad \quad
		%!TEX root = ../../_main.tex
\begin{tikzpicture}[scale=0.8,bend angle=10]
\newcommand{\xspace}{1}
\newcommand{\yspace}{1}
\tikzstyle{vertex}=[draw,thick,circle,minimum size=2mm,inner sep=0pt]
\tikzstyle{edge}=[thick,->]
\tikzstyle{onesided}=[edge,dashed]
\tikzstyle{bothsided}=[edge]
\tikzstyle{unhighlighted}=[]
\tikzstyle{highlighted}=[]
\definecolor{s1color}{RGB}{130,76,253}
\definecolor{s2color}{RGB}{76,253,78}
\definecolor{s3color}{RGB}{253,76,124}
\definecolor{s4color}{RGB}{76,176,253}
\tikzstyle{s1}=[]
\tikzstyle{s2}=[]
\tikzstyle{s3}=[]
\tikzstyle{s4}=[]
\node[vertex,unhighlighted] (0) at (\xspace*0,\yspace*0) {};
\node[vertex,unhighlighted] (1) at (\xspace*-1.125,\yspace*1.5) {};
\node[vertex,unhighlighted] (2) at (\xspace*-0.375,\yspace*1.5) {};
\node[vertex,unhighlighted] (3) at (\xspace*0.375,\yspace*1.5) {};
\node[vertex,unhighlighted] (4) at (\xspace*1.125,\yspace*1.5) {};
\node[vertex,unhighlighted] (5) at (\xspace*-1.875,\yspace*3) {};
\node[vertex,unhighlighted] (6) at (\xspace*-1.125,\yspace*3) {};
\node[vertex,unhighlighted] (7) at (\xspace*-0.375,\yspace*3) {};
\node[vertex,unhighlighted] (8) at (\xspace*0.375,\yspace*3) {};
\node[vertex,unhighlighted] (9) at (\xspace*1.125,\yspace*3) {};
\node[vertex,unhighlighted] (10) at (\xspace*1.875,\yspace*3) {};
\node[vertex,unhighlighted] (11) at (\xspace*-1.875,\yspace*4.5) {};
\node[vertex,unhighlighted] (12) at (\xspace*-1.125,\yspace*4.5) {};
\node[vertex,unhighlighted] (13) at (\xspace*-0.375,\yspace*4.5) {};
\node[vertex,unhighlighted] (14) at (\xspace*0.375,\yspace*4.5) {};
\node[vertex,unhighlighted] (15) at (\xspace*1.125,\yspace*4.5) {};
\node[vertex,unhighlighted] (16) at (\xspace*1.875,\yspace*4.5) {};
\node[vertex,unhighlighted] (17) at (\xspace*-1.5,\yspace*6) {};
\node[vertex,unhighlighted] (18) at (\xspace*-0.75,\yspace*6) {};
\node[vertex,unhighlighted] (19) at (\xspace*0,\yspace*6) {};
\node[vertex,unhighlighted] (20) at (\xspace*0.75,\yspace*6) {};
\node[vertex,unhighlighted] (21) at (\xspace*1.5,\yspace*6) {};
\node[vertex,unhighlighted] (22) at (\xspace*-0.75,\yspace*7.5) {};
\node[vertex,unhighlighted] (23) at (\xspace*0,\yspace*7.5) {};
\node[vertex,unhighlighted] (24) at (\xspace*0.75,\yspace*7.5) {};
\node[vertex,unhighlighted] (25) at (\xspace*0,\yspace*9) {};
\draw[s1,onesided,unhighlighted] (0) edge (1);
\draw[s4,onesided,unhighlighted] (0) edge (4);
\draw[s4,onesided,unhighlighted] (1) edge (7);
\draw[s1,bothsided,unhighlighted] (2) edge (5);
\draw[s4,onesided,unhighlighted] (2) edge (9);
\draw[s1,onesided,unhighlighted] (3) edge (6);
\draw[s4,bothsided,unhighlighted] (3) edge (10);
\draw[s1,onesided,unhighlighted] (4) edge (7);
\draw[s4,onesided,unhighlighted] (5) edge (12);
\draw[s4,bothsided,unhighlighted] (6) edge (14);
\draw[s1,bothsided,unhighlighted] (8) edge (11);
\draw[s4,bothsided,unhighlighted] (8) edge (15);
\draw[s1,bothsided,unhighlighted] (9) edge (12);
\draw[s1,onesided,unhighlighted] (10) edge (14);
\draw[s4,bothsided,unhighlighted] (11) edge (18);
\draw[s1,bothsided,unhighlighted,bend right] (13) edge (17);
\draw[s4,bothsided,unhighlighted] (13) edge (20);
\draw[s1,bothsided,unhighlighted] (15) edge (18);
\draw[s1,bothsided,unhighlighted] (16) edge (19);
\draw[s4,bothsided,unhighlighted,bend left] (16) edge (21);
\draw[s4,bothsided,unhighlighted] (17) edge (22);
\draw[s4,bothsided,unhighlighted] (19) edge (23);
\draw[s1,bothsided,unhighlighted] (20) edge (22);
\draw[s1,bothsided,unhighlighted] (21) edge (23);
\draw[s1,bothsided,unhighlighted,bend right] (24) edge (25);
\draw[s4,bothsided,unhighlighted,bend left] (24) edge (25);
\end{tikzpicture}
		\caption{Hasse diagrams of $Wk(A_4, \id)$ after removing $s_3,s_4$ edges in the left, $s_2,s_4$ edges in the middle and $s_2,s_3$ edges in the right diagram}
		\label{fig:a4_s1s3-and-a4_s2s4}
	\end{figure}
\end{exam}

\begin{coro}
	\typedlabel{coro:deduction-of-s-action-in-rank-2-residue}

	Let $wC_{\{s,t\}}$ be a finite rank-2-residue and suppose $w = \min(C,\{s,t\})$. Let $n \in \nn$ be the smallest number with $w[\ul s \ul t]^{2n} = w$. Define
	\begin{align*}
		a_1 & = l(w \ul s) - l(w), \\
		a_2 & = l(w[\ul s \ul t]^{n}) - l(w[\ul s \ul t]^{n-1}), \\
		a_3 & = l(w \ul t) - l(w), \\
		a_4 & = l(w[\ul t \ul s]^{n}) - l(w[\ul t \ul s]^{n-1}).
	\end{align*}
	Then $a_1 + a_2 = a_3 + a_4$.

	\begin{proof}
		There are two geodesics from $w$ to the unique maximal element in $wC_{\{s,t\}}$. By \ref{lemm:one-both-sided-action-symmetric-in-rank-2-residues} in both geodesics at most the first and last step can be multiplicative and by \ref{prop:geodesics-have-same-count-of-multiplicative-steps} both geodesics must contain the same count of multiplicative steps. The sum $4-a_1-a_2$ is the count of multiplicative steps in the first geodesic while $4-a_3-a_4$ is the count of multiplicative steps in the second geodesic. Hence $4-a_1-a_2 = 4-a_3-a_4 \iff a_1+a_2 = a_3+a_4$.
	\end{proof}
\end{coro}

\begin{lemm}
	\typedlabel{lemm:max-twisted-circle-height}
	Let $w \in S$, $s,t \in S$ be two distinct generators and $m = \ord(st) < \infty$. Then $|wC_{\{s,t\}}| \leq 2m$.

	\begin{proof}
		Without loss of generality let $w = \min(C,\{s,t\})$. Let $n$ be the smallest number (or $\infty$ if non such number exists) for that $w[\ul s \ul t]^{n} \succ w[\ul s \ul t]^{n+1}$. Then by the convexity from \ref{prop:rank-2-residues-are-convex} we have $|wC_{\{s,t\}}| = 2n$. If $n < m$, then we are done. In return assume $n \geq m$, i.e. $w \prec w \ul s \prec \ldots \prec w[\ul s \ul t]^{m}$. But $w[\ul s \ul t]^{m}$ ends with $stst \cdots$ ($m$-times) and hence by
		$$ \underbrace{stst \cdots}_{m-\textrm{times}} = \underbrace{tsts \cdots}_{m-\textrm{times}} $$
		we get $s,t \in D_R(w[\ul s \ul t]^{m})$. So by \ref{lemm:rho-w-ul-s-minus-rho-w-differs-by-1} we have $w[\ul s \ul t]^{m} \succ w[\ul s \ul t]^{m+1}$, hence $n = m$ and so $|wC_{\{s,t\}}| = 2m$.
	\end{proof}
\end{lemm}

\begin{prop}
	\typedlabel{prop:rank-2-residue-minimal-k}
	Let $w \in S$ and $s,t \in S$ be two distinct generators with $\ord(st) < \infty$. Suppose $k \in \nn$ to be the smallest number with $w = w(\ul s \ul t)^k$. Then for any $n \in \nn$ with $w = w(\ul s \ul t)^n$ we have $k \ | \ n$.

	\begin{proof}
		Let $n = qk + r$ for $q \in \nn_0$ and $r \in \{0,\ldots,k-1\}$. Then
		$$ w = w(\ul s \ul t)^n = w(\ul s \ul t)^{qk+r} = w(\ul s \ul t)^{qk} (\ul s \ul t)^r = w (\ul s \ul t)^{q(k-1)} (\ul s \ul t)^r = \ldots = w(\ul s \ul t)^r. $$
		For $r > 0$ we would have a contradiction to the minimality of $k$, hence $r = 0$, $q > 0$ and therefore $k \ | \ n$.
	\end{proof}
\end{prop}

\begin{coro}
	\typedlabel{coro:rank-2-residue-gcd-n-m}
	Let $w \in S$ and $s,t \in S$ be two distinct generators with $w \ul s \neq w \ul t$. Suppose $w = w(\ul s \ul t)^m = w(\ul s \ul t)^n$. Then $\gcd(m,n) > 1$.

	\begin{proof}
		Let $k$ be the same as in \ref{prop:rank-2-residue-minimal-k}. Since $w \ul s \neq w \ul t$ we have $k > 1$. Both, $k \ | \ n$ and $k \ | \ m$, hence $\gcd(m,n) \geq k > 1$.
	\end{proof}
\end{coro}

This constraints the possible size of rank-2-residues.