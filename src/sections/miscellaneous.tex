\section{Miscellaneous}

\begin{ques}
	\typedlabel{ques:main}
	Let $(W,S)$ be a Coxeter system, $\theta : W \to W$ an automorphism of $W$ with $\theta^2 = \id$ and $\theta(S) = S$, and $K \subset S$ a subset of $S$ generating a finite subgroup of $W$ with $\theta(K) = K$. Futhermore let $T,S_1,S_2,S_3 \subset S$ be four pairwise disjoint sets of generators. For which Coxeter groups $W$ does the implication
	\begin{equation}
		\label{eq:main}
		w \in w_K C_{T \cup S_i}, i=1,2,3 \Rightarrow w \in w_K C_T
	\end{equation}
	hold for any possible $K,\theta,T,S_1,S_2,S_3$ and $w$?
\end{ques}

\begin{prop}
	\typedlabel{prop:counterexample-simplification}
	Let $(W,S)$ be a Coxeter system and $K,T,S_1,S_2,S_3$ be like in \ref{ques:main}. Suppose we have $w \in W$ and $a_1,\ldots,a_n \in T \cup S_1$, $b_1,\ldots,b_n \in T \cup S_2$, $c_1,\ldots,c_n \in T \cup S_3$ with
	\begin{align*}
	w & = w_K \ul{ a_1 \cdots a_n } \\
	  & = w_K \ul{ b_1 \cdots b_n } \\
	  & = w_K \ul{ c_1 \cdots c_n }
	\end{align*}
	and \eqref{eq:main} does not hold for these three expressions, i.e. $w \notin w_K C_T$. Then there exist $t_1,\ldots,t_m \in T$ and $a'_1,\ldots,a'_{n-m} \in T \cup S_1$, $b'_1,\ldots,b'_{n-m} \in T \cup S_2$, $c'_1,\ldots,c'_{n-m} \in T \cup S_3$ such that
	\begin{align*}
		w \ul {t_1 \ldots t_m} & = w_K \ul{ a'_1 \cdots a'_{n-m} } \\
							   & = w_K \ul{ b'_1 \cdots b'_{n-m} } \\
							   & = w_K \ul{ c'_1 \cdots c'_{n-m} }
	\end{align*}
	with $a'_{n-m},b'_{n-m},c'_{n-m} \notin T$.

	\begin{proof}
		Suppose at least one element of $a_n,b_n,c_n$ to be in $T$, for example $a_n \in T$. Then we can apply $\ul{a_n}$ to all three expressions. Since $\rho (w \ul{a_n}) < \rho(w)$ the exchange condition for $\ti{\theta}$ \cite[Proposition 3.10]{hultman:comb-twisted-invo}yields
		\begin{align*}
			w \ul{a_n} & = w_K \ul{ a_1 \cdots a_n a_n } = w_K \ul{ a_1 \cdots a_{n-1} } \\
					   & = w_K \ul{ b_1 \cdots b_n a_n } = w_K \ul{ b_1 \cdots \hat b_i \cdots b_n } \\
					   & = w_K \ul{ c_1 \cdots c_n a_n } = w_K \ul{ c_1 \cdots \hat c_j \cdots c_n }
		\end{align*}
		where $\hat \cdot$ means omission. The omission cannot occur within $w_K$ since all three expressions are still of same twisted length and in the first expression we can see, that $w_K \leq w \ul {a_n}$ still holds. This step can be repeated until $w = w_K$ or $a_n,b_n,c_n \notin T$.
	\end{proof}
\end{prop}

\begin{lemm}
	\typedlabel{lemm:counterexample-simplification}
	A counterexample to \ref{ques:main} can only exist, if there
	is an element $u \in w C_T$ and three distinct generators $s_1,s_2,s_3 \in
	D_r(u)$ such that $u \ul {s_i} \notin w C_T$ for $i=1,2,3$.

	\begin{proof}
		According to \ref{prop:counterexample-simplification}.
	\end{proof}
\end{lemm}

\begin{lemm}
	\typedlabel{lemm:counterexample-simplification2}
	A counterexample to \ref{ques:main} can only exist, if there are three not
	neseccarily distinct elements $a,b,c \in w_K C_{S \setminus T}$, three
	distinct generators $s_1 \in A_r(a)$, $s_2 \in A_r(b)$, $s_3 \in A_r(c)$ and an
	element $u \notin w_K C_{S \setminus T}$ such that
	$$ a \ul{s_1} = b \ul{s_2} = c \ul{s_3} = u. $$

	\begin{proof}
		If there is a counterexample, then the two residuums $w_K C_{S \setminus T}$ and
		$w C_T$ are disjunct. Since we are only interested in $w$ with $w_K \leq w$
		it follows, that any geodesic from $w_K$ to $w$ is contained in the union set
		of both residuums. Hence having one element in $u \in w C_T$ with three distinct
		generators $s_1,s_2,s_3$ with $u \ul{s_i} \notin w C_T$ is equivalent to having
		three elements $a,b,c \notin w C_T$ and the same three generator $s_1,s_2,s_3$
		with $a \ul{s_1} = b \ul{s_2} = c \ul{s_3} = u \in w C_T$.
	\end{proof}
\end{lemm}