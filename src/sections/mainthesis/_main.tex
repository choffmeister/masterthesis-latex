%!TEX root = ../../../_main.tex
\section{Main Thesis}
\label{sec:main-thesis}

\begin{ques}
	\typedlabel{ques:main}
	Let $(W,S)$ be a Coxeter system, $\theta : W \to W$ an automorphism of $W$ with $\theta^2 = \id$ and $\theta(S) = S$, and $K \subset S$ a subset of $S$ generating a finite subgroup of $W$ with $\theta(K) = K$. Futhermore let $T,S_1,S_2,S_3 \subset S$ be four pairwise disjoint sets of generators. For which Coxeter groups $W$ does the implication
	\begin{equation}
		\label{eq:main}
		w \in w_K C_{T \cup S_i}, i=1,2,3 \Rightarrow w \in w_K C_T
	\end{equation}
	hold for any possible $K,\theta,T,S_1,S_2,S_3$ and $w$?
\end{ques}

The reader might wonder, why we handle with three sets $S_1,S_2,S_3$ and not just with two. The reason for that is also the main reason, why $Wk(\theta)$ is less accessible than $\Br(\ti{\theta})$: In $Wk(\theta)$ there is the possibilty for $w \ul s = w \ul t$ for two distinct generators $s,t \in S$. Within the Hasse diagram this situation appears in form of double edges between two vertices. For example, let $W = A_3$ and $\theta$ be the Coxeter system automorphism swapping $s_1$ with $s_3$. Then we have $e \ul s_1 = s_3 s_1 = s_1 s_3 = e \ul s_3$. Double edges can also occure for $\theta = \id$, but in this situation they cannot appear next to the neutral element $e$, since $\theta(s)es = e$ for all $s \in S$, hence $e \ul s = s \neq t = e \ul t$ for all $s,t \in S$ with $s \neq t$. Therefore, if we had written \ref{eq:main} with just two sets, then it would be false immediately for any Coxeter system automorphism, that swaps two commutating generators and presumably there could be found many more counterexamples.

\begin{prop}
	\typedlabel{coro:no-triple-edges}
	Let $w \in W$ and $w \ul s \succ w$. Then $|\{ t \in S \setminus D_R(w) : w \ul t = w \ul s \}| \in \{1,2\}$.

	\begin{proof}
		Suppose $t \in S \setminus D_R(w)$ with $w \ul t = w \ul s$. Because of the ordinary length either both $\ul s$ and $\ul t$ act by multiplication on $w$, or both act by twisted conjugation on $w$. Suppose they act by multiplication, then $ws = w \ul s = w \ul t = wt$, hence $s = t$. Conversely, assume they act by twisted conjugation. Then $\theta(s) w s = w \ul s = w \ul t = \theta(t)wt$. Because of $\theta(t) w t t = \theta(t) w = \theta(s) w s t$ we have $l(\theta(s) w s t) < l(\theta(s) w s)$ and so by \ref{coro:exchange-condition} there are three possible cases
		$$ \theta(t)w = \theta(s) w s t = \begin{cases}
			\theta(s) w & \Rightarrow s = t, \\
			w s & \Rightarrow \theta(t) = w s w^{-1} \textrm{ or} \\
			\theta(s) \overline w s & \Rightarrow w = \theta(t) \theta(s) \overline w s, \\
		\end{cases} $$
		where $\overline w$ denotes a well choosen subexpression of $w$. The first case is trivial, the second determines $t$ unambiguously. The third case is impossible, since by \ref{coro:exchange-condition} and \ref{rema:exchange-condition-left-sided} we would have a reduced expression for $w$ beginning with $\theta(s)$ or ending with $s$ (or both), yielding $l(\theta(s)ws) \leq l(w)$, which contradicts to $\rho(w \ul s) = \rho(\theta(s)ws) > \rho(w)$. Therefore, there cannot be more than two distinct $s,t \in S \setminus D_R(w)$ with $w \ul s = w \ul t$.
	\end{proof}
\end{prop}

A much stronger version of \ref{ques:main} proves to be false.

\begin{ques}
	\typedlabel{ques:cap-of-three-residuums}
	Let $w \in \ti{\theta}$ and $S_1,S_2,S_3 \subseteq S$. Is $wC_{S_1} \cap wC_{S_2} \cap wC_{S_3} = wC_{S_1 \cap S_2 \cap S_3}$?
	\todo
\end{ques}

\begin{lemm}
	Let $w \in \ti{\theta}$ and $u = w \ul s_1 \ldots \ul s_k = w \ul t_1 \ldots \ul t_k$ be two reduced twisted expressions with same prefix $w$. Then $u \ul t_k = w \ul s_1 \ldots \ul{\hat s}_i \ldots \ul s_k$ for some $i = 1,\ldots,k$.

	\begin{proof}
		By \ref{prop:exchanged-property-for-twisted-expressions} there must be the possibility for omission, but it could be the case, that only an omission within the prefix $w$ yields an expression for $u \ul t_k$. Suppose this is the case, i.e. our hypothesis is false. By the proof of \ref{prop:exchanged-property-for-twisted-expressions} we have $u \ul t_k \ul s_k \ldots \ul s_i < u \ul t_k \ul s_k \ldots \ul s_{i+1}$ for all $i = 1,\ldots,k$. Then $u \ul t_k < u$, $s_k \in D_R(u)$ and $s_k \in D_R(u \ul t_k)$, hence by \ref{lemm:lifting-property-for-ul-s} we have $u \ul t_k \ul s_k < u \ul s_k$. Since we assumed, that $u \ul t_k \ul s_k \ldots \ul s_i$ is strictly descending in twisted length, we can repeat this step to obtain $v := u \ul t_k \ul s_k \ldots \ul s_1 < u \ul s_k \ldots \ul s_1 = w$.
		\todo
	\end{proof}
\end{lemm}

% \begin{prop}
% 	\typedlabel{prop:counterexample-simplification}
% 	Let $(W,S)$ be a Coxeter system and $K,T,S_1,S_2,S_3$ be like in \ref{ques:main}. Suppose we have $w \in W$ and $a_1,\ldots,a_n \in T \cup S_1$, $b_1,\ldots,b_n \in T \cup S_2$, $c_1,\ldots,c_n \in T \cup S_3$ with
% 	\begin{align*}
% 	w & = w_K \ul a_1 \ldots \ul a_n \\
% 	  & = w_K \ul b_1 \ldots \ul b_n \\
% 	  & = w_K \ul c_1 \ldots \ul c_n
% 	\end{align*}
% 	and \eqref{eq:main} does not hold for these three expressions, i.e. $w \notin w_K C_T$. Then there exist $t_1,\ldots,t_m \in T$ and $a'_1,\ldots,a'_{n-m} \in T \cup S_1$, $b'_1,\ldots,b'_{n-m} \in T \cup S_2$, $c'_1,\ldots,c'_{n-m} \in T \cup S_3$ such that
% 	\begin{align*}
% 		w \ul t_1 \ldots \ul t_m & = w_K \ul a'_1 \cdots \ul a'_{n-m} \\
% 								 & = w_K \ul b'_1 \cdots \ul b'_{n-m} \\
% 								 & = w_K \ul c'_1 \cdots \ul c'_{n-m}
% 	\end{align*}
% 	with $a'_{n-m},b'_{n-m},c'_{n-m} \notin T$.

% 	\begin{proof}
% 		Suppose at least one element of $a_n,b_n,c_n$ to be in $T$, for example $a_n \in T$. Then we can apply $\ul a_n$ to all three expressions. Since $\rho (w \ul a_n) < \rho(w)$ the \ref{prop:exchanged-property-for-twisted-expressions} yields
% 		\begin{align*}
% 			w \ul a_n & = w_K \ul a_1 \ldots \ul a_n \ul a_n = w_K \ul a_1 \ldots \ul a_{n-1} \\
% 					  & = w_K \ul b_1 \ldots \ul b_n \ul a_n = w_K \ul b_1 \ldots \ul {\hat b}_i \ldots \ul b_n \\
% 					  & = w_K \ul c_1 \ldots \ul c_n \ul a_n = w_K \ul c_1 \ldots \ul {\hat c}_j \ldots \ul c_n
% 		\end{align*}
% 		where $\hat \cdot$ means omission. The omission cannot occur within $w_K$ since all three expressions are still of same twisted length and in the first expression we can see, that $w_K \preceq w \ul a_n$ still holds. This step can be repeated until $w = w_K$ or $a_n,b_n,c_n \notin T$.
% 	\end{proof}
% \end{prop}

% \begin{lemm}
% 	\typedlabel{lemm:counterexample-simplification}
% 	A counterexample to \ref{ques:main} can only exist, if there
% 	is an element $u \in w C_T$ and three distinct generators $s_1,s_2,s_3 \in
% 	D_R(u)$ such that $u \ul s_i \notin w C_T$ for $i=1,2,3$.

% 	\begin{proof}
% 		According to \ref{prop:counterexample-simplification}.
% 	\end{proof}
% \end{lemm}

% Suppose we are in a situation like in \ref{ques:main} with $\rho(w) - \rho(w_K) = n$. If $n = 1$, then \ref{eq:main} holds, since $Wk(\theta)$ cannot contain triple edges. We proceed by induction on $n$. There are three twisted expressions
% \begin{align*}
% 	w & = w_K \ul a_1 \ldots \ul a_n \\
% 	  & = w_K \ul b_1 \ldots \ul b_n \\
% 	  & = w_K \ul c_1 \ldots \ul c_n
% \end{align*}
% with $a_i \in T \cup S_1$, $b_i \in T \cup S_2$ and $c_i \in T \cup S_3$. Since $\rho (w \ul a_n) < \rho(w)$ the \ref{prop:exchanged-property-for-twisted-expressions} yields
% \begin{align*}
% 	w \ul a_n & = w_K \ul a_1 \ldots \ul a_n \ul a_n = w_K \ul a_1 \ldots \ul a_{n-1} \\
% 			  & = w_K \ul b_1 \ldots \ul b_n \ul a_n = w_K \ul b_1 \ldots \ul {\hat b}_i \ldots \ul b_n \\
% 			  & = w_K \ul c_1 \ldots \ul c_n \ul a_n = w_K \ul c_1 \ldots \ul {\hat c}_j \ldots \ul c_n
% \end{align*}
% where $\hat \cdot$ means omission. The omission cannot occur within $w_K$ since all three expressions are still of same twisted length and in the first expression we can see, that $w_K \preceq w \ul a_n$ still holds. It is $\rho(w \ul a_n) - \rho(w_K) = n - 1$, hence by induction hypothesis it is $w \ul a_n \in w_K C_T$. Analogues we get $w \ul b_n, w \ul c_n \in w_K C_T$. If we assume at least one generator of $a_n,b_n,c_n$ to be in $T$, say $a_n \in T$, then $w = w \ul a_n \ul a_n \in w_K C_T$, too. Now suppose $a_n,b_n,c_n \notin T$. Then $a_n,b_n,c_n$ are pairwise different.