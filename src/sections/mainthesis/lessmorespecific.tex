%!TEX root = ../../../_main.tex
\subsection{Results in less and more specific cases}
\label{sec:sec:main-thesis-less-more-specific}

%\begin{exam}
%	Let $W = A_5$, $\theta = \id$ and $w = \ul s_1 \ul s_5 \ul s_3 \ul s_4 \ul s_2 \ul s_3$ as in Figure~\ref{fig:main-thesis-weakend-hypothesis-counterexample}. Denote the maximal element by $w_0$. Let $S_{12} = \{s_1,s_2\}$, $S_{23} = \{s_5,s_6\}$ and $S_{31} = \{s_1,s_5,s_6\}$. Then $w_0 \in wC_{S_i}$ for $i=1,2,3$ by $w_0 = w \ul s_2 \ul s_1 \ul s_2 = w \ul s_6 \ul s_5 \ul s_6 = w \ul s_6 \ul s_1 \ul s_5$, but $w_0 \notin wC_{S_{12} \cap S_{23} \cap S_{31}} = wC_{\emptyset} = \{w\}$.
%
%	\begin{figure}[ht]
%		\centering
%		%!TEX root = ../../_main.tex
\begin{tikzpicture}[scale=1,bend angle=10]
\newcommand{\xspace}{1}
\newcommand{\yspace}{1}
\tikzstyle{vertex}=[draw,thick,circle,minimum size=2mm,inner sep=0pt]
\tikzstyle{edge}=[->]
\tikzstyle{onesided}=[edge,dashed]
\tikzstyle{bothsided}=[edge]
\tikzstyle{unhighlighted}=[]
\tikzstyle{highlighted}=[ultra thick]
\definecolor{s1color}{RGB}{130,76,253}
\definecolor{s2color}{RGB}{76,253,78}
\definecolor{s3color}{RGB}{253,76,124}
\definecolor{s4color}{RGB}{76,176,253}
\definecolor{s5color}{RGB}{228,253,76}
\tikzstyle{s1}=[s1color]
\tikzstyle{s2}=[s2color]
\tikzstyle{s3}=[s3color]
\tikzstyle{s4}=[s4color]
\tikzstyle{s5}=[s5color]
\node[vertex,unhighlighted] (56) at (\xspace*-3.375,\yspace*9) {};
\node[vertex,unhighlighted] (57) at (\xspace*-2.625,\yspace*9) {};
\node[vertex,unhighlighted] (58) at (\xspace*-1.875,\yspace*9) {};
\node[vertex,unhighlighted] (59) at (\xspace*-1.125,\yspace*9) {};
\node[vertex,unhighlighted] (60) at (\xspace*-0.375,\yspace*9) {};
\node[vertex,unhighlighted] (61) at (\xspace*0.375,\yspace*9) {};
\node[vertex,unhighlighted] (62) at (\xspace*1.125,\yspace*9) {};
\node[vertex,unhighlighted] (63) at (\xspace*1.875,\yspace*9) {};
\node[vertex,highlighted] (64) at (\xspace*2.625,\yspace*9) {};
\node[vertex,unhighlighted] (65) at (\xspace*3.375,\yspace*9) {};
\node[vertex,unhighlighted] (66) at (\xspace*-1.875,\yspace*10.5) {};
\node[vertex,unhighlighted] (67) at (\xspace*-1.125,\yspace*10.5) {};
\node[vertex,highlighted] (68) at (\xspace*-0.375,\yspace*10.5) {};
\node[vertex,unhighlighted] (69) at (\xspace*0.375,\yspace*10.5) {};
\node[vertex,unhighlighted] (70) at (\xspace*1.125,\yspace*10.5) {};
\node[vertex,highlighted] (71) at (\xspace*1.875,\yspace*10.5) {};
\node[vertex,unhighlighted] (72) at (\xspace*-0.75,\yspace*12) {};
\node[vertex,highlighted] (73) at (\xspace*0,\yspace*12) {};
\node[vertex,highlighted] (74) at (\xspace*0.75,\yspace*12) {};
\node[vertex,highlighted] (75) at (\xspace*0,\yspace*13.5) {};
\draw[s5,bothsided,unhighlighted] (56) edge (66);
\draw[s3,bothsided,unhighlighted] (57) edge (66);
\draw[s4,onesided,unhighlighted] (57) edge (67);
\draw[s5,bothsided,unhighlighted] (58) edge (67);
\draw[s3,bothsided,unhighlighted] (58) edge (68);
\draw[s2,bothsided,unhighlighted] (59) edge (66);
\draw[s4,bothsided,unhighlighted] (59) edge (69);
\draw[s2,onesided,unhighlighted] (60) edge (67);
\draw[s3,bothsided,unhighlighted] (60) edge (69);
\draw[s5,bothsided,unhighlighted] (61) edge (69);
\draw[s2,bothsided,unhighlighted] (61) edge (70);
\draw[s1,bothsided,unhighlighted] (62) edge (66);
\draw[s4,bothsided,unhighlighted] (62) edge (70);
\draw[s1,bothsided,unhighlighted] (63) edge (67);
\draw[s3,bothsided,unhighlighted] (63) edge (71);
\draw[s1,bothsided,highlighted,bend right] (64) edge (68);
\draw[s4,bothsided,highlighted,bend left] (64) edge (68);
\draw[s2,bothsided,highlighted,bend right] (64) edge (71);
\draw[s5,bothsided,highlighted,bend left] (64) edge (71);
\draw[s1,bothsided,unhighlighted] (65) edge (69);
\draw[s4,bothsided,unhighlighted] (66) edge (72);
\draw[s3,bothsided,unhighlighted] (67) edge (73);
\draw[s5,bothsided,highlighted] (68) edge (73);
\draw[s2,bothsided,highlighted] (68) edge (74);
\draw[s2,bothsided,unhighlighted] (69) edge (72);
\draw[s1,bothsided,unhighlighted,bend right] (70) edge (72);
\draw[s5,bothsided,unhighlighted,bend left] (70) edge (72);
\draw[s3,onesided,unhighlighted] (70) edge (74);
\draw[s1,bothsided,highlighted] (71) edge (73);
\draw[s4,bothsided,highlighted] (71) edge (74);
\draw[s3,onesided,unhighlighted] (72) edge (75);
\draw[s2,bothsided,highlighted,bend right] (73) edge (75);
\draw[s4,bothsided,highlighted,bend left] (73) edge (75);
\draw[s1,bothsided,highlighted,bend right] (74) edge (75);
\draw[s5,bothsided,highlighted,bend left] (74) edge (75);
\end{tikzpicture}
%		\caption{Upper end of Hasse diagram of $Wk(A_5,\id)$}
%		\label{fig:main-thesis-weakend-hypothesis-counterexample}
%	\end{figure}
%\end{exam}

In this section we investigate some results and examples, in situations that are less or more specific than the situation from \ref{ques:main}.

\begin{exam}
	\typedlabel{exam:trivial-counterexample-for-arbitary-sets-of-generators}
	Let $W = A_3$ and $\theta$ be the Coxeter system autmorphism swapping $s_1$ and $s_3$. Then $e \ul s_1 = e \ul s_3$ but $e \ul s_1 \notin eC_{\{s_1\} \cap \{s_1\} \cap \{s_2\}} = eC_\emptyset = \{e\}$.
\end{exam}

Such a trivial counterexample like in \ref{exam:trivial-counterexample-for-arbitary-sets-of-generators} can not occur in the situation from \ref{ques:main}.

\begin{prop}
	Consider the situation from \ref{ques:main}. Let $w,v \in \ti{\theta}$ with $\rho(v) - \rho(w) = 1$ and let $v \in w C_{S_{ij}}$ for $1 \leq i < j \leq 3$. Then we have $v \in wC_T$.

	\begin{proof}
		There are at most two (not necessarily distinct) $s,t \in S$ with $w \ul s = v$ and $w \ul t = v$. Each set $S_{12},S_{23},S_{31}$ must at least contain $s$ or $t$, hence $s$ or $t$ is in at least two sets, say $s \in S_{12},S_{23}$. Hence $s \in S_1,S_2,S_3$ and therefore $v \in wC_T$.
	\end{proof}
\end{prop}

A hypothesis, that is much stronger than \ref{ques:main}, reads $wC_I \cap wC_J = wC_{I \cap J}$. If this would be true, \ref{ques:main} could be concluded immediately. Unfortunately it proves to be false. Again, double-edges yield a simple counterexample.

\begin{exam}
	\typedlabel{exam:cap-of-residuums}
	Let $w \in \ti{\theta}$ and $s,t$ two distinct generators with $w \ul s = w \ul t = v$. Then $wC_{\{s\}} \cap wC_{\{t\}} = \{w,v\} \neq \{w\} = wC_{\emptyset} = wC_{\{s\} \cap \{t\}}$.
\end{exam}

\begin{prop}
	\typedlabel{prop:s1-in-s2-main-special-case}
	Consider the situation from \ref{ques:main}. Suppose one set of $S_1,S_2,S_3$ is contained in another. Then
	$$ \forall \ 1 \leq i < j \leq 3 : v \in w C_{S_{ij}} \quad \Rightarrow \quad v \in w C_T. $$

	\begin{proof}
		Without loss of generality let $S_1 \subset S_2$. Then we have $S_{12} = S_1$. By this we get the identity
		$$ T = S_1 \cap S_2 \cap S_3 = S_{12} \cap S_3 = S_1 \cap S_3. $$
		Hence $v \in w C_T = w C_{S_{31}}$.
	\end{proof}
\end{prop}