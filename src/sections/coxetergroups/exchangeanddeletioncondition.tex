%!TEX root = ../../../_main.tex
\subsection{Exchange and Deletion Condition}
\label{sec:coxeter-groups-exchange-deletion-condition}

We now obtain a way to get a reduced expression of an arbitrary element $s_1 \cdots s_r = w \in W$. But first we define what a reflection is. Any element $w \in W$ that is conjugated to an generator $s \in S$ is called \defword{reflection}. Hence the set of all reflections in $W$ is
$$ T = \bigcup_{w \in W} wSw^{-1}. $$

\begin{theo}[Strong Exchange Condition]
	%\cite[Theorem 5.8]{humphreys:coxeter}
	\namedlabel{theo:strong-exchange-condition}
	Let $(W,S)$ be a Coxeter system, $w \in W$ an arbitrary element and ${s_1 \cdots s_r = w}$ with $s_i \in S$ a not necessarily reduced expression for $w$. For each reflection $t \in T$ with $l(wt) < l(w)$ there exists an index $i$ for which $wt = s_1 \cdots \hat s_i \cdots s_r$, where $\hat s_i$ means omission. In case we started from a reduced expression, then $i$ is unique.

	\begin{proof}
		See \cite[Theorem 5.8]{humphreys:coxeter}.
	\end{proof}
\end{theo}

The \ref{theo:strong-exchange-condition} can be weaken when insisting on $t \in S$ to receive the following corollary.

\begin{coro}[Exchange Condition]
	\namedlabel{coro:exchange-condition}
	Let $(W,S)$ be a Coxeter system, $w \in W$ an arbitrary element and ${s_1 \cdots s_r = w}$ with $s_i \in S$ a not necessarily reduced expression for $w$. For each generator $s \in S$ with $l(ws) < l(w)$ there exists an index $i$ for which $ws = s_1 \cdots \hat s_i \cdots s_r$, where $\hat s_i$ means omission.

	\begin{proof}
		Directly from \ref{theo:strong-exchange-condition}.
	\end{proof}
\end{coro}

The \ref{coro:exchange-condition} immediately yields another corollary for Coxeter groups:

\begin{coro}[Deletion Condition]
	\namedlabel{coro:deletion-condition}
	Let $(W,S)$ be a Coxeter system, $w \in W$ and $w = s_1 \cdots s_r$ with $s_i \in S$ a unreduced expression of $w$. Then there exist two indices $i,j \in \{1,\cdots,r\}$ with $i < j$, such that $w = s_1 \cdots \hat s_i \cdots \hat s_j \cdots s_r$, where $\hat s_i$ and $\hat s_j$ mean omission.

	\begin{proof}
		Since the expression is unreduced there must be an index $j$ for that the twisted length shrinks. That means for $w' = s_1 \cdots s_{j-1}$ is $l(w' s_j) < l(w')$. Using the \ref{coro:exchange-condition} we get $w' s_j = s_1 \cdots \hat s_i \cdots s_{j-1}$ yielding $w = s_1 \cdots \hat s_i \cdots \hat s_j \cdots s_r$.
	\end{proof}
\end{coro}

This corollary is called \defword{Deletion Condition} and allows us to reduce expressions, i.e. to find a subexpression that is reduced. Due to the Deletion Condition any unreduced expression can be reduced by omitting a even number of generators (we just have to apply the Deletion Condition inductively).

The \ref{theo:strong-exchange-condition}, the \ref{coro:exchange-condition} and the \ref{coro:deletion-condition}, are some of the most powerful tools when investigating properties of Coxeter groups. We can use the second to prove a very handy property of Coxeter groups. The intersection of two parabolic subgroups is again a parabolic subgroup.

\begin{defi}
	\typedlabel{defi:parabolic-subgroup}
	Let $(W,S)$ be a Coxeter system. For a subset of generators $I \subset S$ we call the subgroup $W_I \leq W$ that is generated by the elements in $I$ with the corrosponding relations a \defword{parabolic subgroup} of $W$.
\end{defi}

\begin{lemm}
	Let $(W,S)$ be a Coxeter system and $I,J \subset S$ two subsets of generators. Then ${W_I \cap W_J} = W_{I \cap J}$.

	\begin{proof}
		Let $w \in W_{I \cap J}$. Then $w \in W_I$ and $w \in W_J$. To show the other inclusion we induce over the length $r$. For $r = 0$ we have $w = e$ and so $w \in W_{S'}$ for any $S' \subset S$. So suppose we have proven the assumption for all lengths up to $r-1$. Let $w \in W_I \cap W_J$ with $l(w) = r$. Then we have two reduced expressions $w = s_1 \cdots s_r = t_1 \cdots t_r$ with $s_i \in I$ and $t_i \in J$. By applying $s_r$ from the right we get $w s_r = s_1 \cdots s_{r-1} = t_1 \cdots t_r s_r$. The expression $t_1 \cdots t_r s_r$ is of length $r-1$, so \ref{coro:exchange-condition} yields $w s_r = s_1 \cdots s_{r-1} = t_1 \cdots \hat t_i \cdots t_r$, hence $w s_r \in W_I \cap W_J$. Due to induction we know that $w s_r \in W_{I \cap J}$. \todo
	\end{proof}
\end{lemm}